\section{Введение}



Выделяют два основных режима течения жидкости в круглых трубах --- ламинарный и турбулентный. Режим течения определяется безразмерной комбинацией параметров, называемой числом Рейнольдса. Как правило, если оно ниже некоторого критического значения, то реализуется ламинарный режим течения, если выше --- то турбулентный. В круглых трубах критическое значения Re, введенного через расходную скорость, диаметр трубы и кинематическую вязкость, близко к 2000. В тоже время, было показано, что организуя плавный вход жидкости в трубу и уменьшая уровень возмущений в потоке, можно удержать течения ламинарным при значительно больших числах Рейнольдса. Например, в экспериментах ... удалось сохранить течение ламинарным при $Re = 10^6$. Математически этот факт находит свое выражение в том, что течение Пуазейля устойчиво к малым возмущениям. Течением Пуазейля называется аналитическое решение, соответствующее ламинарному течения, установившемуся на удалении от входа в трубу. В нем все частицы жидкости двигаются строго по прямым линиям. 


Во многих дрвгих сдвиговых течениях наблюдается аналогичная ситуация. Ламинарное течение в плоском канале Пуазейля оказывается условиевым при всех числах Рейноьдса. Ламинарное течение в плоском канале Куэтта , хотя и теряет линейную устойчивость, делает это при значительно больших числах Рейнольсда, чем то, при котором происходит переход к турбелнтности. 


Несмотря на то, что достаточно малые возмущения ламинарного течения затухают, возмущения достаточно большой амплитуды могут вызвать переход к турбулентности. В таких условиях существуют возмущения некоторой граничной амплитуды, которые не затухнут, не вызовут переход к турбулентности, а так и останутся на границе. В фазовом пространстве эта граница называется сепаратрисой и отделает области притяжения решений, соответствующих ламинарному и турбулентному режимам течения. Решения, лежищие на сепаратрисе, каким-то образом эволючионируют на ней. Приближаясь с течение времени к некотором режиму. 

Хотя каждое течение, лежащее на сепаратрисе, неустойчиво. 


Как было показано в работе ..., решение на сепаратрисе воспроизводит основные особенности турбулентного течения, но имеет более простою форму и динамику. Решение на сепарарисе в кгурлых трубах впервые былопосчитано в работах ... . Если расчетная область имеет достаточную протяженность, то решение на сепратарисе оказывается локализованным в пространстве, как и турбулентность при переходных числах Рейнольдса. Причем в значительно более широком диапазоне числе Рейнольдса. В работе ... было показано, что при наложении дополнительных условий симметрии на течение, привеенных в постановке задачи, решение на сепаратрисе в трубе оказывается не просто локализованным, но периодическим по времени в подвижной системе отсчета, связанной ассоциированным с ним порывом. Мы находим это решение уникальным, в нем в некотором виде присуствует механизм самоподдрежания и механизм, ведущий к пространственной локализации, но в отличии от турбулентного порыва, оно может быть изучено во всех деталях. Полученные при его изучении выводы могут оказаться полезны при исследовании турбулнетного поырва. 


Как было показано во многих работах, в ограниченной области турбулентное течение имеет ограниченное время жизни. По крайней мере это верно при переходных числах Рейнольдса. В связи с этим обстоятельством определение сепаратрисе для численных расчетов теряет смысл, так как с течение времени любое решение выйдет на ламинарный режим. Тем не менее, на практике не составляет труда. Так как оно провид на турбулентном течении значительно большее время. 




