\section{Метод поиска решения на сепаратрисе}


Для того, чтобы построить решение на сепаратрисе, необходимо выделить два начальных поля скорости, одно из которых выходит на ламинарный режим течения $\v_{lam}$, второе --- на турбулентный $\v_{turb}$. Наиболее очевидным является выбор в качестве $\v_{lam}$ течения Пуазейля, а в качестве $\v_{turb}$ --- мгновенного поля скорости, соответствующего турбулентному течения, но возможны и другие вариант. Так, в качестве $\v_{lam}$ можно выбрать осесимметричную составляющую поля $\v_{turb}$.  Известно, что все осесимметричные возмущения течения Пуазейля затухают, и такое поле скорости соответствует ламинарному режиму течения. Или в качестве $\v_{lam}$ может выступать поле скорости, постоянное по всему сечению трубы, в котором $v_x = 0.5$, $v_r = v_\theta = 0$. 


По полям скорости $\v_{lam}$ и $\v_{turb}$ может быть составлено новое начальное поле скорости $\v_{\alpha}$, определяемое параметром $\alpha \in [0,1]$, по следующей формуле:
\begin{equation}
\v_{\alpha} = (1 - \alpha)\v_{lam} + \alpha \v_{turb}
\end{equation}
Если $\alpha = 0$, то уже в первый момент времени реешнеи оказывается ламинарным. Если $\alpha = 1$, то турбулентным. В общем наблюдая за эволючией поля скорости при каждом значении $\alpha$, можно сопоставить ему ламинарный или турбулентный режим течения. Существует некоторое критическое значения $\alpha^*$, при котором происходит смена режима течения. Точно найти его значение не представляется возможным, но можно найти сколь угодно близкое к нему приближение например, методом деления отрезка пополам. 


Метод деления отрезка пополам состоит в том, что 



Решение на сепаратрисе ищется в круглой трубе при Re=2200. На течение накладываются дополнительные условия отражения около радиального сечения и $\pi$-периодичности по углу. В качестве $\v_{lam}$ выбератеся течение Пуазейля. В качестве $\v_{turb}$ --- некоторое мгновенное поле скорости, соответствующее турбулентному режиму течения. На рис. изображена серия последовательных приближений к ламинарному течения. Ламинарному решению соответствует нулевая амплитуда. 

Часть решений ламинариизуется, часть турбулезуется, но между ними вырисовывается новое решение, которое и является решением на сепаратрисе. На грацике лишь в большом приближении видно, что оно периодическое. 



Каждое невое приближение позволяет продержаться на решении новых 10 ед по времени. Двойная точность вещественного числа позволяет сделать 100 итераций. Таким образом в сумме удается продержаться 1000 ед. по времени на сепаратрисе. Исключая ериод начальной эволюции в 300 единиц, на периодическом решении решение держится 10 0ъоаъ ВОптпэ 


Первоначальный расчет выполнялся при $c_f = 0.5$. С такой скоростью перемещается турбулентный порыва. Но решение на сепаратрисе перемещается быстрее. Так как мы наблюдали за интегральной характеристикой, мы игнорировали положение порыва вдоль трубы и это важно. Но теперь нужно найти систему отсчета, которая движется со скорость порыва, в окторой он покоится. Если испольовать решение на сепратрисе, найденное выше, то в новой системе отсчета оно разваливатеся. 


Итерации итераций. Поиск связанной с порывом системы отсчета. 

Далее будет описан метод Ньютона. Для него такое приближение оказывается достаточным, чтобы найти точное решение. Можно использваоть его 










