
Рассматривается движение вязкой несжимаемой жидкости в прямой трубе круглого сечения. Движение вызывается внешним градиентом давления, направленным вдоль трубы. Внешние массовые силы отсутствуют. Жидкость считается однородной, то есть её плотность $\rho$ и коэффициент динамической вязкости $\mu$ постоянны во всем объеме трубы. 


Движение жидкости описывается уравнениями Навье-Стокса и неразрывности:
\begin{equation}\label{NS_dynamic_eq}
\rho\pd{\v}{t} = - \rho (\v, \nabla) \v - \nabla P + \mu \nabla^2 \v,
\end{equation}
\begin{equation}\label{incompres_eq}
\div \v = 0. 
\end{equation}
Здесь $\v$ --- векторное поле скорости, $P$ ---  скалярное поле давления. Уравнения решаются в цилиндрической системе координат $(x,r,\theta)$, $t$ --- время.


На стенках трубы ставится условие прилипания $\v = 0$, вдоль потока --- условие периодичности с периодом $L_x$. Давление $P$, непосредственно, периодическим не является. Оно представляется в виде суммы периодической $p$ и линейной вдоль трубы $Dx$ составляющих. Подстановка $P = p + Dx$ в \eqref{NS_dynamic_eq} дает уравнение, в котором условию периодичности удовлетворяют все неизвестные:
\begin{equation}\label{NS_D_dynamic_eq}
\rho \pd{\v}{t} =  - \rho (\v, \nabla) \v - \i D - \nabla p + \mu \nabla^2 \v.
\end{equation}
Здесь $\i$ --- направление вдоль трубы. Величина $D$ представляет собой внешний градиент давления и определяется из условия постоянства расхода. Поставленная таким образом задача имеет единственное решение для каждого начального поля скорости $\v_0$. 


При наличии внешних потенциальных сил $\F$ уравнение \eqref{NS_D_dynamic_eq} остается справедливо, но $D$ в этом случае является суммой внешнего градиента давления и средней вдоль трубы составляющей $F_x$. Не вошедшая в $D$ часть $\F$ потенциальна. Её потенциал в сумме с периодической вдоль трубы частью давления составляет $p$. Внешние силы $\F$ предполагаются периодическими вдоль потока.


Приведенная постановка задачи имеет аналитическое решение, соответствующее ламинарному течению, называемому течением Пуазейля. В соответствии с этим решением жидкие частицы двигаются строго по прямым, параллельным оси трубы, а поперечная компонента их скорости равна нулю. Это решение имеет следующий вид:
\begin{equation}
v_x = U (1 - (r / R)^2), 
v_r = v_\theta = 0. 
\end{equation}
Здесь величина $U = - D R^2\mu^{-1}/4 $ определяется по градиенту давления $D$ и соответствует максимальной скорости в ламинарном течении, которую оно достигает на оси трубы. Несложно показать, что средняя скорость течения $U_q$, вычисленная как отношение расхода к площади сечения трубы, равна $U/2$. 


В работе все вычисления производятся в безразмерном виде. В качестве масштабов выступают радиус трубы $R$, максимальная скорость в ламинарном течении $U$ и плотность жидкости $\rho$. Переход к безразмерным единицам измерения, обозначенным штрихами, выполняется по формулам: 
\begin{equation}
R x' = x,  
R r' = r, 
RU^{-1} t' = t, 
U\v' = \v , 
\rho U^2 p' = p, 
\rho U^2 R^{-1} D' = D, 
R L'_x = L_x.
\end{equation}
Постановка задачи в безразмерных единицах измерения принимает следующий вид (штрихи опущены): 
\begin{equation}\label{NS_Re_eq}
\pd{\v}{t} = - (\v, \nabla) \v - \i D - \nabla p + \frac{1}{\Re} \nabla^2 \v,
\end{equation}
\begin{equation}\label{incompres1_eq}
\div \v = 0,
\end{equation}
\begin{equation}\label{bc_eq}
\v = 0 \text{ при } r = 1,
\end{equation}
\begin{equation}\label{periodicity_eq}
(\v, p)(x,r,\theta,t) = (\v, p)(x+L_x,r,\theta,t)
\end{equation}
Здесь $\Re = \rho R U / \mu$ --- число Рейнольдса, один из двух параметров системы. Вторым параметром является $L_x$. Величина $D$ подбирается из условия постоянства расхода так, что $U_q = 1/2$ в безразмерных единицах. В случае ламинарного течения она выражается в явном виде: $D = - 4 \Re^{-1}$. Ламинарное течение в безразмерном виде задается более простым выражением: 
\begin{equation}
v_x = 1 - r^2, v_r = v_\theta = 0.
\end{equation}


Многие авторы вводят число Рейнольдса иначе, через расходную скорость $U_q$ и диаметр трубы $D$. Несложно показать, что значение числа Рейнольдса, введенного таким образом, совпадает со значением числа Рейнольдса, введенного выше. Однако, например, единица измерения времени $DU_q^{-1}$ в этом случае увеличивается в 4 раза.

\begin{comment}
В цилиндрической системе координат уравнения \eqref{NS_Re_eq} и \eqref{incompres1_eq} имеют вид:
\begin{equation*}
\pd{v_x}{t}  = - v_x \pd{v_x}{x} - v_r \pd{v_x}{r} - \frac{v_\theta}{r} \pd{v_x}{\theta} - D - \pd{p}{x} + \frac{1}{Re}  \left( \pdd{v_x}{x} + \pdd{v_x}{r} + \frac{1}{r^2}\pdd{v_x}{\theta} + \frac{1}{r}\pd{v_x}{r} \right)
\end{equation*} 
\begin{equation*}
\pd{v_r}{t}  = - v_x \pd{v_r}{x} - v_r \pd{v_r}{r} - \frac{v_\theta}{r} \pd{v_r}{\theta} - \frac{v^2_\theta}{r}  - \pd{p}{r} + \frac{1}{Re}  \left( \pdd{v_r}{x} + \pdd{v_r}{r} + \frac{1}{r^2}\pdd{v_r}{\theta} + \frac{1}{r}\pd{v_r}{r} - \frac{2}{r^2}\pd{v_\theta}{\theta} - \frac{v_r}{r^2} \right)
\end{equation*}
\begin{equation*}
\pd{v_\theta}{t}  = - v_x \pd{v_\theta}{x} - v_r \pd{v_\theta}{r} - \frac{v_\theta}{r} \pd{v_\theta}{\theta} + \frac{v_r v_\theta}{r}  - \frac{1}{r}\pd{p}{\theta} + \frac{1}{Re}  \left( \pdd{v_\theta}{x} + \pdd{v_\theta}{r} + \frac{1}{r^2}\pdd{v_\theta}{\theta} + \frac{1}{r}\pd{v_\theta}{r} - \frac{2}{r^2}\pd{v_r}{\theta} - \frac{v_\theta}{r^2} \right)
\end{equation*}
\begin{equation*}
\pd{v_x}{x} + \pd{v_r}{r} + \frac{1}{r} \pd{v_\theta}{\theta} + \frac{v_r}{r} = 0
\end{equation*}


При нахождении решения на сепаратрисе на течение накладываются дополнительные условия симметрии, а именно условие симметрии относительно радиального сечения $\theta = 0$:
\begin{equation}\label{th0_reflect_eq}
(v_x, v_r, v_\theta)(x, r, \theta, t) = (v_x, v_r, -v_\theta)(x, r, -\theta, t).
\end{equation}
Также ставится условие периодичности по углу с периодом $\pi n$: 
\begin{equation}\label{th_period_eq}
\v(x, r, \theta, t) = \v(x, r, \pi n + \theta, t).
\end{equation}
В случае $n = 2$ условие \eqref{th_period_eq} является естественным. При больших целых $n$ решение остается реализуемым с математической точки зрения. В расчетах могут быть получены решения при произвольных действительных $n > 0$, что иногда имеет смысл. 


Несложно показать, что течение, удовлетворяющее условиям \eqref{th0_reflect_eq} и \eqref{th_period_eq}, удовлетворяет также условию симметрии относительно плоскости $\theta = \pi n / 2$:
\begin{equation}
(v_x, v_r, v_\theta)(x, r, \pi n/2 + \theta, t) = (v_x, v_r, -v_\theta)(x, r, \pi n / 2 - \theta, t).
\end{equation}
Оно позволяет сформулировать задачу в области, располагающейся при $0 < \theta < \pi n/2$, с условием отражения на стенках в угловом направлении. 
\end{comment}


В расчетах часто возникает необходимость перехода в подвижную систему координат. Выполняя переход, удобно сохранить в качестве тела отсчета, относительно которого определяется скорость движения жидкости, стенку трубы. В таком случае граничные и начальные условия не зависят от скорости перемещения системы координат, а уравнение движения \eqref{NS_Re_eq} принимает вид: 
\begin{equation}\label{NS_cf_eq}
\pd{\v}{t} = c_f \pd{\v}{x} - (\v, \nabla) \v - \i D - \nabla p + \frac{1}{\Re} \nabla^2 \v. 
\end{equation}
Здесь $c_f$ --- скорость перемещения системы координат относительно стенки трубы. Она может быть функцией времени. Возникшее в уравнении новое слагаемое отвечает за перемещение решения вдоль трубы со скоростью $-c_f$ относительно начала координат. Другие уравнения в постановке задачи при переходе в подвижную систему координат не меняются. 


Приведенная постановка задачи является классической при исследовании характеристик развитого турбулентного течения и применяется многими авторами \cite{}. Она позволяет воспроизводить течение на удалении от входа в трубу, где режим течения устанавливается. Условие периодичности вдоль трубы освобождает от необходимости устанавливать условия на входе и выходе из нее. В тоже время, влияние этого условия на течение можно свести к минимуму, увеличивая $L_x$. Так как корреляция турбулентных пульсаций падает по мере увеличения расстояния между точка, в которых она вычисляется, на движение в конкретной точке пространства оказывает существенное влияние движение лишь из некоторой её окрестности. Таким образом, хотя в точках, удаленных от выделенной на расстояние $L_x$, корреляция равна единице, при достаточно больших значениях $L_x$ движение в этих точках не оказывает существенного влияния на движение в выделенной. 





