
Рассматривается движение вязкой несжимаемой жидкости в прямой трубе круглого сечения. Движение вызывается внешним градиентом давления, направленным вдоль потока. Внешние массовые силы отсутствуют. Жидкость считается однородной, то есть её свойства постоянны во всем объеме трубы. 


Движение жидкости описывается уравнениями Навье-Стокса и неразрывности:
\begin{equation}\label{NS_dynamic_eq}
\rho\pd{\v}{t} = - \rho (\v, \nabla) \v - \nabla P + \mu \nabla^2 \v,
\end{equation}
\begin{equation}\label{incompres_eq}
\div \v = 0. 
\end{equation}
Здесь $\v$ --- векторное поле скорости, $P$ ---  скалярное поле давления, $\rho$ --- плотность, $\mu$ --- динамическая вязкость, $t$ --- переменная времени. Уравнения решаются в цилиндрической системе координат $(x,r,\theta)$.


На стенках трубы ставится условие прилипания $\v = 0$, вдоль потока --- условие периодичности с периодом $L_x$. Давление $P$, непосредственно, периодическим не является. Оно представляет собой сумму периодической $p$ и линейной вдоль трубы $Dx$ составляющих. Подстановка $P = p + Dx$ в \eqref{NS_dynamic_eq} дает:
\begin{equation}\label{NS_D_dynamic_eq}
\rho \pd{\v}{t} =  - \rho (\v, \nabla) \v - \i D - \nabla p + \mu \nabla^2 \v.
\end{equation}
Здесь $\i$ --- направление вдоль трубы. Величина $D$ представляет собой внешний градиент давления и определяется из условия постоянства расхода. 


При наличии внешних потенциальных сил $\F$ уравнение \eqref{NS_D_dynamic_eq} также справедливо, но $D$ в этом случае является суммой внешнего градиента давления и средней величины $F_x$. Не вошедшая в $D$ часть $\F$ потенциальна. Её потенциал в сумме с периодической частью давления составляет $p$. Внешние силы $\F$ предполагаются периодическими вдоль потока.


Приведенная постановка задачи имеет аналитическое решение, соответствующее ламинарному течению. Оно называется течением Пуазейля и имеет следующий вид:
\begin{equation}
v_x = U (1 - (r / R)^2), 
v_r = v_\theta = 0
\end{equation}
Здесь величина $U = - D R^2\mu^{-1}/4 $ определяется по градиенту давления $D$ и соответствует максимальной скорости в ламинарном течении, которую оно достигает на оси трубы. Несложно показать, что средняя скорость течения $U_q$, вычисленная как отношение расхода к площади сечения трубы, равна $U/2$. 


Имеет смысл привести уравнения к безразмерному виду. В качестве единиц изменения выступают радиус трубы $R$, удвоенная расходная скорость, равная максимальной скорости в ламинарном течении, $U$ и плотность жидкости $\rho$. Переход к безразмерным единицам измерения, обозначенным штрихами, выполняется по формулам: 
\begin{equation}
R x' = x,  
R r' = r, 
RU^{-1} t' = t, 
U\v' = \v , 
\rho U^2 p' = p, 
\rho U^2 R^{-1} D' = D, 
R L'_x = L_x
\end{equation}
В безразмерных единицах измерения постановка задачи принимает вид (штрихи упущены): 
\begin{equation}\label{NS_Re_eq}
\pd{\v}{t} = - (\v, \nabla) \v - \i D - \nabla p + \frac{1}{\Re} \nabla^2 \v,
\end{equation}
\begin{equation}\label{incompres1_eq}
\div \v = 0,
\end{equation}
\begin{equation}\label{bc_eq}
\v = 0 \text{ при } r = 1,
\end{equation}
\begin{equation}\label{periodicity_eq}
\v(x,r,\theta,t) = \v(x+L_x,r,\theta,t)
\end{equation}
\begin{equation}\label{periodicity1_eq}
p(x,r,\theta,t) = p(x+L_x,r,\theta,t)
\end{equation}
Здесь $\Re = \rho R U / \mu$ --- число Рейнольдса, один из двух параметров системы. Вторым параметром системы является $L_x$. Величина $D$ подбирается из условия постоянства расхода так, что $U_q = 1/2$ в безразмерных единицах. В случае ламинарного течения она выражается в явном виде: $D = - 4 \Re^{-1}$. Ламинарное течение в безразмерном виде задается более простым выражением: 
\begin{equation}
v_x = 1 - r^2, v_r = v_\theta = 0
\end{equation}


%\begin{comment}
В цилиндрической системе координат уравнения \eqref{NS_Re_eq} и \eqref{incompres1_eq} имеют вид:
\begin{equation*}
\pd{v_x}{t}  = - v_x \pd{v_x}{x} - v_r \pd{v_x}{r} - \frac{v_\theta}{r} \pd{v_x}{\theta} - D - \pd{p}{x} + \frac{1}{Re}  \left( \pdd{v_x}{x} + \pdd{v_x}{r} + \frac{1}{r^2}\pdd{v_x}{\theta} + \frac{1}{r}\pd{v_x}{r} \right)
\end{equation*} 
\begin{equation*}
\pd{v_r}{t}  = - v_x \pd{v_r}{x} - v_r \pd{v_r}{r} - \frac{v_\theta}{r} \pd{v_r}{\theta} - \frac{v^2_\theta}{r}  - \pd{p}{r} + \frac{1}{Re}  \left( \pdd{v_r}{x} + \pdd{v_r}{r} + \frac{1}{r^2}\pdd{v_r}{\theta} + \frac{1}{r}\pd{v_r}{r} - \frac{2}{r^2}\pd{v_\theta}{\theta} - \frac{v_r}{r^2} \right)
\end{equation*}
\begin{equation*}
\pd{v_\theta}{t}  = - v_x \pd{v_\theta}{x} - v_r \pd{v_\theta}{r} - \frac{v_\theta}{r} \pd{v_\theta}{\theta} + \frac{v_r v_\theta}{r}  - \frac{1}{r}\pd{p}{\theta} + \frac{1}{Re}  \left( \pdd{v_\theta}{x} + \pdd{v_\theta}{r} + \frac{1}{r^2}\pdd{v_\theta}{\theta} + \frac{1}{r}\pd{v_\theta}{r} - \frac{2}{r^2}\pd{v_r}{\theta} - \frac{v_\theta}{r^2} \right)
\end{equation*}
\begin{equation*}
\pd{v_x}{x} + \pd{v_r}{r} + \frac{1}{r} \pd{v_\theta}{\theta} + \frac{v_r}{r} = 0
\end{equation*}
%\end{comment}


При нахождении решения на сепаратрисе на течение накладываются дополнительные условия симметрии, а именно условие симметрии относительно радиального сечения $\theta = 0$:
\begin{equation}\label{th0_reflect_eq}
(v_x, v_r, v_\theta)(x, r, \theta, t) = (v_x, v_r, -v_\theta)(x, r, -\theta, t).
\end{equation}
Также ставится условие периодичности по углу с периодом $\pi n$: 
\begin{equation}\label{th_period_eq}
\v(x, r, \theta, t) = \v(x, r, \pi n + \theta, t).
\end{equation}
В случае $n = 2$ условие \eqref{th_period_eq} является естественным. При больших целых $n$ решение остается реализуемым с математической точки зрения. В расчетах могут быть получены решения при произвольных действительных $n > 0$, что иногда имеет смысл. 


Несложно показать, что течение, удовлетворяющее условиям \eqref{th0_reflect_eq} и \eqref{th_period_eq}, удовлетворяет также условию симметрии относительно плоскости $\theta = \pi n / 2$:
\begin{equation}
(v_x, v_r, v_\theta)(x, r, \pi n/2 + \theta, t) = (v_x, v_r, -v_\theta)(x, r, \pi n / 2 - \theta, t).
\end{equation}
Оно позволяет сформулировать задачу в области, располагающейся при $0 < \theta < \pi n/2$, с условием отражения на стенках в угловом направлении. 


В расчетах возникает необходимость перехода в подвижную систему отсчета. 


Задача полостью определена, нужны только начальные условия. 







