
\chapter{Метод решения}

\section{Введение}




\section{Математическая постановка задачи}

Рассматривается движение вязкой несжимаемой жидкости в прямой трубе круглого сечения. Жидкость считается однородной, то есть её плотность $\rho$ и вязкость $\mu$ во всем объеме трубы постоянны. Движение вызывается внешним градиентом давления, направленным вдоль трубы.


Поле скорости $\v(\x,t)$ удовлетворяет уравнению несжимаемости
\begin{equation} \label{eq0}
\nabla \cdot \v = 0.
\end{equation}

Движение жидкости описывается уравнениями Навье-Стокса, которые для несжимаемой жидкости имеют вид
\begin{equation} \label{NSeq_dynamic}
\pd{\v}{t} = - (\v, \nabla) \v - \frac{1}{\rho}\nabla \Pi + \frac{\mu}{\rho} \nabla^2 \v.
\end{equation}
Здесь $\Pi = \Pi(\x,t)$ -- поле давления. Уравнения решаются в цилиндрической системе координат $(x,r,\theta)$, где $x$ -- продольное, $r$ -- радиальное и $\theta$ -- угловое направления, $t$ -- время. 


На твердой стенке трубы радиуса $R$ ставятся условия прилипания
\begin{equation} \label{bc0}
\v = 0 \text{ при } r = R.
\end{equation}
В направлении движения устанавливаются условие периодичности с периодом $L_x$. Давление $\Pi$, непосредственно, периодическим не является. Оно представляется в виде суммы периодической $p$ и линейной вдоль трубы $Dx$ составляющих. Подстановка $\Pi = p + Dx$ в \eqref{NSeq_dynamic} дает уравнение
\begin{equation} \label{NSeq_periodic}
\pd{\v}{t} = - (\v, \nabla) \v - \frac{D}{\rho}\i - \frac{1}{\rho}\nabla p + \frac{\mu}{\rho} \nabla^2 \v.
\end{equation}
Здесь $\i$ -- единичный орт, направленный вдоль трубы. В уравнении \eqref{NSeq_periodic} все переменные удовлетворяют условию периодичности вдоль трубы
\begin{equation} \label{bc1}
[\v, p](x,r,\theta,t) = [\v, p](x+L_x,r,\theta,t).
\end{equation}

Величина $D$ представляет собой внешний градиент давления, которые определяется из условия постоянства расхода жидкости, протекающей через трубу. В случае наличия внешних потенциальных сил $\F = \nabla U$, периодических вдоль потока, постановка задачи не меняется. За $D$ следует обозначить сумму внешнего градиента давления и средней вдоль трубы составляющей $F_x$. Не вошедшая в $D$ часть $\F$ потенциальна. Её потенциал в сумме с периодической вдоль трубы частью давления составляет $p$.

Поставленная задача, определяемая уравнениями \eqref{eq0}, \eqref{bc0}, \eqref{NSeq_periodic}, \eqref{bc1}, имеет единственное решение для каждого начального поля скорости $\v_0(\x)$, удовлетворяющего условию несжимаемости  \eqref{eq0}. Известно аналитическое решение задачи, называемое течением Пуазейля, соответствующее ламинарному режиму течения. Оно существует при всех значениях параметров и задается формулой
\begin{equation}
v_x = U (1 - (r / R)^2), 
v_r = v_\theta = 0. 
\end{equation}
Здесь величина $U = - D R^2\mu^{-1}/4 $ определяется по градиенту давления $D$ и соответствует максимальной скорости в ламинарном течении, которую оно достигает на оси трубы. Несложно показать, что средняя скорость течения $U_q$, вычисленная как отношение расхода к площади сечения трубы, в том случае равна $U/2$. 


В работе все вычисления выполняются в безразмерных переменных. В качестве масштабов выступают радиус трубы $R$, максимальная скорость в ламинарном течении $U$ и плотность жидкости $\rho$. Переход к безразмерным единицам измерения, обозначенным штрихами, выполняется по формулам
\begin{equation} \label{dim_less_eqs}
R x' = x,  R r' = r, RU^{-1} t' = t, U\v' = \v , \rho U^2 p' = p, \rho U^2 R^{-1} D' = D, R L'_x = L_x.
\end{equation}

Постановка задачи в безразмерных единицах измерения принимает вид (штрихи опущены)
\begin{equation}\label{NSeq_Re}
\pd{\v}{t} = - (\v, \nabla) \v - \i D - \nabla p + \frac{1}{\Re} \nabla^2 \v,
\end{equation}
\begin{equation}\label{eq0_Re}
\nabla \cdot \v = 0,
\end{equation}
\begin{equation}\label{bc0_Re}
\v = 0 \text{ при } r = 1,
\end{equation}
\begin{equation}\label{eq1_Re}
[\v, p](x,r,\theta,t) = [\v, p](x+L_x,r,\theta,t)
\end{equation}
Здесь $\Re = \rho R U / \mu$ --- число Рейнольдса, один из двух параметров системы. Вторым параметром является $L_x$. Величина $D$ подбирается из условия постоянства расхода так, что $U_q = 1/2$ в безразмерных единицах. В случае ламинарного течения она выражается в явном виде: $D = - 4 \Re^{-1}$. Ламинарное течение в безразмерном виде задается более простым выражением 
\begin{equation}
v_x = 1 - r^2, v_r = v_\theta = 0.
\end{equation}

Многие авторы вводят число Рейнольдса иначе, через расходную скорость $U_q$ и диаметр трубы $D$. Несложно показать, что значение числа Рейнольдса, введенного таким образом, совпадает со значением числа Рейнольдса, введенного выше. Однако, например, единица измерения времени $DU_q^{-1}$ в этом случае увеличивается в 4 раза.

В процессе решения задачи возникает необходимость перехода в подвижную систему координат. Выполняя переход, удобно сохранить в качестве тела, относительно которого определяются скорости, стенки трубы. В этом случае граничные условия и значения скорости не зависят от системы отсчета, но в уравнении движения \eqref{NSeq_Re} возникает новое слагаемое
\begin{equation}\label{NSeq_cf}
\pd{\v}{t} = c_f \pd{\v}{x} - (\v, \nabla) \v - \i D - \nabla p + \frac{1}{\Re} \nabla^2 \v. 
\end{equation}
Возникшее слагаемое отвечает за перенос решения вдоль трубы со скоростью $-c_f$, где $c_f = c_f(t)$ --- скорость перемещения системы координат, может быть функцией времени. Уравнение неразрывности \eqref{eq0_Re} при переходе, выполненном таким образом, не меняется. 


Поставленная задача сформулирована традиционным образом для прямого расчета турбулентных потоков в прямых каналах. В такой постановке удается воспроизводить характеристики течения, устанавливающегося на большом удалении от входа в трубу, решая уравнения лишь в ограниченной расчетной области. Условие периодичности освобождает от необходимости устанавливать условия на входе и выходе из трубы. В тоже время, увеличивая длину периода $L_x$, можно минимизировать влияние этого условия на поток. 

 
\section{Численный метод}

Численное решение задачи проводится методом [14]. Он сочетает консервативную конечно-разностную схему аппроксимации уравнений по пространственным координатам и полунеявный метод Рунге--Кутты 3-его порядка [15] интегрирования по времени.


Численный метод формулируется относительно уравнения \eqref{NSeq_Re}, преобразованного к виду, называемому иногда формой Громеки-Лемба
\begin{equation}\label{NSeq_om}
\pd{\v}{t} = \v \times \om - \i D - \nabla P + \frac{1}{\Re} \rot \om
\end{equation}
Здесь $\om = \rot \v$ --- вектор завихренности, посчитанный по полю скорости $\v$, $P = p + \v^2/2$ --- полное кинематическое давление. В такой форме проще учесть граничные условия и кривизну системы координат. 



Конечно-разностная аппроксимация проводится на так называемых смещенной или разнесенной сетке, когда разные характеристики движения определяются в различных точках пространства. 

Приводимые ниже результаты расчетов в диапазоне $1670\leqslant Re\leqslant 2800$ получены при $L_x=200$ с пространственным разрешением $2048 \times 64 \times 128$ в продольном, радиальном и угловом направлении соответственно. Расчеты на более грубой сетке $1024\times32\times64$ во всех рассмотренных случаях дают результаты совпадающие качественно и близкие количественно.


\section{Методические расчеты}

Стартуя с начальных данных в виде некоторого трехмерного возмущения течения Пуазейля, уравнения Навье--Стокса интегрируются до выхода решения на тот или иной режим. Установление решения, отвечающего турбулентному течению происходит в том случае, когда амплитуда начального возмущения достаточно велика, в противном случае возмущения затухают со временем, и решение в конечном итоге возвращается к ламинарному течению Пуазейля. Турбулентный режим за пределами диапазона переходных чисел Рейнольдса $Re\geqslant3000$ имеет вид статистически стационарного процесса и не зависит от конкретного вида начальных условий, при которых он был получен. Течение при этом однородно в продольном направлении, его статистические характеристики согласуются с имеющимися экспериментальными данными. При $\Re\leqslant2600$ в распределении скорости вдоль трубы появляется неоднородность, которая при $\Re\lesssim2200$ приобретает форму двигающейся вдоль трубы цепочки из нескольких пространственно-локализованных структур, разделенных участками ламинарного течения. Конкретное число получающихся в решении турбулентных структур зависит от начальных условий. Кроме того, как было отмечено выше, это число может меняться в процессе эволюции в результате исчезновения или деления отдельных структур. Получаемые в расчетах пространственно-локализованные турбулентные структуры хорошо согласуются с наблюдаемыми в экспериментах турбулентными порывами, что позволяет нам пользоваться этим их наименованием. Отметим, что турбулентные порывы формируются и на некотором отрезке времени существуют в расчетах и при $Re<2000$, вплоть до $Re=1670$. Однако, в этом случае не только число порывов в пределах расчетной области, но и время их существования является случайной величиной и зависит от конкретных начальных условий. Представленная на фиг.~1 визуализация рассчитанных течений в диапазоне $1680\leqslant Re\leqslant2800$ демонстрирует эволюцию локализованных структур при изменении числа Рейнольдса.




