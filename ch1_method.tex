
\chapter{Метод решения}

\section{Введение}

\section{Математическая постановка задачи}

Рассматривается течение вязкой несжимаемой жидкости в прямой трубе круглого сечения. Движение вызывается внешним градиентом давления, направленным вдоль трубы. Формулировка задачи традиционна для прямого расчета турбулентных течений в прямых каналах. Течение описывается уравнениями Навье--Стокса для несжимаемой жидкости. На твердой стенке трубы ставятся условия прилипания, а в направлении движения --- условия периодичности. Величина среднего градиента давления, вызывающего движение, определяется из условия постоянства расхода жидкости. Длина расчетной области (период) $L_x$ выбирается достаточно большой для возможности адекватного воспроизведения протяженных локализованных турбулентных структур. Численное решение задачи проводится методом [14]. Он сочетает консервативную конечно-разностную схему аппроксимации уравнений по пространственным координатам и полунеявный метод Рунге--Кутты 3-его порядка [15] интегрирования по времени. В дальнейшем все величины представляются в безразмерном виде. В качестве масштабов длины и скорости принимаются радиус трубы и максимальная скорость течения Пуазейля (удвоенная средняя скорость течения). 


\section{Численный метод}

Приводимые ниже результаты расчетов в диапазоне $1670\leqslant Re\leqslant 2800$ получены при $L_x=200$ с пространственным разрешением $2048\times64\times128$ в продольном, радиальном и угловом направлении соответственно. Расчеты на более грубой сетке $1024\times32\times64$ во всех рассмотренных случаях дают результаты совпадающие качественно и близкие количественно.


\section{Методические расчеты}

Стартуя с начальных данных в виде некоторого трехмерного возмущения течения Пуазейля, уравнения Навье--Стокса интегрируются до выхода решения на тот или иной режим. Установление решения, отвечающего турбулентному течению происходит в том случае, когда амплитуда начального возмущения достаточно велика, в противном случае возмущения затухают со временем, и решение в конечном итоге возвращается к ламинарному течению Пуазейля. Турбулентный режим за пределами диапазона переходных чисел Рейнольдса $Re\geqslant3000$ имеет вид статистически стационарного процесса и не зависит от конкретного вида начальных условий, при которых он был получен. Течение при этом однородно в продольном направлении, его статистические характеристики согласуются с имеющимися экспериментальными данными. При $\Re\leqslant2600$ в распределении скорости вдоль трубы появляется неоднородность, которая при $\Re\lesssim2200$ приобретает форму двигающейся вдоль трубы цепочки из нескольких пространственно-локализованных структур, разделенных участками ламинарного течения. Конкретное число получающихся в решении турбулентных структур зависит от начальных условий. Кроме того, как было отмечено выше, это число может меняться в процессе эволюции в результате исчезновения или деления отдельных структур. Получаемые в расчетах пространственно-локализованные турбулентные структуры хорошо согласуются с наблюдаемыми в экспериментах турбулентными порывами, что позволяет нам пользоваться этим их наименованием. Отметим, что турбулентные порывы формируются и на некотором отрезке времени существуют в расчетах и при $Re<2000$, вплоть до $Re=1670$. Однако, в этом случае не только число порывов в пределах расчетной области, но и время их существования является случайной величиной и зависит от конкретных начальных условий. Представленная на фиг.~1 визуализация рассчитанных течений в диапазоне $1680\leqslant Re\leqslant2800$ демонстрирует эволюцию локализованных структур при изменении числа Рейнольдса.




