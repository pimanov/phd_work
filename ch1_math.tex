
\chapter{Постановка задачи} 

Рассматривается течение вязкой несжимаемой жидкости в прямой трубе круглого сечения. Движение вызывается внешним градиентом давления, направленным вдоль трубы. Формулировка задачи традиционна для прямого расчета турбулентных течений в прямых каналах. Течение описывается уравнениями Навье--Стокса для несжимаемой жидкости. На твердой стенке трубы ставятся условия прилипания, а в направлении движения --- условия периодичности. Величина среднего градиента давления, вызывающего движение, определяется из условия постоянства расхода жидкости. Длина расчетной области (период) $L_x$ выбирается достаточно большой для возможности адекватного воспроизведения протяженных локализованных турбулентных структур. Численное решение задачи проводится методом [14]. Он сочетает консервативную конечно-разностную схему аппроксимации уравнений по пространственным координатам и полунеявный метод Рунге--Кутты 3-его порядка [15] интегрирования по времени. В дальнейшем все величины представляются в безразмерном виде. В качестве масштабов длины и скорости принимаются радиус трубы и максимальная скорость течения Пуазейля (удвоенная средняя скорость течения). 


