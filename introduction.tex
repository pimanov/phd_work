\chapter{Введение}








Возможным объяснением может быть то, что скорость жидкости падает после того, как в ней возникает турбулентноть, число Рейнольдса падает, и она не распространяется. 

\chapter{Введение}


Цикл самоподдержания пристенной трубулнетности связывают с когерентными стурктурами в буерном слое, наблдаемыми как в экспериментах \cite{kline1967structure}, так и в численных расчетах. Они представляют собой полосы повыешнной и пониженной скорости, вытянутые около стенки вдоль поткоа. Счиатется что они возникают за счет конвекции в нормальной к основному потоку плосоксти, там где медленная жидкость перемещаюется ближе к стенке трубы. 


1995 год: Пристанные структуры самовоспроизводятся. Одни умирают, порождая новые. Основные структуры около стенки - это полосы повышенной и пониженной скорости и продольные вихри. Расстояние между ними -- Kline1967, SmithMetzler1983, Kim1987). После многочисленных исследований кинематика когерентных структур хорошо описана (e.g. Kline et al. 1967; Robinson 1991). но механизмы скрыты, и характеристика не предсказать. 

Direct examination of the flow dynamics in a fully turbulent flow is complicated by the random distribution of the coherent structures in space and time, by the fact that no two realizations of a structure are identical, and by the presence of additional structures (‘debris’) which may not be necessary components of the regeneration process.

Цель ученых упростить задачу так, чтобы остались только существенные её особенности. 

JiminezMoin (1991) --- Численное исследование течения в канале 2000 < Re < 5000. --- и турбулентные законы воспроизвели, и видели полосы, вихри. 
SendstadMoin (1992) --- Тоже считали пристенные структуры. 
Aubry et al. (1988) --- подход динамических систем для исследования продольных вихрей. Как и турбулентность, вихри демонстрируют перемежаемость. Только наблюдение, никаких объяснений. 

Jang, Benney, Gran (1986) --- первая попытка объяснить происхождение пристенных полос. ‘direct resonance’ theory

Waleffe, Kim, Hamilton (1993) --- эта теория не работает. Более того, не один из исследованных ими линейных механизмов не может объяснить расстояние между полосами. 

Butler, Farrell (1993) --- оптимальные возмущения дают полосы. Есть противоречия с другими авторами. 

Jimknez, Moin (1991) --- Турбулентность пропадает, если ширина расчетной области меньше 100 юнитов --- ширины между полосами. Полосы --- важная деталь турбулентности. Waleffe et al. (1993) --- Re лучше вводить через трансверсальную ширину канала; тогда 100 -- критическое число. Это верно для плоского канала, трубы, и еще каких-то штук. 

Jimknez (1994) --- модель течения, воспроизводящая ЦСПТ. Sreenivasan (1988) -- другая, на основе идеи Coles (1978) --- неустойчивость Гетлера (центрифугальная). (Craik, Leibovich 1976) --- похожая неустойчивость. Langmuir circulations --- разработка этой неустойчивости применительно к продольным вихрям, возникающем в ветре на поверхности океана! Craik (1982)

В этой работе течение Куэтта (как у Ньютона). Конец введения. 

Формирование полос за счет продольных вихрей --- Bakewell \& Lumley 1967; Blackwelder \& Eckelmann 1979; Landahl 1980. 

Streak breakdown --- Kim, Kline \& Reynolds 1971; Swearingen \& Blackwelder 1987

Jimhez \& Moin (1991) --- Механизм генерации продольных вихрей: поперечные вихри поворачиваются так, что приобретают нормальную компоненту, затем нормальные вихри поворачивают поперечные так, что они приобретают продольную компоненту. Sendstad \&  Moin (1992) --- причина тоже наклон вихрей. В этом исследовании не так. 

Sendstad \& Moin (1992) --- выделили слагаемое $ - (\partial w / \partial x) (\partial u / \partial y)$, ответственное за генерацию продольных вихрей. Потом листы завихренности сворачиваются в сингулярные вихри. В этой статье говорят, что это слагаемое не работает, так как не дает стационарного вклада. Гетлера неустойчивость тоже не работает. 

Продольные вихри производит нелинейное взаимодействие альфа мод. 
Есть картинка цикла самоподдержания. Выводы --- это цикл. Разнесен во времени. Как-будто нашли нужно слагаемое, но не поняли этого. 

2003 Кавахара \cite{kawahara2003linear}: 

(Kline et al. 1967) --- полосы описал. (Stretch 1990; Jiménez \& Moin 1991; Jeong et al. 1997) --- вихри возникают по бокам от полосы замедления в шахмотном порядке. (Jiménez 1994; Hamilton, Kim \& Waleffe 1995; Waleffe 1997; Kawahara \& Kida 2001) --- Цикл самоподдержания. Panton (1997) --- обзор таких работ --- нет в сети. (Kline et al. 1967, Bakewell \& Lumley 1967) --- полосы производят продольные вихри. (Orlandi \& Jiménez 1994) --- полосы сохдают существенную часть трения на стенке. 

Механизмы генерации продольных вихрей: Jiménez \& Moin (1991) --- они возникают из-за растяжения вихрей(каких?). (Sreenivasan 1988) --- это эффект внешнего потока. (Brooke \& Hanratty 1993) --- эффект прилипания на стенке. Jiménez \& Pinelli (1999) --- полосы необходимы для их возникновнеия. 

Hamilton et al. (1995) --- нашел, что полоса линейно неустойчива (синусоидально), нелинейное трехмерное взаимодействие пульсаций ведет к вихрям. (Waleffe 1997) --- синусоидальная неустойчивость ведет к вихрям?. Вихри нужны только для предотвращения затухания плос за счет вязкости (Itano \& Toh 2001). 

Waleffe (1995, 1997) and Waleffe \& Kim (1997) --- неустойчивость полос из-за точкиперегиба между полосами повышенной и пониженной скорости. Swearingen \& Blackwelder (1987) -- перложен такой тип неустойчивости. Hall \& Horseman (1991) --- другая неустойчивость, между полосой замедления и стенкой, наверное. Важны обе. Обзор здесь Reddy et al. (1998). (Jiménez 1994) --- возникающие вихри нормальны к стенке, но они изгибаются и растягиваются так, что производят продольные. (wake-like mechanism)

Schoppa \& Hussain (1997, 1998) --- продольные вихри возникают еще на стадии линейного роста пульсаций на полосе. Неустойчивость похожа на неустойчивость плоского слоя смешения. Какие-то вихри сприваются. (Schoppa, Hussain \& Metcalfe 1995) --- предположение, что механизм как в слое смешения. (Schoppa 2000; Schoppa \& Hussain 2002) --- неустойчивость полос может быть приближена взволнованным в трансверсальном направлении слоем смешения. Приведены сравнения собственных функций в этом случае с наблюдаемыми. 

Kawahara et al. (1998) --- какая-то модель гладкая течеяни, которую он иследует. Schoppa \& Hussain (1997) --- похожая. Нашли три механизма неустойчивости. 

Schoppa \& Hussain (2002) --- исследовали зависимость скорости роста возмущений от длины волны и амплитуды полос. Затем изсследовали влияние оптимальной волны на полосы. 

(see Schoppa \& Hussain 1997) --- механимы связаны с точкой перегиба а значет не связаны с вязкостью. 

(Moore 1979; Cowley, Baker \& Tanveer 1999). --- скорость роста возмущений на слое смешения тем выше, чем короче их длина. За конечное время сингулярность. 

Ничего не понятно.. 

Schoppa 2002: 

Картинка полос. 

(Townsend 1956; Kline et al. 1967; Kovasz-nay, Kibens \& Blackwelder 1970; Cantwell 1981; Panton 1997) --- пристенные стурктуры. (e.g. Kim, Moin \& Moser 1987). --- доминирующая роль полос в пристенном трении. (Adrian, Meinhart \& Tomkins 1999) --- во внешнем потоке тоже есть структуры. (Jimenez \& Pinelli 1999) --- внешний слой сам по себе, (e.g. Kline et al. 1967) согласен. (e.g. Rao, Narasimha \& Narayanan 1971) --- противоположное мнение, что они связаны. (Jeong \& Hussain 1992; Jeong et al. 1997 (ited by 563)) --- подель цикла самоподержания воспроихводит особенности пристенного течения. 

kawahara2001periodic --- механизм самоподдержания, не только бегущие волны, но и периодические решения воспроихводят цикл. Продолжение темы статьи 1995. 

\section{Обзоры}

\cite{kerswell2005recent} --- обзор Керсвела 2005 год. 

\section{Простые решения}

Кавахара \cite{kawahara2012significance} --- обзор найденных к 2012 году решений, и пояснение смысла этой работы. 

\cite{altmeyer2015role} --- добыли очень устойчивую орбиту из турбулентного течения в трубе. 

\cite{budanur2015periodic} --- 

\section{Турбулентность}


\section{Самоподдержание турбулентного порыва}
\cite{duguet2010slug} --- Путь от решения на сепаратрисе к Слагу. А Слаг, как мы теперь знаем, также структура, что и порыв. 


\section{Самоподдержание пристенной турбулентности}
\cite{waleffe1997self, hamilton1995regeneration, waleffe1995hydrodynamic, waleffe1997self, schoppa2002coherent}

\cite{}


\section{Структуры в турбулентном течении}

Наблюдение полос: \cite{smith1983characteristics, schoppa2002coherent, kline1967structure}
\cite{schoppa2002coherent} -- возможно больше, чем просто наблюдение. 
\cite{klebanoff1962three}

\cite{jeong1997coherent, schoppa2002coherent} -- когерентные структуры. 



