
\chapter{Обзор литературы}

	
\section{Роль численного моделирования при изучении турбулнетных течений}

Бурное развитие вычислительной техники, наблюдаемое в последние десятилетия, открыло новые возможности исследования турбулентных течений. Современные методы обработки экспериментальных данных, такие как оптические методы определения скорости перемещения частиц ("Particle image velocimetry"), позволяют восстанавливать поле скорости в отдельном сечении потока или даже в некотором его объеме \cite{PIVbook}, что дает возможность выделять и объяснять закономерности движения жидкости в турбулентных потоках. 

Также появилась возможность воспроизводить движение жидкости при умеренных параметрах потока численно. Возможность адекватного воспроизведения характеристик турбулентного потока при решении уравнений Навье-Стокса для несжимаемой жидкости была продемонстрирована в \cite{Kim1987}. Сравнение результатов расчета с экспериментальными данными о развитом турбулентном течении в трубах можно найти в \cite{Priymak1998, Nikitin2006}. Численное моделирование дает исчерпывающею информацию о потоке. Кроме того, численно могут быть поставлены и решены задачи, недоступные в эксперименте. Например, то или иное течение может быть исследовано на устойчивость и определены все его неустойчивые моды; возникающие в потоке особенности могут быть выделены и изучены независимо друг от друга. Современные представления о турбулентных течениях существенным образом основаны на результатах численного моделирования, подкрепленных экспериментальными данными \cite{Manneville2015, Manneville2016}.

Вычисления могут быть выполнены лишь в ограниченной расчетной области. Постановка задачи в трубе, как правило, включает периодические граничные условия в продольном направлении, при этом жидкость приводится в движение внешним перепадом давления. В плоских каналах и других неограниченных в трансверсальном направлении течениях условие периодичности также накладывается и в этом направлении. На твердых стенках традиционно ставятся условия прилипания. На границе расчетной области, противоположной твердой стенке, может быть поставлено условие проскальзывания. Такого рода постановка позволяет воспроизводить характеристики установившихся течений в расчетной области небольшого размера \cite{Kim1987, Priymak1998, Nikitin2006}. 


\section{Ламинарно-турбулентный переход в круглых трубах}

В качестве наиболее простого и в тоже время содержательного с практической точки зрения случая, в котором наблюдается турбулентность, можно выделить движение жидкости в прямых трубах круглого сечения (жидкость заполняет все пространство внутри трубы). Известно, что турбулентный режим течения в трубах устанавливается только если скорость потока достаточно велика, при малых скоростях жидкость движется ламинарным образом. Классическим считается результат Осборна Рейнольдса, опубликованный в 1883 году \cite{Reynolds1883}, согласно которому характер течения жидкости определяется безразмерной комбинацией параметров, называемой числом Рейнольдса. Если число Рейнольдса $\Re = RU/\nu$, вычисленное по максимальной скорости $U$, радиусу трубы $R$, и кинематической вязкости $\nu$, ниже критического значения, близкого к $2000$, то реализуется ламинарный режим течения. При больших $\Re$, как правило, течение оказывается турбулентным. 

Стоит отметить, что в лабораторных условиях, снижая уровень возмущений в потоке и организуя плавный вход жидкости в трубу, можно сохранить течения ламинарным при числах Рейнольдса, значительно превышающих критическое, $\Re \sim 10^4$ и больших \cite{Wygnanski1973, Darbyshire1995, vanDoorne2009}. Это связано с тем, что турбулентность в трубах возникает жестким образом, без потери ламинарным течением устойчивости к малым возмущениям. Переход к турбулентности вызывают возмущения некоторой достаточно большой амплитуды \cite{Grossmann2000}, присутствующие в потоке. Следуя \cite{Darbyshire1995, Hof2003, Peixinho2007, Mellibovsky2009critical}, пороговое значение амплитуды возмущения, способного вызвать переход к турбулентности, асимптотически уменьшается по мере увеличения числа Рейнольдса по закону $\Re^{-\alpha}$, где значение $\alpha$ меньше двух и зависит от формы возмущения. Таким образом, с увеличением $\Re$ сохранить течение ламинарным становится сложнее. 

На удалении от входа в трубу ламинарное течение устанавливается, формируя так называемое течение Пуазелйя. Сегодня принято считать, что течение Пуазейля линейно устойчиво при всех числах Рейнольдса \cite{Kerswell2005}. В работе \cite{Salwen1980} показано, что течение Пуазейля устойчиво к осесимметричным возмущениям при всех $\Re$. Устойчивость течения Пуазелйя к возмущениям произвольной формы была продемонстрирована численно в \cite{Meseguer2003} до $\Re = 10^7$. 

То обстоятельство, что турбулентность в трубах может существовать несмотря на линейную устойчивость ламинарной формы течения, позволяет говорить о механизме её самоподдержания, ответственном за сохранение турбулентных пульсаций в потоке. В случае отсутствия такого механизма амплитуда пульсаций будет падать за счет действия вязкости, и со временем установится ламинарный режим течение в силу его линейной устойчивости. На практике наблюдается обратная ситуация --- при достаточно больших значениях $\Re$ турбулентность, однажды возникнув, продолжает существовать, и вернуть поток в ламинарное состояние не представляется возможным. 

Обзоры, посвященные ламинарно-турбулентному переходу в трубах, могут быть найдены в работах \cite{Kerswell2005, Eckhardt2007, Manneville2015, Manneville2016, Kreilos2014, Barkley2016}.



\section{Локализованные турбулентные структуры в трубах} \label{local_structures}

Как отмечал еще Осборн Рейнольдс \cite{Reynolds1883}, турбулентность в трубах первоначально проявляется перемежающимся образом --- участки возмущенного и спокойного движения следуют вдоль трубы друг за другом, практически не меняя своей протяженности. На тот момент причина пространственной локализации турбулентности установлена не была. Сегодня известно, что в разных условиях могут возникать структуры заметно разных типов. 

Организуя плавный вход жидкости в трубу и поддерживая низкий уровень возмущений в потоке течение можно сохранить ламинарным при числах Рейнольдса, значительно превышающих критическое значение. В этом случае ограниченные по времени и в пространстве возмущения могут приводить к образованию локализованных турбулентных структур, называемых турбулентными пробками ("turbulent slugs"). Согласно \cite{Wygnanski1973}, турбулентные пробки возникают при $\Re > 3200$. Они, двигаясь вниз по трубе, увеличивают свою протяженность, вовлекая в турбулентное движение окружающую жидкость на переднем и заднем фронтах. По мере того, как соседние локализованные структуры нагоняют друг друга, сливаясь вместе, происходит переход к сплошной турбулентности. Серия работ посвящена изучению скорости распространения турбулентности вверх и вниз по потоку и структуре фронтов \cite{Lindgren1969, Wygnanski1973, Nishi2008, Duguet2010, Barkley2015}. С увеличением числа Рейнольдса скорость заднего фронта падает, в то время, как скорость переднего возрастает, таким образом, скорость распространения турбулентности растет. Однако стоит отметить, что скорость заднего фронта всегда остается положительной, то есть турбулентность не распространяется вверх по потоку. Это верно по крайней мере до $\Re = 10^5$ \cite{Wygnanski1973}. 

При переходных значения числа Рейнольдса в потоке формируются локализованные структуры принципиально другого типа, называемые {\it турбулентными порывами} ("turbulent puffs"). Следуя \cite{Wygnanski1973}, турбулентные порывы возникают при сильной возмущенности потока на входе в трубу при $2000<\Re<2700$. Порывы сносятся вниз по потоку со скоростью, близкой к средней скорости течения, сохраняя свою форму и пространственную протяженность практически неизменными. В длину порыв имеет несколько десятков диаметров трубы. На заднем фронте порыва скорости жидкости вблизи оси трубы падает скачком на $30-40\%$, затем плавно восстанавливается на переднем его фронте. В указанном диапазоне чисел Рейнольдса турбулентность может существовать только в форме порывов \cite{vanDoorne2009, Moxey2010, Samanta2011}. Даже сплошная турбулентность, полученная при больших значения $\Re$, при снижении $\Re$ до переходных значений распадается на отдельные участки, разделенные ламинарным потоком, приобретающие форму турбулентных порывов. Форма порыва не зависит от начального возмущения, но оно может влиять на их количество и расположение вдоль трубы. Если порывов несколько, они могут быть расположены в трубе нерегулярно, однако расстояние между соседними порывами всегда больше некоторой величины \cite{Samanta2011}. В работе \cite{Wygnanski1975} установлено, что при $\Re<2100$ порывы подвержены спонтанному исчезновению, а при $\Re>2300$ возможно деление порыва на два следующих друг за другом. Введено понятие {\it равновесного порыва}, характеристики которого не меняются по мере его продвижения вдоль трубы. Согласно \cite{Wygnanski1975} это наблюдается при $2100\leqslant \Re \leqslant 2300$. 

В \cite{Moxey2010, Barkley2015, Song2017} было выполнено подробное исследование динамики фронтов, возникающих на границе ламинарного и турбулентного режимов течения. Было показано, что задний фронт турбулентной пробки качественно не отличим от заднего фронта турбулентного порыва. Он имеет ярко выраженную форму --- скорость жидкости вблизи оси трубы падает на нем скачком. По мере увеличения $\Re$, при переходе от динамики турбулентного порыва к турбулентной пробке, скорость заднего фронта плавно снижается. В тоже время, передний фронт претерпевает ряд качественных изменений. Согласно \cite{Moxey2010}, при $\Re < 2250$, когда турбулентность представлена турбулентным порывами, скорость переднего фронта совпадает со скорость заднего. При больших $\Re$ передний фронт приобретает собственную скорость, которая возрастает по мере увеличения $\Re$. Таким образом, область, занятая турбулентностью, начинает увеличиваться, однако формирования протяженных турбулентных структур не происходит. Увеличение длины порыва приводит к его делению на два, следующих друг за другом \cite{Moxey2010}. Передний фронт претерпевает еще одно качественное изменение в диапазоне чисел Рейнольдса $2600 < \Re < 3200$ \cite{Barkley2015}. При меньших значениях $\Re$ в сравнении с задним фронтом передний размыт --- скорость жидкости на оси трубы меняется плавно. При б\'{о}льших $\Re$ передний фронт приобретает ярко выраженную форму, повторяющую формой заднего фронта. На размытом переднем фронте турбулентность постепенно затухает. Напротив, на ярко выраженных фронтах наблюдается повышенный уровень производства энергии турбулентных пульсаций, которые затем попадают внутрь локализованных структур; жидкость, двигающаяся ламинарным образом, активно вовлекается в турбулентное движения \cite{Song2017}. 

Как уже было отмечено, турбулентный порыв способен к спонтанному затуханию, ведущему к ламинаризации потока. Ряд подробных экспериментов \cite{Hof2006finite, Willis2007, Peixinho2007} позволил установить, что с порывом может быть связано характерное время жизни $\tau$. Вероятность найти выделенный порыв через время $t$ падает с ростом $t$ по закону:
$$P_{turb} \approx \exp(-t/\tau).$$ 
C увеличением числа Рейнольдса характерное время жизни порыва растет по суперэкспоненциальному закону \cite{Hof2008, Kuik2010}:
$$\tau \sim \exp(\exp(c_1 \Re + c_2)).$$ 
Конкурирующей с тенденцией к затуханию является тенденция к делению порыва на два, следующих друг за другом. Как было показано в \cite{Avila2011}, с этой тенденцией также может быть связано характерное время --- характерное время до первого деления. С ростом $\Re$ оно падает также по суперэкспоненциальному закону. Согласно точке зрения \cite{Avila2011}, значение $\Re=\Re^*=2040$, при котором происходит смена доминирующей тенденции, является точкой статистического фазового перехода и может быть принята в качестве критического числа Рейнольдса в круглой трубе. При $\Re<\Re^*$ турбулентный порыв скорее погибнет, чем успеет разделиться, так что возникновение развитого турбулентного течения невозможно. Наоборот, при $\Re>\Re^*$ порыв скорее успеет произвести потомство прежде, чем погибнет, что приводит к развитию незатухающего турбулентного движения. 

Стоит отметить, что при $\Re = \Re^*$ характерное время жизни порыва, совпадающее с временем до первого деления, имеет порядок $10^8 R/U$. Описанные результаты согласуются с представлениями о равновесном порыве, наблюдаемом при близких значениях $\Re$, так как вероятность какой-либо из рассматриваемых событий в типовом эксперименте крайне мала. При $\Re > 2250$, когда, согласно \cite{Moxey2010}, порыв начинает активно делиться, характерное время до первого его деления падает на несколько порядков. 

Турбулентный порыв представляет собой интересный гидродинамический объект, который в некотором отношении можно рассматривать как структурную единицу турбулентности. Можно сформулировать ряд вопросов, касающихся его поведения. До конца не понятен механизм, обуславливающий пространственную локализацию и самоподдержание порыва, неясны причины, побуждающие его к делению или затуханию, неизвестны факторы, определяющие его протяженность и скорость перемещения вдоль трубы. Среди большого количества работ, посвященных изучению порыва, можно выделить несколько, в которых сделала попытка объяснить те или иные его особенности, выделить механизм его самоподдержания. 

В \cite{Barkley2015, Barkley2016} предложена феноменологическая модель, воспроизводящая ряд особенностей ламинарно-турбулентного перехода в трубах, таких как пространственная локализация турбулентности, скорость перемещения турбулентных структур вдоль трубы, их спонтанное затухание и деление; структура и скорость перемещения фронтов на границах ламинарной и турбулентной форм течения, переход к сплошной турбулентности. Модель оперирует двумя величинами --- наполненностью среднего профиля скорости и уровнем турбулентных пульсаций, как функциями продольной координаты и времени. Их взаимодействие определяется из физических соображений общего характера и описывается уравнениями в частных производных с небольшим числом параметров. От части, модель позволяет объяснить пространственную локализацию турбулентного порыва. Жидкость, двигающаяся ламинарным образом, активно вовлекается в турбулентное движение на заднем фронте порыва, что соответствует \cite{Song2017}, таким образом турбулентность распространяется верх по потоку. Турбулентные пульсации поддерживают свое существование за счет среднего течения, увеличивая наполненность среднего профиля скорости. Когда наполненность среднего течения превышает некоторое критическое значение, турбулентные пульсации пропадают. Таким образом, жидкость, попадающая внутрь турбулентного порыва на его заднем фронте, со временем приобретает наполненный профиль скорости, на котором течение становится ламинарным и формируется передний фронт порыва. До тех пор, пока профиль скорости не восстановится, возникновение нового порыва не возможно. При б\'{о}льших значениях параметра, играющего роль числа Рейнольдса, скорость восстановления среднего профиля скорости за счет вязких сил превышает скорость его деградации, вызванной турбулентными пульсациями, и ограничение на длину турбулентных структур снимается. Вопрос механизма, по средством которого турбулентные пульсации поддерживают свое существование, в работах \cite{Barkley2015, Barkley2016} не рассматривается.

Попытка объяснения механизма самоподдержания турбулентного порыва предпринята в \cite{Shimizu2009}. Турбулентные пульсации внутри порыва приводят к образованию неоднородности потока в угловом направлении. Вблизи стенки формируются вытянутые вдоль потока области, скорость жидкости внутри которых выше или ниже среднего значения, называемые пристенными полосами. Формирование пристенных полос является неотъемлемой частью всех сценариев самоподдержания пристенной турбулентности \cite{Hamilton1995, Waleffe1997, Schoppa2002}. Вместе с жидкостью, отстающей от порыва, пристенные полосы переносятся в его заднюю часть, где между полосой пониженной скорости и набегающим ламинарным течением формируется сдвиговый слой, подверженный неустойчивости типа Кельвина-Гельмгольца. Возникающие в результате неустойчивости пульсации переносятся в переднюю часть порыва вблизи оси трубы, поддерживая существование турбулентности внутри локализованной структуры. Так, согласно \cite{Shimizu2009}, выглядит цикл самопроизводства турбулентных пульсаций внутри порыва и цикл самоподдержания самой этой структуры. 

В работе \cite{Hof2010} механизм самоподдержания порыва также связывают со сдвиговым слоем в задней его части. В задней части порыва средний профиль скорости имеет точку перегиба в радиальном направлении, возникающую между ламинарным течением, проникающим внутрь порыва в центральной части трубы, и турбулентным течением, сохраняющимся вблизи стенки. В области небольшой протяженности, где сдвиговый слой наиболее ярко выражен, достигает наибольшего значения интенсивность турбулентных пульсаций и формируется существенная часть продольных вихрей, которые затем за счет конвекции переносятся вниз и вверх по потоку, поддерживая его неоднородность. В работе \cite{Hof2010} также было предложено управление, позволяющее ламинаризовать поток при небольших значениях числа Рейнольдса, когда турбулентность представлена порывами. Однако сделанные в работе выводы нельзя считать обоснованными в полной мере. 


\section{Пристенные турбулентные структуры} \label{structure_subsection}

Особенности турбулентного порыва, выделенные в \cite{Shimizu2009}, и связанный с ними механизм самоподдержания являются характерными для широкого класса пристенных турбулентных течений. Известно, что вблизи стенки в турбулентном течении существуют долгоживущие крупномасштабные структуры, способные к самоподдержанию. В первую очередь они представлены пристенными полосами ("streaks") --- вытянутыми вдоль потока областями, скорость жидкости внутри которых выше или ниже среднего значения на данном расстоянии от стенки \cite{Klebanoff1962, Kline1967}. Хотя полосы могу изгибаться и перемещаться вдоль стенки, пропадать и возникать вновь, они хорошо различимы на фоне беспорядочных турбулентных пульсаций. Геометрические характеристики полос в широком диапазоне параметров оказываются универсальными постоянными в пристенных единицах длины $l^+ = \nu / \sqrt{\tau_{w} / \rho}$, вычисленных по локальным характеристикам потока, а именно кинематической вязкости $\nu$, среднему трению на стенке $\tau_{w}$ и плотности жидкости $\rho$. Среднее расстояние между соседними полосами одного знака оценивают в $100 l^+$. Полосы достигают наибольшей интенсивности на расстоянии от $10 l^+$ до $20 l^+$ от стенки. Длину полосы можно оценить в $1000 l^+$. С пристенными полосами связывают возникновение турбулентных пульсаций, так как полосчатый профиль скорости может быть неустойчив. В отличии от ламинарного течения, полосчатый профиль скорости содержит точки перегиба, расположенные между соседними полосами замедления и ускорения или между полосой замедления и стенкой. 

Формирование пристенных полос связывают с вытянутыми вдоль потока вихрями \cite{Blackwelder1979, Jeong1997}, перемещающими жидкость в нормальной к основному потоку плоскости. Там, где вихри переносят медленную жидкость от стенки трубы в основной поток, формируются полосы пониженной скорости. Там, где жидкость перемещается ближе к стенке, возникают полосы повышенной скорости. Описанный механизм называют лифт-ап эффектом ("lift-up effect"). Считается, что продольные вихри возникают в результате нелинейного взаимодействия пульсаций, возникающих на полосчатом профиле скорости. Таким образом, с пристенными когерентными структурами может быть связан цикл самоподдержания \cite{Hamilton1995, Waleffe1997, Schoppa2002, Kawahara2003}. На настоящее время наиболее спорным остается вопрос о том, каким образом нелинейное взаимодействие пульсаций ведет к образованию продольных вихрей. 

В открытых течениях, где в нормальном к стенке направлении поток можно считать неограниченным, формируются вихри, имеющие форму подковы или шпильки для волос ("hairpin vortices") \cite{Head1981, Robinson1991, Adrian2000, Adrian2007}. Такую структуру образует пара продольных вихрей, имеющих противоположное направление вращения, расположенных по бокам от полосы замедления, смыкаются ниже по течению на некотором удалении от стенки. 

Характерной особенностью пристенных турбулентных течений является логарифмический профиль средней скорости (зависимость продольной скорости от расстояния до стенки) \cite{Kim1987}. В пристенном масштабе профиль скорости имеет универсальную форму. Возможно, формирование логарифмического профиля скорости можно связать с существованием когерентных структур, так как полосы находятся на том же расстоянии от стенки, на котором происходит переход от ламинарного подслоя к логарифмическому слою.


\section{Параллели с другими сдвиговыми течениями}

Особенности перехода к турбулентности, наблюдаемые в круглых трубах, являются характерными для ряда сдвиговых течений \cite{Manneville2015, Manneville2016}, таких как течения в плоском канале Пуазейля и Куэтта, или в трубах прямоугольного сечения. Отчасти, аналогичные свойства демонстрирует пограничный слой на плоской пластине. При небольших значениях числа Рейнольдса (при обтекании плоской пластины локального числа Рейнольдса, вычисленного по расстоянию до передней кромки) жидкость движется ламинарным образом. При достаточно больших $\Re$ устанавливается турбулентный режим течения. Во многих сдвиговых течениях переход к турбулентности происходит жестким образом без потери ламинарным течением линейной устойчивости. Кроме того, при переходных значениях $\Re$ в них проявляется ламинарно-турбулентная перемежаемость --- области, занятые ламинарным и турбулентным течением сменяют друг друга в продольном направлении.  

В плоском канале жидкость заключена между двумя параллельными плоскими стенками. В плоском канале Пуазейля жидкость приводится в движение внешним перепадом давления, направленным вдоль потока; в плоском канале Куэтта --- смещением одной из стенок относительно второй с постоянной скоростью так, что расстояние между ними остается постоянным. В обоих случаях ламинарный поток, устанавливающийся с течением времени, может быть представлен в аналитическом виде и исследован на линейную устойчивость, в частности, в рамках уравнений Навье-Стокса. В плоском канале Пуазейля установившееся ламинарное течение, называемое течением Пуазейля, теряет линейную устойчивость при $\Re = 5772$ \cite{Orszag1971}, однако турбулентность в потоке наблюдается уже при $\Re = 1000$ \cite{Orszag1980}. Число Рейнольдса определяется по половине ширины канала и максимальной скорости потока. В плоском канале Куэтта турбулентность возникает при числах Рейнольдса, близких к $300$ \cite{Bottin1998}, однако установившееся ламинарное течение в этом случае устойчиво при всех значениях $\Re$ \cite{Romanov1973}. Здесь число Рейнольдса определяется по половине ширины канала и половине разности скоростей стенок. 

Как и в круглой трубе, в плоском канале Куэтта \cite{Prigent2002, Barkley2005} и Пуазейля турбулентность при переходных значениях $\Re$ принимает форму локализованных вдоль потока структур. Плавно снижая число Рейнольдса от значения, при котором наблюдается сплошная турбулентность, в канале Куэтта можно наблюдать формирование пространственной неоднородности потока \cite{Duguet2010Couette}. При $340 < \Re < 415$ турбулентность формирует косые полосы, проходящие под некоторым углом к направлению течения; при $325 < \Re < 340$ сплошные полосы распадаются на отдельные фрагменты и турбулентные пятна; при меньших $\Re$ продолжительное время турбулентность существовать не может. В плоском канале Пуазейля аналогичный эксперимент дает качественно неотличимые результаты. В интервале чисел Рейнольдса $800 < \Re < 1000$ турбулентность также существует в форме косых полос; при числах Рейнольдса, близких к 800, косые полосы распадаются на отдельные участки \cite{Tuckerman2014, Lernoult2014, Sano2015}, при меньших $\Re$ турбулентность перестает быть устойчивой. 

Известно, что ламинарный пограничный слой на плоской пластине теряет линейную устойчивость к волнам Толмина-Шлихтинга при $\Re \sim 520$ \cite{Schlichting2004}. Число Рейнольдса в этом случае определено по расстоянию от передней кромки пластины и скорости набегающего потока на бесконечности. Однако при достаточно высоком уровне возмущений в набегающем потоке турбулентность может возникнуть при значениях $\Re$ ниже критического. В этом случае внесенные в поток возмущения приводят к образованию турбулентных порывов \cite{Katasonov2014} (не путать с турбулентными порывами в трубах). По мере перемещения вниз по потоку они увеличиваются в размере: их передний фронт перемещается со скоростью $0.9$ скорости набегающего потока в то время, как задний со скорость $0.5$. При этом их ширина и толщина практически не меняются. Порывы представлены модуляцией преимущественно продольной компоненты скорости в пристенном сдвиговом течении. В них могут быть выделены продольные полосы повышенной и пониженной скорости. В тот момент, когда сдвиговые слои, возникающие между полосами, теряют устойчивость, на месте порывов формируются турбулентные пятна. Такие же пятна возникают на месте вол Толмина-Шлихтинга в случае мягкого сценария перехода к турбулентности. Сливаясь вместе пятна дают начало сплошной турбулентности. 


\section{Инвариантные решения уравнений Навье-Стокса}



Численное моделирование позволяет не только воспроизводить характеристики турбулентных потоков, но и получать недостижимые в экспериментах режимы течения. Так, численно могут быть найдены решения уравнений Навье-Стокса, имеющие регулярное поведение в пространстве и во времени. Такие решения называют инвариантными решениями или точными когерентными структурами ("exact coherent structures"). Наиболее простым примером инвариантного решения является нелинейная бегущая волна --- периодическая вдоль потока, стационарная в некоторой подвижной системе отсчета. Более общим примером может быть решение, меняющееся во времени периодическим образом. В сдвиговых течениях при переходных значениях $\Re$ найдено некоторое количество инвариантных решений \cite{Kawahara2012}. Хотя все известные решения неустойчивы, размерность неустойчивого многообразия многих из них невелика. Такие решения могут играть существенную роль в организации турбулентного движения \cite{Chaosbook}. Ожидается, что можно выделить небольшое число инвариантных решений, образующих <<скелет>> турбулентного движения. В фазовом пространстве траектория, соответствующая турбулентному течению, значительную часть времени проводит вблизи таких решений, и покидая одно из них вдоль его неустойчивого направления, попадает в область притяжения другого решения, так как размерность его неустойчивого многообразия мала. Кроме того, все найденные инвариантные решения в некоторое степени воспроизводят особенности пристенной турбулентности и связанный с ними механизм самоподдержания, который, в силу простоты их поведения может быть полностью исследован. 

Как бегущие волны, так и периодические по времени решения удовлетворяют некоторому нелинейному уравнению, которое может быть решено методом Ньютона, обобщенным на многомерный случай \cite{Viswanath2007, Dijkstra2014}. Основной сложностью при нахождении инвариантных решений оказывается отыскание подходящего начального приближения, с которым метод Ньютона сойдется. Когда одно решение уже найдено, оно может быть использовано в качестве начального приближения для решения с близкими значениям параметров, что позволяет непрерывным образом переводить решение от одних значений параметров к другим. 

Впервые решение типа бегущей волны было найдено в \cite{Nagata1990} для плоского канала Куэтта. Для этого сперва была получена устойчивая бегущая волна, возникающая в результате потери линейной устойчивости ламинарным течением Куэтта-Тейлора, формирующегося между двумя соосными вращающимися цилиндрами. Затем, при уменьшении кривизны цилиндров до нуля, полученное решение непрерывным образом было переведено в плоский канал Куэтта. Эта же бегущая волна получена и исследована в \cite{Clever1992, Waleffe1998, Waleffe2003}. В \cite{Clever1992} введена разность температур между стенками плоского канала (расположенными горизонтально), что также позволяет получить устойчивую бегущую волну, которая, при уменьшении разности температур до нуля, приводится к решению \cite{Nagata1990}. Решение \cite{Waleffe1998, Waleffe2003} получено на основе представлений о механизме самоподдержания пристенной турбулентности. В этом случае устойчивую бегущую волну позволило получить введение массовой силы, создающей продольные вихри. В работах \cite{Gibson2008, Gibson2009, Itano2009, Nagata1997, Schmiegel1999} были найдены новые бегущие волны для плоского канала Куэтта, по-видимому, несвязанные с решением \cite{Nagata1990}. В плоском канале Пуазейля в отличии от канала Куэтта, ламинарное течение теряет устойчивость к малым возмущениям при конечном значении $\Re$, в результате чего течение принимает форму двумерной бегущей волны. В \cite{Ehrenstein1991} была найдена трехмерная бегущая волна, возникающая из двумерного течения. Найденные в \cite{Waleffe1998, Waleffe2001, Waleffe2003, Itano2001} бегущие волны в плоском канале по видимому с ламинарным течение не связаны. Бегущая волна \cite{Waleffe1998, Waleffe2001, Waleffe2003} получена непрерывным переносом решения \cite{Nagata1990} из канала Куэтта в плоский канал. В круглой трубе также было найдено некоторое количество решений, имеющих вид бегущих волн, \cite{Faisst2003, Wedin2004, Pringle2007, Pringle2009, Altmeyer2015}. В \cite{Hof2004} сообщается об экспериментальном наблюдении некоторых из бегущих волн \cite{Faisst2003, Wedin2004} при небольших $\Re$. В частности, при $\Re = 2000$ на переднем фронте турбулентного порыва формируется бегущая волна, обладающая $2\pi/3$-периодичностью в угловом направлении, имеющая ярко выраженные полосы замедления и ускорения. Однако позднее было установлено \cite{Kerswell2007}, что даже в расчетной области длиной $10$-$20R$ при переходных $\Re$ турбулентное течение проводит вблизи известных бегущих волн не более $10\%$ времени. Возможно, приблизить динамику турбулентного течения позволят более сложные решения, меняющиеся во времени периодическим образом.

Сегодня известно некоторое число периодических по времени решений. Исследуя бегущую волну \cite{Nagata1990} в работе \cite{Clever1997} была обнаружена бифуркация Андронова-Хопфа, в которой из нее рождается периодическое по времени решение. Также периодические по времени решения для канала Куэтта были найдены в работах \cite{Kawahara2001, Viswanath2007}. Периодическое по времени решение для плоского канала Пуазейля найдено в работе \cite{Toh2003}. В \cite{Duguet2008} найдено периодическое решение в круглой трубе, возникающее в результате бифуркации от одной из бегущих волн \cite{Pringle2007}. Также в круглой трубе непосредственно из турбулентного течения удалось выделить несколько периодических траекторий, имеющих значительно больший период по времени и сложную структуру \cite{Altmeyer2015}. Сообщается, что турбулентное течение, попадая в окрестность выделенной периодической траектории, проводит около нее значительное время. 

Все представленные выше инвариантные решения являются периодическими вдоль потока, причем величина периода небольшая, порядка десяти $H$, где $H$ --- радиус трубы или полуширина канала. Хотя такие решения воспроизводят многие особенности турбулентного движения, они не позволяют приблизить локализованные турбулентные структуры, возникающие при переходных значениях $\Re$. В канале Куэтта в \cite{Cherhabili1997, Ehrenstein2008, Schneider2010} были найдены бегущие волны, локализованные в направлении потока. В \cite{Gibson2014} были найдены инвариантные решения в канале Куэтта и Пуазейля, локализованные в поперечном направлении. Позднее, как в плоском канале \cite{Zammert2014}, так и в канале Куэтта \cite{Brand2014}, были найдены полностью локализованные в пространстве инвариантные решения. В круглой трубе локализованное в пространстве периодическое по времени решение было найдено в \cite{Avila2013}. Это решение принадлежит сепаратрисе, отделяющей области притяжения, соответствующие ламинарному и турбулентному режимам течения, и воспроизводит ряд особенностей турбулентного порыва. 


\section{Решение на сепаратрисе в круглой трубе}

В прямых трубах, как и других сдвиговых течениях, турбулентность возникает вопреки линейной устойчивости ламинарного течения. Переход от ламинарного режима течения к турбулентному вызывают возмущения достаточно большой амплитуды, внесенные в поток, в то время как малые возмущения затухают. В таких условиях существуют возмущения некоторой граничной амплитуды, инициирующие режим течения, балансирующий между ламинарным и турбулентным состояниями. В фазовом пространстве такое движение происходит на сепаратрисе, отделяющей область притяжения решений, соответствующих ламинарному и турбулентному режимам течения. Хотя решение на сепаратрисе неустойчиво, оно может быть найдено численно. Как правило, решение на сепаратрисе сохраняет характерные черты турбулентного течения, но имеет более простую форму и динамику.

Метод поиска решения на сепаратрисе был предложен в \cite{Skufca2006}. Позднее в расчетной области небольшой протяженности решение на сепаратрисе было найдено в круглой трубе \cite{Schneider2007}, в канале Куэтта \cite{Schneider2008} и других сдвиговых течениях \cite{Kreilos2013}. Хотя возникающие решения имеют более простую динамику, чем соответствующее турбулентное течение, они остаются хаотическими. В более поздних расчетах, проведенных в круглой трубе \cite{Mellibovsky2009transition} и в канале Куэтта \cite{Duguet2009}, было установлено, что в протяженной расчетной области решение на сепаратрисе оказывается локализованным в пространстве. Однако, если в трубе пространственная локализации турбулентности проявляется при $\Re < 3000$, решение на сепаратрисе оказывается локализованным по крайней мере до $\Re = 6000$ \cite{Duguet2010}, и с увеличение $\Re$ его амплитуда (отклонения от ламинарного течения) и пространственная протяженность падают. Как отмечают в \cite{Duguet2010, Avila2013}, локализованное решение на сепаратрисе воспроизводит ряд характерных особенностей турбулентного порыва, и приближается к нему при небольших значениях $\Re$. Структуры, напоминающие локализованное решение на сепаратрисе, наблюдались в экспериментах \cite{deLozar2012} и в численных расчетах \cite{Manneville2011} при затухании турбулентного течения. 

Работа \cite{Duguet2010} посвящена переходу от решения на сепаратрисе к турбулентному течению, что может быть полезно для понимания ламинарно-турбулентного перехода. При потере решением на сепаратрисе устойчивости и развитии турбулентного течения, сперва возрастает амплитуда локализованной структуры, затем тонкий сдвиговый слой на заднем её фронте теряет устойчивость, что ведет к значительному увеличению интенсивности пульсаций на нем и снижению скорости его перемещения вдоль трубы. Вероятно, в \cite{Shimizu2009, Hof2010} механизм самоподдержания порыва связывают именно с этим сдвиговым слоем (конец раздела \ref{local_structures}). Однако, уже в решении на сепаратрисе скрыт некоторый механизм самоподдержания, не связанный с выделенным сдвиговым слоем, и можно ожидать, что он остается существенным в турбулентном порыве. 

В работе \cite{Avila2013} было обнаружено, что при наложении дополнительных условий симметрии на течение в трубе, при переходных значениях $\Re$ решение на сепаратрисе выходит на условно периодический режим --- периодический в подходящей подвижной системе отсчета. Возникающее периодическое решение воспроизводит локализованную в пространстве структуру, воспроизводящую характерные особенности турбулентного порыва, которую мы будем называть {\it модельным порывом}. Простота поведения модельного порыва позволяет провести исчерпывающее исследование его свойств, которые, как мы полагаем, прояснят определенные детали поведения турбулентного порыва. 

Продлевая решение \cite{Avila2013} по параметру, можно получить новые решения, характеристики которых ближе к характеристикам турбулентного течения. Уже авторам \cite{Avila2013}, продлевая решение в сторону уменьшения $\Re$, удалось достичь точки бифуркации, в которой рождается две ветви решения, и перейти с нижней ветви на верхнюю. Решения на верхней ветви сохраняют пространственную локализацию и временное поведение, но их характеристики оказываются ближе к характеристикам турбулентного течения. Если решения с нижней ветви принадлежат границе области притяжения, соответствующей турбулентному режиму течения, решения с верхней ветви могут участвовать в организации турбулентного аттрактора. Можно ожидать, что результаты, полученные при исследовании верхней ветви, имеют большее отношение к турбулентному порву, чем результаты исследования исходного решения, принадлежащего сепаратрисе. 


