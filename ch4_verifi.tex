
\chapter{Верификация полученных результатов}


В предыдущих главах, исследуя модельный порыв, удалось получить некоторые представления о его структуре и механизме самоподдержания. В настоящей главе поднимается вопрос об общности полученных результатов. Наличие пристенных полос повышенной и пониженной скорости является характерной особенности многих инвариантных решений уравнений Навье-Стокса \cite{Kawahara2012} и пристенной турбулентности непосредственно \cite{Kline1967, Smith1983, Schoppa2002}. Это обстоятельство позволяет надеяться, что выделенные при исследовании модельного порыва особенности движения, могут быть в некоторой степени обобщены и на эти случаи. Выполнить строгое исследование турбулентного течения сегодня не представляется возможным, однако полученные результаты могут быть проверены на других инвариантны решения. Продлевая решение, соответствующее модельному порыву, по числу Рейнольдса \cite{Sanchez2004, Viswanath2007, Dijkstra2014}, удалось получить новое локализованное в пространстве условно периодическое по времени решение, характеристики которого оказываются ближе к характеристикам турбулентного течения. Также удалось получить несколько различных бегущих волн в круглой трубе и плоском канале Пуазейля. Результаты исследования найденных инвариантных решений представлены в настоящей главе. 


\section{Семейство условно периодических по времени решений поставленной задачи}

Решение поставленной задачи $\v_p(x,r,\theta,t)$ является периодическим по времени с периодом $T$, если оно удовлетворяет условию:
\begin{equation} \label{tper_eq}
\v_p(x,r,\theta,t) = \v_p(x - c_p T, r, \theta, t + T).
\end{equation}
Здесь $c_p$ --- скорость перемещения периодического решения вдоль трубы. В конечно-разностной постановке условие \eqref{tper_eq} может быть сформулировано иначе:
\begin{equation}\label{P_eq}
\phi(\v_p, T, c_p, \Re) - \v_p = 0.
\end{equation}
Здесь функция $\phi(\v, t, c, \Re)$ возвращает поле скорости, возникающее в результате эволюции поля скорости $\v$ в течении времени $t$ при числе Рейнольдса $\Re$ в системе отсчета, перемещающейся со скоростью $c$. В число параметров функции $\phi$ могут входить также и другие величины, такие как длина периода вдоль трубы $L_x$ или длина периода в угловом направлении $2\pi/n$, если позволить $n$ принимать действительные значения. С практической точки зрения, вычисление функции $\phi$ требует численного интегрирования поля скорости $\v$ по времени. 

Периодические решения обладают двумя непрерывными симметриями: решение остается решение при его смещении вдоль трубы на произвольное расстояние и при смещении по времени на произвольную величину. Для того, чтобы каждое решение \eqref{P_eq} однозначно определялось полем скорости $\v_p$, необходимо дополнить систему двумя уравнениями, исключающими возможность указанных смещений, имеющими вид:
\begin{equation}\label{Pplus_eq}
r_{1,2}(\v_p) = 0.
\end{equation}
В этом случае, если поле скорости однозначно задается $N$ переменными, то количество уравнений в системе равно $N+2$. Для того, чтобы число неизвестных в системе совпадало с числом уравнений, необходимо вместе с полем скорости $\v_p$ положить неизвестными два параметра системы. В случае, если решение ищется при фиксированном значении $\Re$ на заданной сетке, это могу быть $T$ и $c_p$.

Система \eqref{P_eq}, \eqref{Pplus_eq} может быть представлена в виде:
\begin{equation}\label{F_eq}
F(\x) = 0, 
\end{equation}
где вектор $\x = (\v_p, T, c_p, \Re)$ объединяет все переменные. Рассмотрим её полный дифференциал
\begin{equation}\label{dF_eq}
dF = \pd{F}{\x}d\x.
\end{equation}
В случае, если фиксированы все параметры кроме трех, например, $(T, c_p, \Re)$, Якобиан ${\d F}/{\d \x}$ оказывается прямоугольной матрицей, содержащей $(N+2)$ строк и $(N+3)$ столбцов. Система \eqref{dF_eq} недоопределена. В этом случае в окрестности каждого решения существует направление $d\x^*$, для которого $dF = 0$, при движении вдоль которого решение остается решением, так как значение $F$ сохраняется равным нулю. Тогда решения в пространстве трех параметров принадлежат некоторой кривой, задаваемой одним параметром $s$, полученной интегрирование вдоль направления $d\x^*$:
$$
\x_p = \Gamma(s).
$$ 
Аналогично, в пространстве четырех параметров решения принадлежат двухпараметрическому семейству, и т.д. 

Таким образом, решение, соответствующее модельному порыву, принадлежит семейству условно периодических по времени решений. Естественно считать геометрию расчетной области постоянной. Тогда неизвестными остаются три параметра $\Re$, $T$ и $c_p$. В пространстве трех параметров решение принадлежит однопараметрическому множеству. В соответствии с \cite{Avila2013}, продлевая решение в сторону уменьшения числа Рейнольдса, удается достичь точки бифуркации, в которой рождается две ветви решения. (Кривая, которой принадлежат решения, совершает разворот, так что при меньших $\Re$ решение не существует.) Метод продления решения по параметру приведен в следующих разделах. Верхняя ветвь решения характеризуется большей амплитудой пульсаций, скорость перемещения вдоль трубы соответствующего решению порыва оказывается ближе к скорости перемещения турбулентного порыва. Если решения с нижней ветви принадлежат сепаратрисе, верхняя ветвь решения находится внутри области притяжения турбулентного режима течения и может участвовать в организации турбулентного аттрактора. Мы считаем, что результаты, полученные при изучении решения с верхней ветви, имеют большую ценность, так как его характеристики ближе к характеристика турбулентного течения. 

Дополнительно можно отметить, что вместо условия периодичности по времени \eqref{tper_eq} может быть применено условие отражения относительно плоскости $\theta = 0$ со сдвигом на половину периода по времени $T/2$, которое выполнено для модельного порыва. Условие имеет вид:
\begin{multline}\label{shift_eq}
(v_{x,p}, v_{r,p}, - v_{\theta,p})(x,r,\pi/4 + \theta,t) = \\ =(v_{x,p}, v_{r,p}, v_{\theta,p})(x - c_p T/2,r,\pi/4 - \theta,t+T/2).
\end{multline}
В этом случае \eqref{P_eq} уступит место уравнению:
\begin{equation}\label{P2_eq}
(v_{x,p}, v_{r,p}, - v_{\theta,p})(x,r,\pi/4 - \theta) = \phi(\v_p, T/2, c_p, \Re)(x,r,\pi/4 + \theta).
\end{equation}
Его вычисление требует вдвое меньше времени, что может быть существенно при проведении численного исследования. При решении исходной системы \eqref{P_eq} существует возможность потери симметрии \eqref{shift_eq} в процессе продления решения, что исключено при решении системы \eqref{P2_eq}.


\section{Метод Ньютона-Крылова для поиска условно периодических по времени решений поставленной задачи} \label{Newton_seq}

Численно найти условно периодические по времени решения поставленной задачи, удовлетворяющие системе \eqref{F_eq}, позволяет метод Ньютона, обобщенный на многомерный случай. Метод ньютона итерационный и на каждом шаге уточняет существующее приближение к решению. Пусть $\x_m$ --- приближение к решению на шаге $m$, а $\x^*$ --- точное решение. Разложение выражения $F(\x^*)$ в ряд около точки $\x_m$ имеет вид:
\begin{equation}
F(\x^*) = F(\x_m) + \pd{F}{\x}\bigg|_{\x_m} (\x^* - \x_m) + O(\Delta \x^2). 
\end{equation}
Пренебрегая малыми второго порядка, учитывая, что $F(\x^*) = 0$, получим линейную систему на поправку к решению $\Delta \x_m = \x_{m+1} - \x_m$:
\begin{equation}\label{Newton_eq}
\pd{F}{\x}\bigg|_{\x_m} \Delta \x_m = - F(\x_m). 
\end{equation}
Основной задачей при применении метода Ньютона является решение системы \eqref{Newton_eq} и нахождение $\Delta \x_m$. Выполнение шага метода Ньютона завешается вычисление нового приближения к решению $\x_{m+1} = \x_m + \Delta x_m$. 


\def\l{\mathbf{l}}
В случае численного решения задач гидродинамики размерность системы \eqref{Newton_eq} оказывается достаточно большой (в нашем случае $N \sim 10^6$), её решение требует значительных вычислительных ресурсов. Еще более сложной задачей является формирование матрицы Якоби в явном виде. При поиске периодических решений привести аналитическое выражение для Якобиана не представляет возможным. Для формирования матрицы Якоби пользуются тем фактом, что её произведение с произвольным вектором единичной длины $\l$ равно производной исходно функции $F$ вдоль этого направления:
\begin{equation} \label{Jl_eq}
\pd{F}{\x} \l = \pd{F}{\l}. 
\end{equation}
Значение производной функции $F$ может быть получено численно, как конечная разность, по формуле:
\begin{equation}\label{fd_eq}
\pd{F}{\l} \approx \frac{F(\x + \varepsilon \l) - F(\x)}{\varepsilon},
\end{equation}
при достаточно малом значении $\varepsilon$. В расчетах рекомендуется выбирать значение $\varepsilon$ порядка $10^{-7}$ \cite{Viswanath2007}. Вычисление матрицы Якоби сводится к вычислению производной функции $F$ вдоль каждого из базисных направлений, число которых равно числу неизвестных. Вычисление производной $F$ вдоль одного направления требует вычисления функции $F$ в новой точке. Таким образом, формирование матрицы Якоби требует $O(N)$ вызовов функции интегрирования по времени в течении времени $T$, что является крайне трудоемкой задачей. 


Решить линейную систему \eqref{Newton_eq} позволяют итерационные методы, основанные на подпространствах Крылова. В этом случае обращение к матрице Якоби происходит только в форме её умножения на вектор, что в соответствии с \eqref{fd_eq} сводится к вычислению конечно-разностной производной функции $F$ вдоль направления, задаваемого этим вектором. При решении системы вида
\begin{equation}\label{Ax_eq}
Ax = b
\end{equation}
подпространство Крылова $K_i$ представляет собой линейную оболочку $i$ векторов:
\begin{equation}\label{Ki_eq}
K_i = L(b, Ab, A^2b, \dots, A^{i-1}b).
\end{equation}
Имея базис подпространства $K_i$, для того, чтобы построить базис в подпространстве $K_{i+1}$, необходимо выполнить только одно умножение матрицы $A$ на уже известный вектор $A^{i-1}b$. При решении системы \eqref{Ax_eq} приближение к решению ищется в базисе подпространства Крылова. Крыловские методы оказываются эффективны при поиске инвариантных решений с небольшим числом неустойчивых направлений (решение на сепаратрисе имеет одно неустойчивое направление \cite{Avila2013}). Для уточнения решения на порядок требуется только несколько десятков базисных векторов и их число не зависит от размерности подпространства $\x$. Метод Ньютона, в котором для решение линейной системы \eqref{Newton_eq} применяются методы Крыловского типа, называется также методом Ньютона-Крылова \cite{Sanchez2004}. 

\def\r{\mathbf{r}}
В работе был реализован основанный на подпространствах Крылова метод минимизации невязки \cite{EEbook}. Суть метода состоит в том, что на $i$-ой итерации в подпространстве $K_i$ ищется приближение к решению $\x_i$ таким образом, что длина невязки $\r_i = b - A\x_i$ в выбранной норме минимальна. Можно показать, что невязка имеет наименьшую длину в том и только том случае, когда она перпендикулярна пространству $AK_i$. Проще всего опустить перпендикуляр из вектора $b$ на подпространство $AK_i$, имея в этом подпространстве ортогональный базис. Построим последовательность векторов $q_1, \dots, q_i$ таким образом, что они образуют базис в подпространстве $K_i$, а вектора $p_1 = Aq_1, \dots, p_i = Aq_i$ образуют ортогональный базис в подпространстве $AK_i$. Тогда легко может быть построено ортоганальное разложение правой части уравнения вида $b = \alpha_1 p_1 + \dots + \alpha_i p_i + r_i$, где $b_i =  \alpha_1 p_1 + \dots + \alpha_i p_i$ лежит в пространстве $AK_i$, а $r_i$ перпендикулярно ему. Коэффициенты разложения дает формула:
\begin{equation}
\alpha_k = (b,p_k) / (p_k, p_k),
\end{equation}
где $(\ ,\ )$ ---  скалярное произведение, порождающее норму, в которой минимизируется невязка. 
Так как у каждого вектора $p_k$ известен прообраз $q_k$, линейная комбинация векторов $q_k$ с коэффициентами $\alpha_k$ дает приближение к решению, лежащее в пространстве $K_i$
\begin{equation}
\x_i = \alpha_1 q_i + \dots + \alpha_i q_i. 
\end{equation}
Переход на $i+1$ итерацию алгоритма связан с построением базиса подпространств $K_{i+1}$ и $AK_{i+1}$. Для построения ортогонального базиса в подпространстве $AK_{i+1}$ базис подпространства $AK_i$ пополняется новым вектором $p_{i+1}$, полученным ортогонализацией с уже известными базисными векторами $p_1, \dots, p_i$ вектора $Ap_i$. В процессе ортогонализации также может быть получен вектор $q_{i+1}$, являющийся прообразом вектора $p_{i+1}$. 

Критерием остановки итерационного процесса может служить снижение величины невязки ниже заранее заданного порогового значения, либо привышение заранее заданного числа итераций. Переход на новый шаг выполнения метода требует однократного вычисления произведения матрицы $A$ на вектор, связанного при применении к методу Ньютона с вычисление функции $F$ в новой точке. В процессе вычислений необходимо хранить две последовательности векторов $p_1, \dots, p_i$ и $q_1, \dots, q_i$. На первой итерации $q_1 = b$, $p_1 = Ab$. 
Особенности реализации метода Ньютона-Крылова представлены в следующей разделе.  


\section{Особенности реализации метода Ньютона-Крылова}

Метод может быть реализован поверх уже существующего кода. 

Незначительная модификация метода Ньютона Крылова позволяет решать систему $\eqref{P_eq}$ без дополнительных условий $\eqref{Pplus_eq}$ 


\section{Стратегия продления решения}

Реализованный метод Ньютона-Крылова позволяет находить периодические решения, но только в том случае, когда известно достаточно близкое к решению начальное приближение. С произвольными начальными данными метод Ньютона не сходится. В нашем случае в качестве начального приближения может выступать решение на сепаратрисе, найденное в предыдущей главе. Метод Ньютона-Крылова позволяет его уточнить, но кроме этого, оно может быть использовано в качестве начального приближения для решения с близкими значениями параметров, например, числа Рейнольдса. Если смещение по $\Re$ достаточно мало, метод Ньютона сходится, и дает новое периодическое решение. Таким образом, решение может быть продлено в пространстве параметров. Если уже получено несколько решений, приближение к новому может быть построено интерполяцией. В этой работе в основном применялась линейная интерполяция. Если $\x_1$ и $\x_2$ --- известные решения, полученные при числах Рейнольдса $\Re_1$ и $\Re_2$, то приближение к новому решению $\x_3$ при $\Re_3$ может быть получено по формуле 
\begin{equation}
\x_3 = \frac{\x_2(\Re_3 - \Re_1) - \x_1(\Re_3-\Re_2)}{\Re_2 - \Re_1}
\end{equation}

В случае, когда продвижение выполняется по $\Re$, а в качестве определяемых параметров выступают $T$ и $c_f$, преодолеть точку бифуркации и перейти с нижней ветви решения на верхнюю невозможно. Выполнить такой переход позволяет включение в число определяемых параметров $\Re$ вместо $T$ или $c_f$ на выбор. Другим методом решения может быть включение в число определяемых всех трех параметров $\Re$, $T$ и $c_f$. Как показывает практика, сходимость метода Ньютона при этом практически не меняется, хотя с формальной точки зрения она могла быть испорчена. Если в системе уравнений на периодическое решение \eqref{F_eq} неизвестных больше, чем уравнений, теряется однозначность решения. Линейная система для определения шага метода Ньютона \eqref{Newton_eq} получает бесконечно много решений произвольной длины. В процессе её решения может быть получено любое из них, и хотя оно является решением линейной системы, соответствующий ему шаг в пространстве $\x$ может вывести за пределы сходимости метода Ньютона. 

В более общем случае в пространстве параметров $(\Re, T, c_f)$ можно перейти к новой систем координат, одна из осей которой касается кривой $\Gamma(s)$, которой принадлежат решения. Пусть этой оси соответствует переменная $p_0$. Две другие оси, пусть $p_1$ и $p_2$, перпендикулярны оси $p_0$. При поиске нового решения имеет смысл фиксировать значение $p_0$, отделив тем самым новое решение от уже существующих. Тогда $p_1$ и $p_2$ выступают в качестве определяемых параметров и решение ищется в нормальной к кривой $\Gamma(s)$ плоскости. Получить приближение к направлению $p_0$ можно по уже известным решениям. Такой подход позволяет преодолеть точку бифуркации и другие особенности кривой $\Gamma(s)$ в автоматическом режиме. 

На нижней ветви вблизи решения, соответствующего модельному порыву, если в качестве начального приближения к новому решению выступает уже найденное, максимальный шаг по $\Re$ близок к $10$. При использовании линейной интерполяции, шаг может быть увеличен до величины порядка $100$. Хотя по мере продвижения в пространстве параметров шаг, с которым выполняется переход от уже найденного решения к новому, варьируется, можно выделить общую тенденцию, следуя которой по мере приближения к точке бифуркации допустимый шаг уменьшается. Хотя на верхней ветви решения допустимый шаг несколько увеличивается, он остается ниже, чем на нижней, ввиду большей сложности решения с верхней ветви. 
 

\section{Продление модельного порыва по числу Рейнольдса}

\section{Верификация полученных результатов на решении с верней ветви}

\section{Выделенная из турбулентного течения бегущая волна в плоском канале}




