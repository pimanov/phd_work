
\chapter{Верификация полученных результатов}


В предыдущих главах, исследуя модельный порыв, удалось получить некоторые представления о его структуре и механизме самоподдержания. В настоящей главе поднимается вопрос об общности полученных результатов. Наличие пристенных полос повышенной и пониженной скорости является характерной особенности многих инвариантных решений уравнений Навье-Стокса \cite{Kawahara2012} и пристенной турбулентности непосредственно \cite{Kline1967, Smith1983, Schoppa2002}. Это обстоятельство позволяет надеяться, что выделенные при исследовании модельного порыва особенности движения, могут быть в некоторой степени обобщены и на эти случаи. Выполнить строгое исследование турбулентного течения сегодня не представляется возможным, однако полученные результаты могут быть проверены на других инвариантны решения. Продлевая решение, соответствующее модельному порыву, по числу Рейнольдса \cite{Sanchez2004, Viswanath2007, Dijkstra2014}, удалось получить новое локализованное в пространстве условно периодическое по времени решение, характеристики которого оказываются ближе к характеристикам турбулентного течения. Также удалось получить несколько различных бегущих волн в круглой трубе и плоском канале Пуазейля. В настоящей главе представлены методы получения инвариантных решений и результаты их исследования. 


\section{Семейство условно периодических по времени решений поставленной задачи}

Решение поставленной задачи $\v_p(x,r,\theta,t)$ является периодическим по времени с периодом $T$, если оно удовлетворяет условию:
\begin{equation} \label{tper_eq}
\v_p(x,r,\theta,t) = \v_p(x - c_p T, r, \theta, t + T).
\end{equation}
Здесь $c_p$ --- скорость перемещения периодического решения вдоль трубы. В конечно-разностной постановке условие \eqref{tper_eq} может быть сформулировано иначе:
\begin{equation}\label{P_eq}
\phi(\v_p, T, c_p, \Re) - \v_p = 0.
\end{equation}
Здесь функция $\phi(\v, t, c, \Re)$ возвращает поле скорости, возникающее в результате эволюции поля скорости $\v$ в течении времени $t$ при числе Рейнольдса $\Re$ в системе отсчета, перемещающейся со скоростью $c$. В число параметров функции $\phi$ могут входить также и другие величины, такие как длина периода вдоль трубы $L_x$ или длина периода в угловом направлении $L_{\theta} = 2\pi/n$ (в некоторых случаях $n$ может быть действительным). С практической точки зрения, вычисление функции $\phi$ требует численного интегрирования поля скорости $\v$ по времени. 

Периодические решения обладают двумя непрерывными симметриями: решение остается решение при его смещении вдоль трубы на произвольное расстояние и при смещении по времени на произвольную величину. Для того, чтобы каждому решению \eqref{P_eq} соответствовало только одно поле скорости $\v_p$, необходимо дополнить систему двумя уравнениями, исключающими возможность указанных смещений, имеющими вид:
\begin{equation}\label{Pplus_eq}
r_{1,2}(\v_p) = 0.
\end{equation}
В этом случае, если поле скорости однозначно задается $N$ переменными, то количество уравнений в системе равно $N+2$. Для того, чтобы число неизвестных в системе совпадало с числом уравнений, необходимо вместе с полем скорости $\v_p$ положить неизвестными два параметра системы. В случае, если решение ищется при фиксированном значении $\Re$ на заданной сетке, это будут $T$ и $c_p$. Их значение однозначно определяется вместе с полем скорости.

Система \eqref{P_eq}, \eqref{Pplus_eq} может быть представлена в виде:
\begin{equation}\label{F_eq}
F(\x) = 0, 
\end{equation}
где вектор $\x = (\v_p, T, c_p, \Re)$ объединяет все переменные. Рассмотрим её полный дифференциал
\begin{equation}\label{dF_eq}
dF = \pd{F}{\x}d\x.
\end{equation}
В случае, если фиксированы все параметры кроме трех, например, $T$, $c_p$ и $\Re$, Якобиан ${\d F}/{\d \x}$ оказывается прямоугольной матрицей, содержащей $(N+2)$ строк и $(N+3)$ столбцов. Система \eqref{dF_eq} недоопределена. В этом случае в окрестности каждого решения существует направление $d\x^*$, для которого $dF = 0$, при движении вдоль которого решение остается решением, так как значение $F$ сохраняется равным нулю. Тогда решения в пространстве трех параметров принадлежат некоторой кривой, задаваемой одним параметром $s$, полученной интегрирование вдоль направления $d\x^*$:
$$
\x_p = \Gamma(s).
$$ 
Аналогично, в пространстве четырех параметров решения принадлежат двухпараметрическому семейству, и т.д. 

Решение, соответствующее модельному порыву, принадлежит семейству условно периодических по времени решений. Естественно считать геометрию расчетной области постоянной. Тогда неизвестными остаются три параметра $\Re$, $T$ и $c_p$. В пространстве трех параметров решение принадлежит однопараметрическому множеству. В соответствии с \cite{Avila2013}, продлевая решение в сторону уменьшения числа Рейнольдса, удается достичь точки бифуркации, в которой рождается две ветви решения (кривая, которой принадлежат решения, совершает разворот так, что при меньших $\Re$ решение не существует). Метод продления решения по параметру представлен в следующем разделе. Исходное решение принадлежит нижней ветви. Верхняя ветвь решения характеризуется большей амплитудой пульсаций, скорость перемещения вдоль трубы соответствующего решению порыва оказывается ближе к скорости перемещения турбулентного порыва. Если решения с нижней ветви принадлежат сепаратрисе, верхняя ветвь находится внутри области притяжения турбулентного режима течения и может участвовать в организации турбулентного аттрактора. Мы считаем, что результаты, полученные при изучении решения с верхней ветви, имеют б\'{о}льшую ценность, так как его характеристики ближе к характеристика турбулентного течения. 

Отметим, что вместо условия периодичности по времени \eqref{tper_eq} может быть применено условие отражения относительно плоскости $\theta = 0$ со сдвигом на половину периода по времени $T/2$, которое выполнено для модельного порыва. Условие имеет вид:
\begin{multline}\label{shift_eq}
(v_{x,p}, v_{r,p}, - v_{\theta,p})(x,r,\pi/4 + \theta,t) = \\ =(v_{x,p}, v_{r,p}, v_{\theta,p})(x - c_p T/2,r,\pi/4 - \theta,t+T/2).
\end{multline}
В этом случае \eqref{P_eq} уступит место условию:
\begin{equation}\label{P2_eq}
(v_{x,p}, v_{r,p}, - v_{\theta,p})(x,r,\pi/4 - \theta) = \phi(\v_p, T/2, c_p, \Re)(x,r,\pi/4 + \theta).
\end{equation}
Его вычисление требует вдвое меньше времени, что может быть существенно при проведении численного исследования. При решении исходной системы \eqref{P_eq} существует возможность потери симметрии \eqref{shift_eq} в процессе продления решения, что исключено при решении системы \eqref{P2_eq}.


\section{Метод Ньютона-Крылова для поиска условно периодических по времени решений} \label{Newton_seq}

Численно найти условно периодические по времени решения, удовлетворяющие нелинейной системе \eqref{F_eq}, позволяет метод Ньютона, обобщенный на многомерный случай. Метод Ньютона итерационный и на каждом шаге уточняет существующее приближение к решению. Пусть $x_m$ --- приближение к решению на шаге $m$, $x^*$ --- точное решение. Разложение выражения $F(x^*)$ в ряд около точки $x_m$ имеет вид:
\begin{equation}
F(x^*) = F(x_m) + \pd{F}{x}\bigg|_{x=x_m} \Delta x_m^* + O(\Delta x_m^{*2}), 
\end{equation}
где $\Delta x_m^* = x^* - x_m$. Пренебрегая малыми второго порядка, учитывая, что $F(x^*) = 0$, получим линейную систему на поправку к решению $\Delta x_m$:
\begin{equation}\label{Newton_eq}
\pd{F}{x}\bigg|_{x = x_m} \Delta x_m = - F(x_m). 
\end{equation}
Основной задачей при применении метода Ньютона является решение системы \eqref{Newton_eq} и нахождение $\Delta x_m$. Выполнение шага метода Ньютона завешается вычисление нового приближения к решению: 
\begin{equation} \label{end_NK_eq}
x_{m+1} = x_m + \Delta x_m. 
\end{equation}

В случае численного решения задач гидродинамики размерность системы \eqref{Newton_eq} оказывается достаточно большой (в нашем случае $N \sim 10^6$), её решение требует значительных вычислительных ресурсов. Еще более сложной задачей является формирование матрицы Якоби $J(x) = \partial F / \partial x $ в явном виде. При поиске периодических решений привести аналитическое выражение для Якобиана не представляет возможным. Для формирования матрицы Якоби пользуются тем фактом, что её произведение с произвольным вектором единичной длины $l$ равно производной исходно функции $F$ вдоль этого направления:
\begin{equation} \label{Jl_eq}
\pd{F}{x} l = \pd{F}{l}. 
\end{equation}
Значение производной функции $F$ может быть получено численно, как конечная разность, по формуле:
\begin{equation}\label{fd_eq}
\pd{F}{l} \approx \frac{F(x + \varepsilon l) - F(x)}{\varepsilon}.
\end{equation}
В расчетах значение $\varepsilon$ рекомендуется выбирать близким к $10^{-7}$ \cite{Viswanath2007}. Формирование матрицы Якоби сводится к вычислению производной функции $F$ вдоль каждого из базисных направлений, число которых равно числу неизвестных. Вычисление производной $F$ вдоль одного направления требует вычисления функции $F$ в новой точке. Таким образом, формирование матрицы Якоби сводится к $O(N)$ вызовам функции интегрирования по времени, что является крайне трудоемкой задачей. 


Решить линейную систему \eqref{Newton_eq} позволяют итерационные методы, основанные на подпространствах Крылова \cite{Sanchez2004}. В этом случае обращение к матрице Якоби происходит только в форме её умножения на вектор, что в соответствии с \eqref{fd_eq} сводится к вычислению конечно-разностной производной функции $F$ вдоль направления, задаваемого этим вектором. При решении системы вида
\begin{equation}\label{Ax_eq}
Ax = b
\end{equation}
подпространство Крылова $K_i$ представляет собой линейную оболочку $i$ векторов:
\begin{equation}\label{Ki_eq}
K_i = L(b, Ab, A^2b, \dots, A^{i-1}b).
\end{equation}
Имея базис подпространства $K_i$, для того, чтобы построить базис в подпространстве $K_{i+1}$, необходимо выполнить только одно умножение матрицы $A$ на уже известный вектор $A^{i-1}b$. При решении системы \eqref{Ax_eq} приближение к решению ищется в базисе подпространства Крылова. Крыловские методы оказываются эффективны при поиске инвариантных решений с небольшим числом неустойчивых направлений (решение на сепаратрисе имеет одно неустойчивое направление \cite{Avila2013}). Для уточнения решения на порядок требуется только несколько десятков базисных векторов и их число не зависит от $N$. Метод Ньютона, в котором для решение линейной системы \eqref{Newton_eq} применяются методы Крыловского типа, называется также методом Ньютона-Крылова \cite{Sanchez2004}. 

Подпространство Крылова может быть построено только в случае, если в системе \eqref{Ax_eq} матрица $A$ --- квадратная. Число неизвестных в исходной системе \eqref{F_eq} должно быть равно числу уравнений. Этого можно добиться, фиксировав значения всех параметров, кроме двух, например, $T$ и $c_p$. Либо, если определению подлежит большее число параметров, можно дополнить систему \eqref{F_eq} уравнениями, определяющими связь между ними. 

В работе был реализован основанный на подпространствах Крылова метод минимизации невязки ("MINRES" --- "Minimum residual method")\cite{EEbook}. Суть метода состоит в том, что на $i$-ой итерации в подпространстве $K_i$ ищется приближение к решению $x_i$ таким образом, что длина невязки $r_i = b - Ax_i$ в выбранной норме минимальна. Можно показать, что невязка имеет наименьшую длину в том и только том случае, когда она перпендикулярна пространству $AK_i$. Проще всего опустить перпендикуляр из вектора $b$ на подпространство $AK_i$, имея в этом подпространстве ортогональный базис. Построим последовательность векторов $q_1, \dots, q_i$ таким образом, что они образуют базис в подпространстве $K_i$, а вектора $p_1 = Aq_1, \dots, p_i = Aq_i$ образуют ортогональный базис в подпространстве $AK_i$. Тогда легко может быть построено ортоганальное разложение правой части уравнения вида $b = \alpha_1 p_1 + \dots + \alpha_i p_i + r_i$, где $b_i =  \alpha_1 p_1 + \dots + \alpha_i p_i$ лежит в пространстве $AK_i$, а $r_i$ перпендикулярно ему. Коэффициенты разложения дает формула:
\begin{equation}
\alpha_k = (b,p_k) / (p_k, p_k),
\end{equation}
где $(\ ,\ )$ ---  скалярное произведение, порождающее норму, в которой минимизируется невязка. 
Так как у каждого вектора $p_k$ известен прообраз $q_k$, линейная комбинация векторов $q_k$ с коэффициентами $\alpha_k$ дает приближение к решению, лежащее в пространстве $K_i$
\begin{equation}
\x_i = \alpha_1 q_i + \dots + \alpha_i q_i. 
\end{equation}
Переход на $i+1$ итерацию алгоритма связан с построением базиса подпространств $K_{i+1}$ и $AK_{i+1}$. Для построения ортогонального базиса в подпространстве $AK_{i+1}$ базис подпространства $AK_i$ пополняется новым вектором $p_{i+1}$, полученным ортогонализацией с уже известными базисными векторами $p_1, \dots, p_i$ вектора $Ap_i$. В процессе ортогонализации также может быть получен вектор $q_{i+1}$, являющийся прообразом вектора $p_{i+1}$. 

Критерием остановки итерационного процесса при решении линейной системы может служить снижение величины невязки ниже заранее заданного порогового значения, либо превышение заранее заданного числа итераций. Переход на новый шаг выполнения метода требует однократного вычисления произведения матрицы $A$ на вектор, связанного с вычисление производной функции $F$ вдоль одного направления в соответствии с \eqref{Jl_eq}, \eqref{fd_eq}. В процессе вычислений необходимо хранить две последовательности векторов $p_1, \dots, p_i$ и $q_1, \dots, q_i$. На первой итерации $q_1 = b$, $p_1 = Ab$. Аналогично, критерием остановки итерационного процесса метода Ньютона может служить снижение невязки ниже заранее заданной величины, либо превышение заранее заданного числа итераций. Особенности реализации метода Ньютона-Крылова представлены в следующей разделе.  


\section{Особенности реализации метода Ньютона-Крылова}

Методу Ньютона-Крылова, сформулированному в предыдущем разделе, может быть дана физическая интерпретация. Пусть $\v_m, c_m, T_m, \Re_m$ --- поле скорости приближения к решению на шаге $m$ и его параметры --- скорость перемещения решения вдоль трубы, период его изменения по времени и число Рейнольдса, которое в некоторых случаях также может меняться в процессе уточнения решения. Невязка уравнения \eqref{F_eq} представляет собой разность поля скорости $\v_m$ и поля скорости, возникающего из поля $\v_m$ через время $T_m$, а также невязку дополнительных условий \eqref{Pplus_eq}. Интегрирование поля скорости $\v_m$ выполняется в подвижной системе отсчета, перемещающейся со скоростью $c_m$, при $\Re = \Re_m$. Для того, чтобы найти новое приближение к решению, устанавливается связь между вариациями существующего приближения к решению и невязкой. Подбирается такая поправка к решению, которая обнулит невязку. Оказывается, найти поправку к решению можно, зная, как меняется невязка при смещении решения лишь в небольшом числе направлений. При смещении решения в каждом направлении строится очередной вектор подпространства Крылова, причем, первый из них строится при смещении решения в направлении невязки. Последующие вектора подпространства Крылова строятся при смещении решения в направлении изменения невязки на предыдущем шаге. При этом, поправка поля скорости решения всегда представляет собой поле скорости, а к параметрам решения прибавляются значения невязки уравнений \eqref{Pplus_eq}. Подбор условий \eqref{Pplus_eq}, сохраняющих физический смысл операции построения подпространств Крылова, таких, что их невязка имеет смысл скорости или времени, является отдельной задачей. 

Также при реализации метода Ньютона-Крылова необходимо выбрать скалярное произведение в пространстве векторов $x = (\v, T, c, \Re)$. Скалярное произведение двух полей скорости $\v_1$ и $\v_2$ может быть введено естественным образом, как среднее по объему трубы от скалярного произведения соответствующих векторов скорости в каждой точке:
\begin{equation} \label{NK_vdp_eq}
(\v_1, \v_2) =  \frac{1}{V} \int_{V} \v_1 \cdot \v_2 \ d\tau .
\end{equation}
Здесь через $V$ обозначена расчётная область и её объем. Скалярное произведение векторов $x_1 = (\v_1, T_1, c_1, \Re_1)$ и $x_2 = (\v_2, T_2, c_2, \Re_2)$ может быть сконструировано из скалярного произведения \eqref{NK_vdp_eq} по правилу
\begin{equation}
(\x_1, \x_2) = (\v_1, \v_2) + a_1 T_1 T_2 + a_2 c_1 c_2 + a_3 \Re_1 \Re_2,
\end{equation} 
где веса $a_1, a_2, a_3$ подлежат определению. Обоснованный выбор значения параметров $a_1, a_2, a_3$ также представляет собой некоторую задачу. 

В работе в метод Ньютона-Крылова внесены некоторые модификации, позволяющие сохранить физический смысл за каждой из операций, составляющих его, что в свою очередь позволяет упростить его реализацию, избавится от неоднозначности при определении его параметров и в некоторой степени расширить область применения. 

Характерной особенностью метода минимизации невязки является то, что он позволяет получить решение вырожденных системы, если оно существует. Это дает возможность оказаться от дополнительных условий \eqref{Pplus_eq}. Тогда нелинейная система \eqref{F_eq} совпадает с системой \eqref{P_eq}. Её невязка, выступающая в роли вектора $b$ в \eqref{Ax_eq}, представляет собой разность полей скорости, возникающих в моменты времени $t_0$ и $t_0 + T$. Соответственно, скалярное произведение в пространстве правых частей может быть введено естественным образом в соответствии с \eqref{NK_vdp_eq}. Матрица $A$ в \eqref{Ax_eq} в этом случае теряет квадратную форму, что делает невозможным прямое построение подпространств Крылова \eqref{Ki_eq}. Применена следующая модификация метода минимизации невязки, позволяющая решить проблему. Пространство, на которое выполняется проектирование вектора $b$, представляется в виде ортогональной суммы двух подпространств. Первое получено варьированием параметров решения $(c_m, T_m, \Re_m)$ при фиксированном поле скорости $\v_m$. Его базисные вектора находятся прямым вычислением, их количество совпадает с числом параметров решения и в данном случае равно трем. Второе получено варьированием поля скорости $\v_m$ при фиксированных значениях параметров. Его размерность равна $N$, в нем строятся подпространства Крылова для поиска приближения к решению. Метод минимизации невязки модифицируется таким образом, что на первой его итерации система векторов $p_i$, по которой раскладывается вектор $b$, строится по векторам, возникающим в результате варьирования параметров решения $(c_m, T_m, \Re_m)$, подлежащих определению.  Затем система векторов $p_i$ пополняется базисными векторами подпространств Крылова, полученными при вариации поля скорости $\v_m$, как в классическом методе минимизации невязки. Пространство векторов $q_i$ содержит прообразы векторов $p_i$.

Без дополнительных условий \eqref{Pplus_eq} линейная системы \eqref{Ax_eq} имеет бесконечно много решений, представляющих собой линейное многообразие. В процессе решения линейной системы \eqref{Ax_eq} может быть найдено любое из них, в том числе и имеющее достаточно большую длину, при которой соответствующий шаг метода Ньютона выведет за границы области, где линейное приближение имеет силу. Таким образом, метод Ньютона может потерять сходимость. Однако на практике отсутствие условий \eqref{Pplus_eq} на сходимость существенным образом не влияет. Введение в число определяемых параметров дополнительного (например $\Re_m$ к $c_m$ и $T_m$) также ведет к увеличению размерности пространства, которому принадлежат решения линейной системы \eqref{Ax_eq}, однако и в этом случае метод Ньютона позволяет получить решение нелинейной системы. Как будет показано в следующем разделе, в некоторых случаях это имеет смысл. Добавление в число определяемых нового параметра сводится к пополнению системы векторов $p_i$ новым вектором, возникающем в результате варьирования нового параметра. 

Метод Ньютона-Крылова формулируется в предположении, что переменные, определяющие состояние системы, являются фазовыми, то есть каждой точке в пространстве этих переменных соответствует допустимое состояние системы. Дискретное представление поля скорости в реализованном методе решения уравнений движения не удовлетворяют этому требованию. В нем каждая компонента поля скорости представляется её значениями в соответствующих узлах сетки. Такое представление избыточно, так как поле скорости удовлетворяет условию несжимаемости \eqref{eq0_Re} и условию постоянства расхода вдоль трубы \eqref{Q_Re}. С формальной точки зрения внутреннее представление поля скорости в методе Ньютона-Крылова должно отличаться от его представления в программе для интегрирования уравнений движения, однако на практике в этом нет необходимости. В методе Ньютона-Крылова поправки к полю скорости решения во всех случаях вычисляются, как линейная комбинация некоторых других полей скорости, при этом свойство несжимаемости и постоянства расхода сохраняются. Это позволяет использовать в методе Ньютона-Крылова и в программе для решения уравнений движения одно и тоже внутреннее представление. 

В методе Ньютона-Крылова обращение к уравнениям движения происходит лишь при вычислении функции $F$, причем, в соответствии с \eqref{P_eq}, вычисление функции $F$ требует прямого интегрирование уравнений движения жидкости в течении времени $T$. Это позволяет отделить реализацию метода Ньютона-Крылова от реализации метода интегрирования уравнений движения. Программа, реализующая метод Ньютона-Крылова, написана на высокоуровневом языке python, в которой для расчета движения жидкости выполняются обращения к уже существующей программе, написанной на языке Fortran. Программа, реализующая метод Ньютона-Крылова, не зависит от реализации метода интегрирования по времени, и, более того, может быть применена для поиска нелинейных решений в других задачах. 


\section{Метод продления условно периодических по времени решений по параметру}

Реализованный метод Ньютона-Крылова позволяет находить условно периодические решения, но только в том случае, когда известно достаточно близкое к решению начальное приближение. С произвольными начальными данными метод Ньютона не сходится. В нашем случае в качестве начального приближения может выступать решение на сепаратрисе, найденное в предыдущей главе. Метод Ньютона-Крылова позволяет его уточнить, но кроме этого, оно может быть использовано в качестве начального приближения для решения с близкими значениями параметров, например, числа Рейнольдса. Если смещение по $\Re$ достаточно мало, метод Ньютона сходится, и дает новое условно периодическое решение. Таким образом, решение может быть продлено в пространстве параметров. Если уже получено несколько решений, приближение к новому может быть построено интерполяцией. В работе применялась линейная интерполяция, в соответствии с которой новое решение $x_1$ по уже известным решениям $x_2$ и $x_3$ строится по формуле
\begin{equation} \label{interp_eq}
x_1 = (a + 1) x_2 - a x_3, 
\end{equation}
где параметр $a$ определяет длину шага. 

В случае, когда продвижение выполняется по $\Re$, а в качестве определяемых параметров выступают $T$ и $c_f$, преодолеть точку бифуркации и перейти с нижней ветви решения на верхнюю не представляется возможным. Выполнить такой переход позволяет смена определяемых параметров, например, на $T$ и $\Re$. Тогда, продлевая решение по $c_f$, можно преодолеть точку бифуркации. Другим решения может быть включение в число определяемых сразу трех параметров $\Re$, $T$ и $c_f$.  В комбинации с методом линейной интерполяции \eqref{interp_eq} такой подход позволяет себя эффективным.

В более общем случае в пространстве параметров $(\Re, T, c_f)$ можно перейти к новой систем координат, одна из осей которой касается кривой $\Gamma(s)$, которой принадлежат решения. Пусть этой оси соответствует переменная $w_0$. Две другие оси, пусть $w_1$ и $w_2$, перпендикулярны оси $w_0$. При поиске нового решения имеет смысл задавать значение $w_0$, отделив тем самым новое решение от уже существующих. Тогда $w_1$ и $w_2$ выступают в качестве определяемых параметров и решение ищется в нормальной к кривой $\Gamma(s)$ плоскости. Получить приближение к направлению $w_0$ можно по уже известным решениям. Такой подход позволяет преодолеть точку бифуркации и другие особенности кривой $\Gamma(s)$ в автоматическом режиме. 

На нижней ветви вблизи решения, соответствующего модельному порыву, если в качестве начального приближения к новому решению выступает уже найденное, максимальный шаг по $\Re$ близок к $10$. При использовании линейной интерполяции, шаг может быть увеличен до величины порядка $100$. Хотя по мере продвижения в пространстве параметров шаг, с которым выполняется переход от уже найденного решения к новому, варьируется, можно выделить общую тенденцию, следуя которой по мере приближения к точке бифуркации допустимый шаг уменьшается. Хотя на верхней ветви решения допустимый шаг несколько увеличивается, он остается ниже, чем на нижней ветви. 
 

\section{Продление модельного порыва по числу Рейнольдса}

\section{Верификация полученных результатов на решении с верней ветви}

\section{Выделенная из турбулентного течения бегущая волна в плоском канале}

\section{Выводы по главе}


