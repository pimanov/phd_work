\section{Описание численного метода}


Задача решается численно конечно-разностным методом \cite{Nikitin2006Method}. Расчеты выполняются на сетке, однородной в направлениях $x$ и $\theta$. В направление $r$ введено растяжение, что позволяет сгустить сетку около стенки трубы. На равномерной сетке пространственная аппроксимация имеет второй порядок точности, но с введением растяжения порядок аппроксимации несколько падает. 



Численный метод формулируется относительно уравнения \eqref{NS_Re_eq}

Конечно-разностная аппроксимация выполняется для уравнения \eqref{NS_Re_eq}, приведенного к более удобному виду: 
\begin{equation}\label{NS_GL_eq}
\pd{\v}{t} = \v \times \om - \i D - \nabla P + \frac{1}{\Re} \rot \om
\end{equation}
Здесь $\om = \rot \v$ --- вектор завихренности, посчитанный по полю скорости $\v$, $P = p + \v^2/2$ --- полное кинематическое давление. В такой форме оказывается проще учесть граничные условия и кривизну системы координат. 


Конечно-разностная аппроксимация выполнена на так называемых смещённых или разнесенных сетках. Суть подхода состоит в том, что различные компоненты движения относятся к различным точкам сетки. Так, компоненты вектора скорости относятся к центрам граней ячеек. С каждой гранью связана только одна компонента скорости, нормальная к ней. Компоненты вектора завихренности определены в центрах ребер сетки, причем на каждом ребре определена та компонента вектора завихренности, которая с ним сонаправлена. Давление относится к центрам ячеек. Такой подход позволяет естественным образом ввести пространственную аппроксимацию второго порядка точности. 


Интегрирование по времени выполняется полу-неявным методом Рунге-Кутты третьего порядка точности \cite{Nikitin2006Third}. Уравнения Навье-Стокса нелинейны, что не позволяет использовать полностью неявные схемы его интегрирования по времени. Тем не менее, линейные слагаемые в уравнении могут быть разрешены неявно. В явных схемах интегрирования основным источником неустойчивости оказываются вязкие слагаемые. Они определяют максимально допустимый шаг по времени, с которым может быть выполнено интегрирование. Разрешение вязких слагаемых неявно позволяет увеличить его примерно в десять раз. 


На каждом шаге по времени возникает необходимость по известным полям скорости $\v$ и завихренности $\om$ определить поле давления $P$. Давление определяется из условия несжимаемости. Найти его позволяет решение уравнения Пуассона, возникающего после применения к уравнению Навье-Стокса \eqref{NS_GL_eq} оператора дивергенции:
\begin{equation}\label{pressure_eq}
\nabla^2 P = \div(\v \times \om  + \frac{1}{\Re} \rot \om)
\end{equation}
Правая часть этого уравнения известна. В соответствии с постановкой задачи, на поле давления накладываются условия периодичности вдоль однородных направлений и условия на нормальную производную на границах расчетной области. Условия на нормальную производную возникают в результате проектирования уравнения \eqref{NS_GL_eq} на нормальное к стенке направление. 


Метод решения задачи для давления основан на применении быстрого преобразования Фурье в продольном и угловом направлениях, в которых сетка однородна и выставлены периодические граничные условия. Если условие зеркальной симметрии относительно радиального сечения \eqref{Th0_reflect_eq} применены, то в угловом направлении имеет смысл перейти к базису из косинусов. В радиальном направлении система решается методом прогонки. Общая сложность решения задачи для давления составляет $O(n \log n)$, где $n$ --- количество ячеек в сетке. Решение задачи для давления составляет основную вычислительную сложность используемого численного метода. 




