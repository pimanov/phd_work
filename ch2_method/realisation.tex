\section{Реализация численного метода}



Программа была реализован на языке программирования фортран 77 Никитиным Н.В. (научным руководителем). Значительная часть кода, необходимого при анализе полученных результатов и управлении численными экспериментами, была реализована на скриптовом языке python. 


Помимо последовательного варианта программы был реалзован также параллельный. Для этого был использован интерфейс передачи сообщений (MPI), позволяющий обмениваться данными между различными копиями программы, запущенными в разном адресном пространстве. Реализация параллельной программы такова, что труба делится в проольном напрвени

В параллельной версии программы труба в проольном направлении делится на равные части, которые распределяются между процессами. Расчеты выполнялись на суперкомпьютерах <<Чебышев>> и <<Ломоносов>> суперкомпьютерного комплекса МГУ им. М.В.\,Ломоносова. 



