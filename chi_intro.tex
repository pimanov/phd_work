\chapter{Введение}

\section{Актуальность темы} 

Изучение закономерностей движения жидкостей и газов в трубах имеет большое значение как с практической, так и с теоретической точки зрения. Известно, что при небольших скоростях течения жидкость движется ламинарным образом. Такое движение хорошо организовано --- жидкие частицы могут быть объединены в слои, смещающиеся друг относительно друга без перемешивания. Если скорость жидкости достаточно велика, ламинарный режим течения сменяется турбулентным, характеризующемуся наличием беспорядочных пульсаций скорости, давления, и других характеристик. Кроме того, при переходе к турбулентности многие свойства потока, таких как величина трение на стенке или форма среднего профиля скорости, качественно меняются. Как было установлено Осборном Рейнольдсом в работе 1883 года \cite{Reynolds1883}, характер течение определяет безразмерная комбинация параметров, называемая числом Рейнольдса. Если число Рейнольдса $Re=UR/\nu$, вычисленное по максимальной скорости течения $U$, радиусу трубы $R$ и кинематической вязкости жидкости $\nu$, ниже критического значения, близкого к $2000$, жидкость движется ламинарным образом. При больших $\Re$, как правило, движение оказывается турбулентным.

Уже Рейнольдсом было замечено, что турбулентность первоначально проявляется перемежающимся образом, когда участки возмущенного и спокойного движения следуют вдоль трубы друг за другом, практически не меняя своей протяженности. На тот момент причина пространственной локализации турбулентности установлена не была. Подробное экспериментальное исследование локализованных турбулентных структур в трубах было выполнено в \cite{Wygnanski1973}. Установлено, что в разных условиях могут возникать структуры заметно разных типов. Структуры первого типа --- турбулентные порывы ({\it turbulent puffs}) --- появляются при сильной возмущенности потока на входе в трубу в диапазоне $2000<\Re<2700$. Порывы сносятся вниз по потоку со скоростью, близкой к средней скорости течения в трубе, практически не изменяя своей протяженности. Для порыва характерны размытость переднего фронта, на котором скорость на оси трубы постепенно уменьшается от ламинарного значения на 30 -- 40\% и резкость заднего фронта, на котором происходит возвращение к ламинарному течению. В последующей работе \cite{Wygnanski1975} установлено, что при $\Re<2100$ турбулентные порывы подвержены спонтанному исчезновению, а при $\Re>2300$ возможно деление порыва на два следующих друг за другом. Введено понятие {\it равновесного порыва}, характеристики которого не меняются по мере его продвижения вдоль трубы. Согласно \cite{Wygnanski1975} это наблюдается при $2100\leqslant\Re\leqslant2300$. 

Локализованные турбулентные структуры другого типа --- турбулентные пробки ({\it turbulent slugs}) появляются при б\'{о}льших числах Рейнольдса $\Re>3200$ только когда возмущенность потока на входе недостаточна для непосредственного возникновения турбулентности. Тогда возможен переход через турбулентные пробки --- локализованные образования, расширяющиеся по мере сноса вниз по течению. Продвигаясь по трубе, пробки нагоняют друг друга (передний фронт пробки перемещается быстрее заднего), сливаясь в конечном итоге в единую турбулентную область.

В последние годы выполнен ряд подробных экспериментальных и численных исследований характеристик и свойств турбулентных порывов \cite{Priymak2004, Peixinho2006, Hof2006finite, Willis2007, Hof2008, Kuik2010, Avila2011}. Установлено, что турбулентный порыв является нестабильным образованием, склонным либо к исчезновению, либо к делению. С каждой из двух конкурирующих тенденций связано характерное время: среднее время жизни порыва до его исчезновения и среднее время до его разделения. Первое увеличивается с ростом $\Re$, второе уменьшается. Согласно точке зрения \cite{Avila2011}, значение $\Re=\Re^*=2040$, при котором происходит смена доминирования тенденций, является точкой статистического фазового перехода и может быть принята в качестве минимального критического числа Рейнольдса в круглой трубе. При $\Re<\Re^*$ турбулентный порыв скорее погибнет, чем успеет разделиться, так что возникновение развитого турбулентного течения невозможно. Наоборот, при $\Re>\Re^*$ порыв скорее успеет произвести потомство прежде, чем погибнет, что приводит к развитию незатухающего турбулентного движения.

Турбулентный порыв представляет собой интересный гидродинамический объект, который в некотором отношении может рассматриваться как структурная единица турбулентности. Можно сформулировать ряд вопросов, касающихся поведения порыва. До конца не понятен механизм, обуславливающий пространственную локализацию и самоподдержание порыва, неясны причины, побуждающие его к делению или затуханию, неизвестны факторы, определяющие его протяженность и скорость перемещения вдоль трубы.

В последние годы акцент в изучении механизма самоподдержания турбулентности в пристенных течениях смещается от лабораторного эксперимента в сторону эксперимента вычислительного, основанного на численном решении уравнений Навье--Стокса. Турбулентные порывы впервые были рассчитаны в \cite{Priymak2004}, где было показано, что пространственная локализация является внутренним свойством решений уравнений Навье--Стокса при переходных числах Рейнольдса, а не является следствием специальных начальных условий. Попытка объяснения механизма сапоподдержания турбулентного порыва была предпринята в \cite{Shimizu2009}. В системе отсчета связанной с порывом, пульсации в осевой части трубы сносятся вниз по потоку. Их нелинейное взаимодействие порождает медленно меняющиеся полосчатые структуры, концентрирующиеся в пристенной области трубы, где относительная скорость течения отрицательна. Из-за этого полосчатые структуры отстают от порыва. В хвостовой части порыва в областях расположения полос замедления образуются интенсивные сдвиговые слои с точкой перегиба в профиле скорости, где в силу неустойчивости типа Кельвина--Гельмгольца порождаются мелкомасштабные пульсации, попадающие в приосевую область трубы и сносящиеся вниз по потоку. Так, согласно \cite{Shimizu2009}, выглядит цикл самопроизводства турбулентных пульсаций внутри порыва и цикл самоподдержания самой этой структуры.

Идеализированная схема, предложенная в \cite{Shimizu2009}, выглядит вполне правдоподобно, однако, на наш взгляд, сделанные выводы в должной мере не подкреплены фактическими данными. Реальная динамика порыва сложнее и неопределеннее. Ее изучение осложнено в первую очередь стохастичностью процесса, когда отдельные его фазы следуют друг за другом случайным образом. В этих условиях определенная ясность может быть получена из анализа более простых структур, аппроксимирующих порыв, недавно найденных в \cite{Skufca2006, Avila2013}. Это предельные решения, возникающие на сепаратрисе, разделяющей в фазовом пространстве области притяжения решений, соответствующих ламинарному и турбулентному режимам течения. Такие решения, наследуя ряд качественных характеристик турбулентного порыва, оказываются периодическими по времени в системе отсчета, перемещающейся вдоль трубы с постоянной скоростью. Мы будем называть такие структуры {\it модельными порывами}. Простота поведения позволяет провести исчерпывающее исследование свойств модельного порыва, которые по мнению автора проясняют определенные детали поведения турбулентного порыва. 

Решение, соответствующее модельному порыву, может быть непрерывным образом продлено по параметрам, что позволяет получить новые условно периодические решения, характеристики которых могут оказаться ближе к характеристика турбулентного течения \cite{Viswanath2007, Dijkstra2014}. Продлевая решение, соответствующее модельному порыву, в сторону уменьшения числа Рейнольдса, удается достичь точки бифуркации, в которой возникает две ветви решения, и перейти с нижней на верхнюю \cite{Avila2013}. Решения с верхней ветви сохраняют свойство пространственной локализации и простое поведение во времени, однако их характеристики оказываются ближе к характеристика турбулентного течения. Если нижняя ветвь решения проходит по границе области притяжения, соответствующей турбулентному режиму течения, решения с верхней ветви могут участвовать в организации турбулентного аттрактора. Мы полагаем, что сравнение решений с нижней и верхней ветвей позволит выделить закономерности более общего характера. 

В модельном порыве выделяются полосы повышенной и пониженной скорости, вытянутые вдоль потока. Такие полосы являются характерной особенностью многих пристенных турбулентных течений \cite{Klebanoff1962, Kline1967}. С ними связывают цикл самоподдержания пристенной турбулентности \cite{Hamilton1995, Waleffe1997, Schoppa2002}. Результаты, полученные при изучении модельного порыва, могут оказаться полезными для понимания не только турбулентного порыва, но и более общего класса пристенных турбулентных течений. 


\section{Цель и задачи диссертационной работы}

Работа направлена на определение закономерностей турбулентного движения жидкости, обеспечивающих существование турбулентного порыва; выявление причин, определяющих его форму и основные характеристики. 

С этой целью в работе воспроизводится модельный порыв. Сохраняя ряд особенностей турбулентного порыва, модельный порыв имеет более простую форму и динамику. В подходящей подвижной системе отсчета он оказывается периодическим по времени, что позволяет выполнить его детальное исследование и строго обосновать полученные результаты. 

С целью установления общности полученных при изучении модельного порыва результатов, в работе найдено несколько дополнительных инвариантных решений уравнений Навье-Стокса в геометрии круглой трубы. Продлевая решение, соответствующее модельному порыву, по числу Рейнольдса, были получены новые условно периодические по времени решения, приближающие турбулентный порыв. Также было получено несколько нелинейных бегущих волн. 


\section{Метод исследования и достоверность результатов}

В данной работе движение жидкости воспроизводится и исследуется численно, путем решения полных трехмерных уравнений Навье-Стокса для несжимаемой жидкости. Постановка задачи приведена в разделе \ref{math_section}. Возможность адекватного воспроизведения турбулентных порывов в численных расчетах продемонстрирована в \cite{Priymak2004}. Численный метод, применяемый в работе, совмещает конечно-разносную аппроксимацию второго порядка точности по пространственным переменным и полу-неявный метод Рунге-Кутты интегрирования по времени \cite{Nikitin2006, Nikitin2006third}. Описание метода приведено в разделе \ref{num_method}. Метод используется в лаборатории Общей аэродинамики института механики МГУ уже более 20 лет и хорошо себя зарекомендовал. Программный код, реализующий метод, написан Никитиным Н.В. и отлажен в процессе решения большого числа задач. Реализована параллельная версия программы, что дало возможность использовать ресурсы суперкомпьютерного комплекса МГУ. Подтверждают качество кода и адекватность численного метода результаты моделирования турбулентного течения в трубе при переходных числах Рейнольдса, приведенные в разделе \ref{puff_calc}. Турбулентность в расчетах принимает форму порывов, характеристики которых совпадают с представленными в литературе данными. 

Основным объектом исследования в работе выступает модельный порыв. Его характерным свойством является пространственная локализации и периодическое поведение во времени. Он возникает на сепаратрисе, отделяющей области притяжения решений, соответствующих ламинарному и турбулентному режимам течения. Решение на сепаратрисе не устойчиво и не воспроизводимо в эксперименте, однако оно может быть получено численно. Метод получения модельного порыва описан в \cite{Avila2013}, он опирается на метод прямого численного моделирования движения жидкости. В наших расчетах при наложении дополнительных условий симметрии \cite{Avila2013} решение на сепаратрисе действительно выходит на периодический режим. Характеристики полученного модельного порыва совпадают с приведенными в литературе \cite{Avila2013, Chantry2014}. Совпадение результатов подтверждает, что модельный порыв является решение математической задачи, основанной на уравнениях Навье-Стокса, и не зависит от выбора численного метода и параметров расчета, при которых он был получен. В \cite{Avila2013} был использован полностью спектральный метод \cite{Meseguer2007}, а также спектрально-конечно-разностный метод \cite{Willis2009}, в то время как нами применен полностью конечно-разностный метод. Расчеты проводились на трех различных сетках. В \cite{Chantry2014} модельный порыв был воспроизведен спектрально-конечно-разностным методом \cite{Willis2009}, авторы также сообщают о совпадении результатов с приведенными в \cite{Avila2013}. 

В работе был реализован метод Ньютона-Крылова \cite{Viswanath2007, Dijkstra2014}, позволивший уточнить периодическое решение и, продлевая его по числу Рейнольдса, перейти с нижней ветви на верхнюю. Решение с верхней ветви сохраняет пространственную локализацию и временное поведение, однако его характеристики оказываются ближе к характеристикам турбулентного течения. Подтверждением того, что выбранное для анализа решение с верхней ветви не зависит от численного метода и его параметров, является тот факт, что его характеристики, полученные на трех различных сетках, совпадают между собой и с результатами \cite{Avila2013}. Также, используя метод Ньютона-Крылова и метод поиска решения на сепаратрисе, в работе было получено несколько бегущих волн в плоском канале. Сделанные на основе изучения модельного порыва выводы оказываются справедливы как для решения с верхней ветви, так и для полученных бегущих волн, что подтверждает общность результатов работы. 

В качестве недостатка работы можно отметить отсутствие проверки сделанных выводов непосредственно на турбулентном течении, однако методика такого исследования пока не проработана. Также, все исследованные решения обладают симметрией отражения в трансверсальном направлении, что может привести к выделению закономерностей, навязанных такой симметрией. 


\section{Научная новизна}

Автором работы были описаны основные характеристики модельного порыва, а также выделен механизм его самоподдержания. 

Полученные результаты дополняют существующие представления о механизме самоподдержания пристенных турбулентных структур. В частности, в работе был выделен нелинейный механизм образования продольных вихрей, ответственных за формирование полос повышенной и пониженной скорости. В результате линейной неустойчивости полосчатого профиля скорости в потоке возникают пульсации. Существование продольных вихрей изменяет форму пульсаций так, что их нелинейное взаимодействие усиливает эти вихри. Таким образом, для адекватного воспроизведения формы пульсационной составляющей движения необходимо учитывать наличие продольных вихрей. 

\section{Практическая ценность полученных результатов}

Полученные в работе результаты могут помочь объяснить причины пространственной локализации турбулентного порыва, его форму, скорость перемещения вдоль трубы и другие свойства. Представления о механизме его самоподдержания могут быть полезны для предсказания свойств турбулентного потока, оценки влияния на поток тех или иных конструкционных особенностей, разработки методов управления турбулентным течением. 


\section{Положения, выносимые на защиту}

\section{Апробация работы и публикации}

Основные результаты, полученные в диссертации, докладывались более, чем на 20 конференциях: 
\begin{itemize}
\item Конференция-конкурс молодых ученых НИИ механики МГУ имени М.В. Ломоносова (НИИ мех МГУ, 2014-2016); 
\item Международная научная конференция студентов, аспирантов и молодых ученых "Ломоносов" (МГУ, Москва, 2014-2017); 
\item Конференция "Ломоносовские чтения" (НИИ мех МГУ, 2014-2017); 
\item Международная конференция "Нелинейные задачи теории гидродинамической устойчивости и турбулентность" (Звенигород, 2014, 2016); 
\item Школа-семинар "Современные проблемы аэрогидродинамики" (Сочи, 2014, 2016);  
\item XI Всероссийский съезд по фундаментальным проблемам теоретической и прикладной механики (Казань, 2015);
\item 7th International Symposium on Bifurcations and Instabilities in Fluid Dynamics (Paris, 2015);
\item 15th European Turbulence Conference (Delft, Netherlands, 2015); 
\item 18th International Conference on the Methods of Aerophysical Research (Пермь, 2016).
\end{itemize}

Результаты диссертационной работы докладывались автором и обсуждались на {\bf научных семинарах}:
\begin{itemize}
\item Научный семинар лаборатории общей аэродинамики института механики МГУ под руководством Никитина Н.В. 
\item Семинар <<Суперкомпьютерные технологии в науке, образовании и промышленности>> на базе научно-образовательного центра <<Суперкомпьютерные технологии>> под руководством В.А.Садовничий. 
\end{itemize}

По материалом диссертации опубликовано более 30 работ, в том числе 3 статьи в журналах из {\bf списка ВАК}:
\begin{itemize}
\item Н.В. Никитин, В.О. Пиманов, Численное исследование локализованных структур в трубах // Изв. РАН. МЖГ. 2015. № 5. С. 64-75;
\item Н.В. Никитин, В.О. Пиманов, Локализованные турбулентные структуры в круглой трубе // Учен.  зап.  Казан.  ун-та.  Сер.  Физ.-матем.  науки. 2015. том 157. книга 3. C. 111–116;
\item Н.В. Никитин, В.О. Пиманов, О поддержании колебаний в локализованных турбулентных структурах в трубах // принята к публикации в Изв. РАН. МЖГ. 2017. 
\end{itemize}
3 статьи в сборниках трудов конференций:
\begin{itemize}
\item В.О. Пиманов, Пространственно-локализованные турбулентные структуры в круглой трубе // Труды конференции-конкурса молодых ученых 13-16 октября 2014 г. готовится к публикации. 
\item В.О. Пиманов, О механизме самоподдержания локализованных турбулентных структур в трубах // Труды конференции-конкурса молодых ученых 12-14 октября 2015 г под редакцией академика РАН А.Г. Куликовского, профессора В.А. Самсонова. Издательство Московского университета. Москва. 2016. С. 44-51; 
\item В.О. Пиманов, Некоторые детали механизма самоподдержания турбулентности в пристенных течениях // Труды конференции-конкурса молодых ученых 10-12 октября 2016 г. С. 46-53. принят к печати. 
\end{itemize}
18 тезисов научных конференций, а также статья в сборнике "Суперкомпьютерные технологии в науке, образовании и промышленности"
\begin{itemize}
\item В.О. Пиманов, Суперкомпьютерное моделирование --- путь к пониманию турбулентности // Суперкомпьютерные технологии в науке, образовании и промышленности. том 7. Издательство Московского университета. Москва. 2017. С.  163-170. 
\end{itemize}


Также работа отмечена наградами:
\begin{itemize}
\item Диплом второй степени конференции-конкурса молодых ученых НИИ механики МГУ 2014 года;
\item Лучший доклад подсекции "Гидродинамика" секции "Механика" на конференции "Ломоносов-2015";
\item Диплом первой степени за лучшую работу аспиранта конференции-конкурса молодых ученых НИИ механики МГУ 2015 года;
\item Диплом третьей степени конференции-конкурса молодых ученых НИИ механики МГУ 2015 года;
\item Вторая премия конкурса молодых научных сотрудников МГУ~им.~М.В.~Ломоносова 2015 года;
\item Диплом первой степени за лучшую работу аспиранта конференции-конкурса молодых учёных НИИ механики МГУ 2016 года;
\item Диплом третьей степени Конференции-конкурса молодых учёных НИИ механики МГУ 2016 года;
\item Лучший доклад подсекции "Гидродинамика" секции "Механика" конференции "Ломоносов-2017".
\end{itemize}


\section{Личный вклад}

Научному руководителю, Никитину Н.В., принадлежит идея исследования, реализация численного метода для прямого расчета движения жидкости. Диссертанту принадлежат численные расчеты и интерпретация полученных результатов. 


\section{Структура и объем диссертации}

\section{Благодарности}


