\renewcommand \thechapter {i}
\addcontentsline{toc}{chapter}{Введение}
\thispagestyle{empty}
\phantom{.}
\chapter*{Введение}


Изучение закономерностей движения жидкостей и газов в трубах имеет большое значение как с практической, так и с теоретической точки зрения. Известно, что при небольших скоростях течения жидкость в трубах движется ламинарным образом. Такое движение хорошо организовано --- жидкие частицы могут быть объединены в слои, смещающиеся друг относительно друга без перемешивания. При достаточно больших скоростях течения, ламинарный режим сменяется турбулентным, характеризующимся наличием беспорядочных пульсаций скорости, давления, и других характеристик. Кроме того, при переходе к турбулентности многие свойства потока, такие как величина трение на стенке или форма среднего профиля скорости, качественно меняются. Как было установлено Осборном Рейнольдсом в работе 1883 года \cite{Reynolds1883}, характер течение определяет безразмерная комбинация параметров, называемая ныне числом Рейнольдса. Если число Рейнольдса $Re=UR/\nu$, вычисленное по максимальной скорости течения $U$, радиусу трубы $R$ и кинематической вязкости $\nu$, ниже критического значения, близкого к $2000$, жидкость движется ламинарным образом. При б\'{о}льших $\Re$ движение, как правило, оказывается турбулентным.

Уже Рейнольдсом было замечено, что турбулентность первоначально проявляется перемежающимся образом, когда участки возмущенного и спокойного движения следуют вдоль трубы друг за другом, практически не меняя своей протяженности. На тот момент причина пространственной локализации турбулентности установлена не была. Подробное экспериментальное исследование локализованных турбулентных структур в трубах было выполнено в \cite{Wygnanski1973}. Установлено, что в разных условиях могут возникать структуры заметно разных типов. Структуры первого типа --- турбулентные порывы ({\it turbulent puffs}) --- появляются при сильной возмущенности потока на входе в трубу в диапазоне $2000<\Re<2700$. Порывы сносятся вниз по потоку со скоростью, близкой к средней скорости течения, практически не меняя свою протяженность. Для порыва характерны размытость переднего фронта, на котором скорость на оси трубы плавно уменьшается от ламинарного значения на 30 -- 40\%, и резкость заднего фронта, на котором происходит возвращение к ламинарному течению. В последующей работе \cite{Wygnanski1975} установлено, что при $\Re<2100$ турбулентные порывы подвержены спонтанному исчезновению, а при $\Re>2300$ возможно деление порыва на два следующих друг за другом. Введено понятие {\it равновесного порыва}, характеристики которого не меняются по мере его продвижения вдоль трубы. Согласно \cite{Wygnanski1975} это наблюдается при $2100\leqslant\Re\leqslant2300$. 

Локализованные турбулентные структуры другого типа --- турбулентные пробки ({\it turbulent slugs}) появляются при б\'{о}льших числах Рейнольдса $\Re>3200$ только когда возмущенность потока на входе недостаточна для непосредственного возникновения турбулентности. Тогда возможен переход через турбулентные пробки --- локализованные образования, расширяющиеся по мере сноса вниз по течению. Продвигаясь по трубе, пробки нагоняют друг друга (передний фронт пробки перемещается быстрее заднего), сливаясь в конечном итоге в единую турбулентную область.

В последние годы выполнен ряд подробных экспериментальных и численных исследований характеристик и свойств турбулентных порывов \cite{Priymak2004, Peixinho2006, Hof2006finite, Willis2007, Hof2008, Kuik2010, Avila2011}. Установлено, что турбулентный порыв является нестабильным образованием, склонным либо к исчезновению, либо к делению. С каждой из двух конкурирующих тенденций связано характерное время: среднее время жизни порыва до его исчезновения и среднее время до его разделения. Первое увеличивается с ростом $\Re$, второе уменьшается. Согласно точке зрения \cite{Avila2011}, значение $\Re=\Re^*=2040$, при котором происходит смена доминирования тенденций, является точкой статистического фазового перехода и может быть принята в качестве минимального критического числа Рейнольдса в круглой трубе. При $\Re<\Re^*$ турбулентный порыв скорее погибнет, чем успеет разделиться, так что возникновение развитого турбулентного течения невозможно. Наоборот, при $\Re>\Re^*$ порыв скорее успеет произвести потомство прежде, чем погибнет, что приводит к развитию незатухающего турбулентного движения.

Турбулентный порыв представляет собой интересный гидродинамический объект, который в некотором отношении может рассматриваться как структурная единица турбулентности. Можно сформулировать ряд вопросов, касающихся поведения порыва. До конца не понятен механизм, обуславливающий пространственную локализацию и самоподдержание порыва, неясны причины, побуждающие его к делению или затуханию, неизвестны факторы, определяющие его протяженность и скорость перемещения вдоль трубы.

В последние годы акцент в изучении механизма самоподдержания турбулентности в пристенных течениях смещается от лабораторного эксперимента в сторону эксперимента вычислительного, основанного на численном решении уравнений Навье--Стокса. Турбулентные порывы впервые были рассчитаны в \cite{Priymak2004}, где было показано, что пространственная локализация является внутренним свойством решений уравнений Навье--Стокса при переходных числах Рейнольдса, а не является следствием специальных начальных условий. Попытка объяснения механизма самоподдержания турбулентного порыва была предпринята в \cite{Shimizu2009}. В системе отсчета связанной с порывом, пульсации в осевой части трубы сносятся вниз по потоку. Их нелинейное взаимодействие порождает медленно меняющиеся полосчатые структуры, концентрирующиеся в пристенной области трубы, где относительная скорость течения отрицательна. Из-за этого полосчатые структуры отстают от порыва. В хвостовой части порыва в областях расположения полос замедления образуются интенсивные сдвиговые слои с точкой перегиба в профиле скорости, где в силу неустойчивости типа Кельвина--Гельмгольца порождаются мелкомасштабные пульсации, попадающие в приосевую область трубы и сносящиеся вниз по потоку. Так, согласно \cite{Shimizu2009}, выглядит цикл самопроизводства турбулентных пульсаций внутри порыва и цикл самоподдержания самой этой структуры.

Идеализированная схема, предложенная в \cite{Shimizu2009}, выглядит вполне правдоподобно, однако, на наш взгляд, сделанные выводы в должной мере не подкреплены фактическими данными. Реальная динамика порыва сложнее и неопределеннее. Ее изучение осложнено в первую очередь стохастичностью процесса, когда отдельные его фазы следуют друг за другом случайным образом. В этих условиях определенная ясность может быть получена из анализа более простых структур, аппроксимирующих порыв, недавно найденных в \cite{Skufca2006, Avila2013}. Это предельные решения, возникающие на сепаратрисе, разделяющей в фазовом пространстве области притяжения решений, соответствующих ламинарному и турбулентному режимам течения. Такие решения, наследуя ряд качественных характеристик турбулентного порыва, оказываются периодическими по времени в системе отсчета, перемещающейся вдоль трубы с постоянной скоростью. Мы будем называть такие структуры {\it модельными порывами}. Простота поведения позволяет провести исчерпывающее исследование свойств модельного порыва, которые, как мы полагаем, проясняют определенные детали поведения турбулентного порыва. 

Продолжение модельного порыва по числу Рейнольдса позволяет получить новые условно периодические решения уравнений Навье-Стокса (периодические в подходящей подвижной системе отчета), имеющие пространственно-локализованную структуру \cite{Avila2013}. Кроме того, в настоящее время известно достаточно большое число решений, имеющих вид трехмерной бегущей волны (периодичных вдоль потока и стационарных в подходящей подвижной системе отсчета) \cite{Kawahara2012}. Такие решения также допускаю детальное исследование и могут быть использованы для установления универсальности выделенных при исследовании модельного порыва закономерностей. 

Характерной особенностью модельного порыва является наличие вытянутых вдоль потока областей, скорость жидкости внутри которых существенно выше или ниже среднего значения --- полос повышенной и пониженной скорости. Такие полосы являются неотъемлемым элементом всех сценариев поддержания пристенной турбулентности \cite{Hamilton1995, Waleffe1997, Schoppa2002}, что дает основания полагать, что полученные при изучении модельного порыва результаты могут быть полезными для понимания динамики не только турбулентного порыва, но и более общего класса пристенных турбулентных течений. 


{\bf Цели диссертационной работы} 	
Выявление механизмов, ответственных за возникновение и  самоподдержание турбулентных порывов.

состоит в определении закономерностей турбулентного движения жидкости, обеспечивающих существование турбулентного порыва; выявление причин, определяющих форму порыва и его основные характеристики. 

С этой целью численно воспроизводится и исследуется модельный порыв. Сохраняя ряд качественных особенностей турбулентного порыва, модельный порыв имеет более простую форму и динамику. В подвижной системе отсчета он меняется во времени периодическим образом, что позволяет выполнить его детальное исследование и строго обосновать полученные результаты. 

Метод продолжения по параметру позволяет получить семейство решений уравнений Навье-Стокса

Также в работе получены некоторые другие инвариантные решения уравнений Навье-Стокса в геометрии круглой трубы и плоского канала. Их анализ позволяет оценить общность полученных при исследовании модельного порыва результатов. 


{\bf Метод исследования и достоверность результатов.}
В работе движение жидкости воспроизводится и исследуется численно, путем решения полных трехмерных уравнений Навье-Стокса для несжимаемой жидкости. Возможность адекватного описания турбулентных порывов численными решениями уравнений Навье-Стокса впервые показана в \cite{Priymak2004}. Применяемый нами численный метод совмещает конечно-разностную аппроксимацию второго порядка точности по пространственным переменным и полу-неявный метод Рунге-Кутты третьего порядка точности интегрирования по времени \cite{Nikitin2006, Nikitin2006third} (детали в разделе \ref{num_method}). Метод используется в лаборатории Общей аэродинамики института механики МГУ и других лабораториях уже более 20 лет и хорошо себя зарекомендовал. Код программы, реализующей метод, написан Н.В.\,Никитиным и отлажен в процессе решения большого числа задач. Расчеты выполнены с привлечением ресурсов суперкомпьютерного комплекса МГУ. Качество программной реализации численного метода и его адекватность целям работы подтверждают результаты моделирования турбулентного течения в трубе при переходных числах Рейнольдса, приведенные в разделе \ref{puff_calc}. Турбулентность в расчетах принимает форму порывов, характеристики которых совпадают с представленными в литературе данными. 

Основным объектом исследования в работе выступает модельный порыв. Решение, соответствующее модельному порыву, является предельным состоянием решения, эволюционирующего на сепаратрисе, отделяющей в фазовом пространстве области притяжения решений, соответствующих ламинарному и турбулентному режимам течения. Решение на сепаратрисе неустойчиво и не может быть получено в эксперименте, но может быть найдено численно. Метод нахождения решения на сепаратрисе приведен в разделе \ref{edge_seq}. В согласии с \cite{Avila2013} решение на сепаратрисе выходит на условно периодический режим и имеет пространственно-локализованную структуру. Характеристики модельного порыва, полученные на нескольких расчетных сетках, согласуются друг с другом и с приведенными в литературе значениями. Совпадение результатов подтверждает, что полученное численно решение приближает соответствующее ему решение уравнений Навье--Стокса и не зависит от выбора численного метода и параметров расчета. В \cite{Avila2013} применен полностью спектральный метод \cite{Meseguer2007} и спектрально-конечно-разностный метод \cite{Willis2009}. В нашей работе применен конечно-разностный метод~\cite{Nikitin2006}. В \cite{Chantry2014} модельный порыв воспроизведен спектрально-конечно-разностным методом \cite{Willis2009}, авторы также сообщают о совпадении результатов с \cite{Avila2013}. 

Оценить степень универсальности полученных при исследовании модельного порыва результатов позволяет анализ других решений уравнений Навье-Стокса, имеющие простое временное поведение. Можно надеяться, что общие для многих решений особенности движения могут иметь место в турбулентном течении. В работе реализован метод Ньютона-Крылова \cite{Viswanath2007, Dijkstra2014} для поиска условно периодических решений (см. раздел  \ref{Newton_seq}). Этот метод позволяет уточнять уже найденные решения и продолжать их по параметру. В согласии с \cite{Avila2013} продолжение модельного порыва по числу Рейнольдса позволило получить новые условно периодические решения уравнений Навье-Стокса. Процедура продолжения модельного порыва по числу Рейнольдса описана в разделе \ref{contin_sec}. Сравнение результатов, полученных на нескольких расчетных сетках, с результатами \cite{Avila2013} позволяет делать выводы о соответствии найденных численных решений решениям уравнений Навье--Стокса. Также, применение метода поиска решения на сепаратрисе и метода Ньютона-Крылова позволило найти три семейства решений в виде бегущих волн (см. разделы \ref{pipeTW_seq}, \ref{ductTW_seq}). Одно семейство решений относится к течению в круглой трубе и два --- к течению в плоском канале. Основные закономерности, выделенные при изучении модельного порыва, оказываются справедливы для всех исследованных решений, что в некоторой степени подтверждает их универсальность. 

{\bf Результаты работы и их научная новизна.}

В работе воспроизведен модельный порыв, описаны его основные характеристики и внутренняя структура. При изучении модельного порыва выделен механизм его самоподдержания. Выделенный механизм был обнаружен также в других инвариантных решениях, исследованных в работе. Это дает основания полагать, что он имеет место в турбулентных течениях и, в частности, в турбулентном порыве.

Полученные результаты подтверждают и дополняют существующие представления о механизме самоподдержания пристенных турбулентных структур. В частности, в работе был выделен нелинейный механизм образования продольных вихрей, ответственных за формирование полос повышенной и пониженной скорости. В результате линейной неустойчивости полосчатого профиля скорости в потоке возникают пульсации. Существование продольных вихрей изменяет форму пульсаций так, что их нелинейное взаимодействие поддерживает эти вихри. 

По результатам работы сделана рекомендация, в соответствии с которой для адекватного воспроизведения формы пульсационной составляющей движения, возникающей в результате линейной неустойчивости полосчатого профиля скорости, необходимо учитывать наличие продольных вихрей. В частности, в работе \cite{Schoppa2002}, одной из наиболее авторитетных, посвященных изучению механизма самоподдержания пристенной турбулентности, при исследовании среднего течения на устойчивость влияние стационарных продольных вихрей игнорируется, что не позволяет выделить механизм их образования. 


{\bf Практическая ценность полученных результатов.}

Полученные в работе результаты могут помочь объяснить пространственную локализацию турбулентного порыва, выделить причины, определяющие его форму, скорость перемещения вдоль трубы и другие характеристики. Представления о механизме его самоподдержания могут быть полезны для предсказания свойств турбулентных потоков, оценки влияния на поток тех или иных конструкционных особенностей, разработки методов управления турбулентными течениями. 


{\bf Выносимые на защиту положения.}

\begin{itemize}
\item Представления о внутренней структуре модельного порыва, механизм его самоподдержания. 
\item Механизм образования продольных вихрей, поддерживающих существование полос повышенной и пониженной скорости. 
\end{itemize}

{\bf Апробация работы и публикации.}

Результаты, полученные в диссертации, были представлены на 19 конференциях: 
\begin{itemize}
\item Конференция-конкурс молодых ученых НИИ механики МГУ имени М.В. Ломоносова (НИИ мех МГУ, 2014, 2015, 2016); 
\item Международная научная конференция студентов, аспирантов и молодых ученых <<Ломоносов>> (МГУ, Москва, 2014, 2015, 2016, 2017); 
\item Конференция <<Ломоносовские чтения>> (НИИ мех МГУ, 2014, 2015, 2016, 2017); 
\item Международная конференция <<Нелинейные задачи теории гидродинамической устойчивости и турбулентность>> (Звенигород, 2014, 2016); 
\item Школа-семинар <<Современные проблемы аэрогидродинамики>> (Сочи, 2014, 2016);  
\item XI Всероссийский съезд по фундаментальным проблемам теоретической и прикладной механики (Казань, 2015) --- пленарный доклад;
\item 7th International Symposium on Bifurcations and Instabilities in Fluid Dynamics (Paris, 2015);
\item 15th European Turbulence Conference (Delft, Netherlands, 2015); 
\item 18th International Conference on the Methods of Aerophysical Research (Пермь, 2016).
\end{itemize}

Также результаты диссертационной работы докладывали на {\bf семинарах}:
\begin{itemize}
\item Научный семинар лаборатории общей аэродинамики института механики МГУ под руководством  Н.В.  Никитина;
\item Семинар <<Суперкомпьютерные технологии в науке, образовании и промышленности>> на базе научно-образовательного центра <<Суперкомпьютерные технологии>> под руководством В.А. Садовничего. 
\end{itemize}

По материалом диссертации опубликовано около 30 научных работ, в том числе 3 статьи в журналах из {\bf списка ВАК}:
\begin{itemize}
\item Н.В. Никитин, В.О. Пиманов, Численное исследование локализованных структур в трубах // Изв. РАН. МЖГ. 2015. № 5. С. 64-75;
\item Н.В. Никитин, В.О. Пиманов, Локализованные турбулентные структуры в круглой трубе // Учен.  зап.  Казан.  ун-та.  Сер.  Физ.-матем.  науки. 2015. том 157. книга 3. C. 111–116;
\item Н.В. Никитин, В.О. Пиманов, О поддержании колебаний в локализованных турбулентных структурах в трубах // принята к публикации в Изв. РАН. МЖГ. 2018. № 1;
\end{itemize}
3 статьи в сборниках трудов конференций:
\begin{itemize}
\item В.О. Пиманов, Пространственно-локализованные турбулентные структуры в круглой трубе // Труды конференции-конкурса молодых ученых 13-16 октября 2014 г. готовится к публикации;
\item В.О. Пиманов, О механизме самоподдержания локализованных турбулентных структур в трубах // Труды конференции-конкурса молодых ученых 12-14 октября 2015 г под редакцией академика РАН А.Г. Куликовского, профессора В.А. Самсонова. Издательство Московского университета. Москва. 2016. С. 44-51; 
\item В.О. Пиманов, Некоторые детали механизма самоподдержания турбулентности в пристенных течениях // Труды конференции-конкурса молодых ученых 10-12 октября 2016 г. С. 46-53. принят к печати;
\end{itemize}
18 тезисов научных конференций, а также статья в сборнике <<Суперкомпьютерные технологии в науке, образовании и промышленности>>:
\begin{itemize}
\item В.О. Пиманов, Суперкомпьютерное моделирование --- путь к пониманию турбулентности // Суперкомпьютерные технологии в науке, образовании и промышленности. том 7. Издательство Московского университета. Москва. 2017. С.  163-170. 
\end{itemize}


Также работа отмечена наградами:
\begin{itemize}
\item Диплом второй степени конференции-конкурса молодых ученых НИИ механики МГУ 2014 года;
\item Лучший доклад подсекции <<Гидродинамика>> секции <<Механика>> на конференции <<Ломоносов-2015>>;
\item Диплом первой степени за лучшую работу аспиранта конференции-конкурса молодых ученых НИИ механики МГУ 2015 года;
\item Диплом третьей степени конференции-конкурса молодых ученых НИИ механики МГУ 2015 года;
\item Вторая премия конкурса молодых научных сотрудников МГУ~им.~М.В.~Ломоносова 2015 года;
\item Диплом первой степени за лучшую работу аспиранта конференции-конкурса молодых учёных НИИ механики МГУ 2016 года;
\item Диплом третьей степени Конференции-конкурса молодых учёных НИИ механики МГУ 2016 года;
\item Лучший доклад подсекции <<Гидродинамика>> секции <<Механика>> конференции <<Ломоносов-2017>>.
\end{itemize}


{\bf Личный вклад автора.}

Научному руководителю, Н.В. Никитину, принадлежит идея исследования, реализация численного метода для прямого интегрирования уравнений движения жидкости, а также руководство работой и ценные советы в процессе её выполнения. Диссертанту принадлежат численные эксперименты и интерпретация полученных результатов, а также реализация метода поиска решения на сепаратрисе и метода Ньютона-Крылова для нахождения условно периодических по времени решений. Основная работа по подготовке текста диссертации и иллюстративного материала также принадлежит диссертанту.


{\bf Структура и объем диссертации.}

{\bf Благодарности.}


