
\chapter{Результаты}

\section{Модельный поорыв} 

Как известно, течение Пуазейля в круглой трубе устойчиво к малым возмущениям. Это значит, что для выхода на турбулентное решение, стартуя с возмущенного течения Пуазейля, амплитуда возмущения должна быть достаточно велика. Для начального возмущения фиксированной формы можно определить такое значение амплитуды, что решение в процессе эволюции будет оставаться на сепаратрисе, разделяющей в фазовом пространстве области притяжения ламинарного и турбулентного решений. Это решение неустойчиво и при численном интегрировании рано или поздно сваливается в ту или иную сторону --- либо выходит на турбулентный режим, либо возвращается к течению Пуазейля. Тем не менее, варьируя начальную амплитуду возмущения, можно проследить поведение балансирующего на сепаратрисе решения на значительном промежутке времени. Как показано в [12], предельное решение, к которому стремится решение на сепаратрисе, сохраняет некоторые черты турбулентного решения, но при этом как правило обладает более простым поведением во времени. В [13] обнаружено, что при наложении дополнительных ограничений симметрии предельное решение на сепаратрисе при $Re\sim2000$ качественно близко к турбулентному порыву, но оказывается при этом условно-периодическим по времени, а именно, периодическим в подходящей подвижной системе отсчета. Простота поведения предельного решения на сепаратрисе позволяет провести его исчерпывающее исследование, что на наш взгляд может быть небесполезным для понимания  механизма образования и самоподдержания турбулентных порывов.


\subsection{Метод получаения модельного порыва}

Следуя [13], решение уравнений Навье--Стокса ищется с дополнительными ограничениями диаметральной симметричности и $\pi$-периодичности поля скорости $\u=(u_x,u_r,u_\theta)$ в угловом направлении:
\begin{equation}
  (u_x,u_r)(t,x,r,-\theta)=(u_x,u_r)(t,x,r,\theta),\ \ u_\theta(t,x,r,-\theta)=-u_\theta(t,x,r,\theta)
 \label{sym}
\end{equation}
\begin{equation}
  (u_x,u_r,u_\theta)(t,x,r,\theta+\pi)\equiv(u_x,u_r,u_\theta)(t,x,r,\theta)
 \label{per}
\end{equation}

Здесь $(x,r,\theta)$ --- цилиндрические координаты. Наложение ограничений (\ref{sym}),(\ref{per}) упрощает поведение решения в пространстве, делает его более определенным. Турбулентные порывы, рассчитанные при $Re=2000$ с учетом и без учета условий (\ref{sym}),(\ref{per}) изображены на фиг.~2 (представлены области пониженной и повышенной на $0.1$ скорости относительно течения Пуазейля). В обоих случаях порыв имеет центральное ядро с пониженной скоростью и систему вытянутых вдоль стенки трубы, чередующихся в угловом направлении полос замедления и ускорения. На симметричном порыве полосы гораздо более структурированы. Их угловое положение не меняется в процессе эволюции: угловые области $\theta=k\pi/2,\ k=0-3$, где в силу (\ref{sym}),(\ref{per}) угловая компонента скорости тождественно равна нулю, заняты полосами ускорения, промежуточные области $\theta=\pi/4+k\pi/2$ --- полосами замедления. На порыве без условий симметрии наблюдаются значительные по амплитуде случайные по пространственному расположению флуктуации, разрывающие сплошность полосчатых структур. На симметричном порыве тоже заметна флуктуирующая компонента, которая в этом случае выглядит гораздо более регулярной. Отметим, что несмотря на заметную пространственную регулярность, временн\'{о}е поведение симметричного порыва остается хаотичным.

Предельное решение на сепаратрисе рассчитывалось при $Re=2200$. С учетом условий (\ref{sym}),(\ref{per}) расчет проводился для четверти объема трубы $0\leqslant\theta\leqslant\pi/2$. Длина расчетной области составляла $L_x=120$ при пространственном разрешении $1024\times40\times32$.

Предварительно найденное турбулентное решение $\u_{puff}(t,\x)$ используется в итерационной процедуре отыскания предельного решения на сепаратрисе. Задача решается с начальным условием
\begin{equation*}
  \u(t=0,\x)=\u_{Pois}(\x)+\alpha(\u_{puff}(t=t_0,\x)-\u_{Pois}(\x))
 \label{init}
\end{equation*}
Здесь $\u_{Pois}=(1-r^2,0,0)$ --- течение Пуазейля, $t_0$ --- некоторый фиксированный момент времени, $\alpha\in[0,1]$ --- скалярный параметр. Значение $\alpha=0$ соответствует нулевому возмущению, и решением при $t>0$ остается течение Пуазейля. Выбирая $\alpha=1$, мы уже в начальный момент времени попадаем на турбулентный режим и остаемся на нем при $t>0$. При промежуточных значениях $\alpha$ происходит стремление решения либо к одному, либо к другому режиму. Применяя метод деления пополам, мы постепенно отыскиваем то значение $\alpha$, при котором решение эволюционирует на сепаратрисе, разделяющей области притяжения двух режимов течения. На фиг.~3 представлены графики $A(t)$ --- среднеквадратичного по всему объему отклонения поля скорости от течения Пуазейля для нескольких значений $\alpha$, демонстрирующие сходимость итерационного процесса. Уточняя значение $\alpha$, мы продлеваем длительность балансирования решения на сепаратрисе.

В согласии с результатами [13], решение на сепаратрисе при $Re=2200$ постепенно выходит на условно периодический режим. Это решение, как и турбулентный порыв, имеет форму локализованой в пространстве структуры, которая сносится вниз по потоку с постоянной скоростью. В подвижной системе отсчета поле скорости в каждой точке испытывает периодические колебания. Для скорости сноса и периода колебаний получены значения $C_f=0.69$ и $T=60$ (в [13] сообщается о $C_f=0.76$ и $T=60$). Сравнение предельного решения на сепаратрисе с турбулентным порывом, представленное на фиг.~2, показывает качественное согласие этих решений. Во всех структурах имеются протяженные области ускоренного и замедленного движения, концентрирующиеся в пристенной области трубы. Сохраняется и основная качественная особенность порыва --- медленное понижение осевой скорости на переднем фронте и более резкое восстановление на заднем.


\subsection{Свойства предельного решения на сепаратрисе.}

Для удобства перейдем в подвижную систему координат, перемещающуюся вдоль трубы со скоростью сноса локализованной структуры $C_f$. В подвижной системе решение представляется в виде суперпозиции стационарной составляющей $\u_s(\x)$ и колебательной $\u_n(t,\x)$. Стационарную составляющую в свою очередь представим в виде суперпозиции осесимметричной $\u_{2D}(\x)$ и трехмерной $\u_{3D}(\x)=\u_s-\u_{2D}$ составляющих. Распределения продольной компоненты осесимметричной составляющей скорости вдоль трубы $u_{x,2D}(x)$ для нескольких расстояний от оси трубы представлены на фиг.~4,а (даны отклонения от течения Пуазейля). Начало системы отсчета $x=0$ помещено в сечение, в котором среднее отклонение скорости от течения Пуазейля максимально. Голова структуры, где начинает проявляться отклонение осевой скорости, располагается на расстоянии $x\approx45$. Хвостовая часть структуры на сепаратрисе очерчена не так четко, как в турбулентных порывах, где восстановление скорости происходит на отрезке длиной в $3-5$ радиусов трубы.  Падение скорости в приосевой области трубы компенсируется ускорением у стенки. Поведение радиальной компоненты $u_{r,2D}$, показанное на фиг.~4,б соответствует изменению осевой скорости --- в зоне замедления на оси происходит растекание жидкости к стенкам, $u_{r,2D}>0$, в передней части происходит обратный процесс и $u_{r,2D}<0$.

На фиг.~5 приведены распределения по $x$ среднеквадратичных по сечению трубы амплитуд трех составляющих движения: стационарной осесимметричной (отклонение от течения Пуазейля) $A_{2D}$, стационарной трехмерной $A_{3D}$ и колебательной $A_n$. Распределение $A_{2D}(x)$ соответствует фиг.~4. Отклонение от течения Пуазейля заметно на значительном отрезке от $x=-30$ до $x=40$. Максимум $A_{2D}$ составляет  8.4\%. Величина $A_{3D}$ характеризует интенсивность полосчатых структур. Как видно на фиг.~2 полосчатые структуры появляются на некотором расстоянии вверх по потоку от головы порыва и сохраняются на значительном расстоянии позади него. В согласии с этим $A_{3D}(x)$ имеет выраженную асимметрию относительно точки $x=-2$, где эта величина достигает максимума. Интенсивность полос быстро падает вниз по потоку и сохраняется на значительном расстоянии в верхней части потока. В отличие от стационарных полосчатых структур, колебательная составляющая движения сосредоточена на сравнительно непротяженном отрезке трубы от $x=-5$ до $x=15$ с максимальной амплитудой в 4\% при $x=2.5$.

\subsection{Механизм самоподдержания модельного порыва} 

Все описанные составляющие движения находятся в динамическом взаимодействии друг с другом. Как видно на рис.~5 наиболее локализованной вдоль трубы оказывается колебательная составляющая. Распределения среднеквадратичной амплитуды колебаний в нескольких сечениях трубы приведены на фиг.~6. Мгновенные колебания в соответствии с условием (\ref{per}) распределены по углу с периодом $\pi$, при этом колебания в точках $(x,r,\theta)$ и $(x,r,\pi/2\pm\theta)$ совпадают со сдвигом на полпериода по времени. Этим объясняется угловая $\pi/2$-периодичность амплитуды колебаний. Доминирующая мода колебательной составляющей пропорциональна $\exp(2\pi it/T)$ во времени и $\exp(2i\theta)$ в угловом направлении. Нелинейное взаимодействие колебательных мод порождает колебания на высших частотах, а также дает вклад в стационарную составляющую движения. В стационарной составляющей кроме осесимметричной части доминирует мода, пропорциональная $\exp(4i\theta)$, то есть с периодом $\pi/2$ в угловом направлении. Именно такой периодичности по углу соответствуют четыре пары полосчатых структур, наблюдающихся при решении задачи с условиями (\ref{sym}),(\ref{per}).

Отметим, что непосредственный вклад колебаний в образование полос не велик. Основной механизм роста полос это так называемый лифтап (lift-up) эффект, связанный с появлением движения в перпендикулярной к основному потоку плоскости. Частицы жидкости, перемещающиеся от стенки в сторону оси трубы, приносят дефект скорости и образуют полосу замедления, а частицы двигающиеся в противоположном направлении --- от оси к стенке, образуют полосу ускорения. Основная роль колебательной составляющей в этом механизме состоит именно в порождении стационарного движения в поперечной плоскости, распределение среднеквадратичной амплитуды которого $A_\bot(x)$ также представлено на фиг.~5. Как видно, область сосредоточения поперечного движения практически совпадает с областью существования колебаний. Некоторое уклонение $A_\bot(x)$ в заднюю сторону объясняется конвективным переносом этого движения (поперечное движение в основном возникает в периферийной части сечения трубы, где скорость потока в выбранной системе отсчета отрицательна).

Стационарное поперечное движение направлено от оси трубы к стенке в областях $\theta=k\pi/2$ и наоборот, от стенки к оси в промежуточных областях $\theta=\pi/4+k\pi/2$. Соответственно, в первых возникают полосы ускорения, во вторых --- замедления. Распределения скорости полосчатых структур в нескольких сечениях трубы приведены на фиг.~7. В сечении $x=0$, где максимальна (среди сечений, представленных на фиг.~7) интенсивность поперечного движения, изображено также векторное поле поперечного движения, демонстрирующее лифтап механизм образования полосчатых структур. На всех сечениях фиг.~7 сплошной линией изображена линия нулевой скорости осесимметричной составляющей движения в подвижной системе отсчета. В приосевой области, ограниченной этой линией, скорость положительна, а в периферийной --- отрицательна. Как видно из рисунков, полосчатые структуры во всех сечениях кроме самого переднего из представленных ($x=5$) располагаются в области отрицательной относительной скорости. Осесимметричное движение с отрицательной скоростью переносит полосчатые структуры в заднюю часть порыва, где они формируют картину, похожую на вытянутые щупальса медузы (см. фиг.~2). При $x>5$ полосчатые структуры концентрируются в приосевой части трубы и конвектируются вперед положительной скоростью относительного движения, благодаря чему в передней части порыва $A_{3D}$ сохраняет заметную величину, несмотря на отсутствие поперечного движения.

Полосчатые структуры достигают максимальной своей амплитуды в области $x\in[-5,0]$, где создаются условия для возникновения колебаний. Наиболее вероятный механизм генерации колебаний --- механизм потери устойчивости стационарной составляющей течения. Для проверки этой гипотезы стационарное течение с полем скорости $\u_s$ было исследовано на устойчивость к малым возмущениям. Линеаризованные относительно возмущений уравнения с некоторыми случайными начальными условиями интегрировались по времени до выхода решения на режим экспоненциального изменения. Обнаружено, что действительно поле скорости $\u_s$ неустойчиво к малым возмущениям. Растущее возмущение $\sim\exp(\lambda+i\omega)t$ имеет коэффициент роста $\lambda=0.012$ и частоту $\omega=0.116$, близкую к частоте колебаний $2\pi/60=0.105$ в решении на сепаратрисе. Что еще более существенно, распределение амплитуды колебаний в растущем решении задачи линейной устойчивости оказывается близким к соответствующим распределениям колебательной составляющей решения на сепаратрисе. Таким образом, нет сомнений, что механизмом появления колебаний является линейная неустойчивость стационарной составляющей движения.

Отметим, что неустойчивость полосчатых структур является неотъемлемой составляющей всех сценариев самоподдержания турбулентности в пристенных течениях. При этом чаще всего предполагается, что неустойчивость возникает в пристенных областях полос замедленного движения, где в локальном профиле скорости $U(r)$ на фоне наибольшего градиента появляется точка перегиба --- источник неустойчивости в механизме типа Кельвина--Гельмгольца. В частности, именно такой механизм предлагается в качестве механизма возникновения колебаний в турбулентном порыве в [11]. В рассматриваемом нами решении на сепаратрисе это определенно не так. Как видно на фиг.~6 в сечении $x=0$, соответствующем максимальной скорости роста колебаний, амплитуда колебаний минимальна как раз в области полосы замедления ($\theta=\pi/4$). Наибольшие колебания развиваются наоборот вблизи полос ускорения, а если быть более точным, в промежуточных областях между полосами. В этих областях стационарная составляющая скорости течения претерпевает наибольшее изменение и имеет точки перегиба, но не как функция радиальной переменной, а как функция угла. Во всех сечениях фиг.~6 сплошными линиями изображены линии нулевой относительной скорости стационарной составляющей течения. Видно, что в сечении $x=0$, где происходит основной рост колебаний, в областях максимальной амплитуды колебаний наблюдается наиболее быстрое изменение скорости как функции угловой переменной.

Отметим также, что точки максимального роста колебаний находятся на линии $r\approx0.4$, что соответствует нулевой относительной скорости. По этой причине область порождения колебаний остается неподвижной относительно порыва. Интересно, что в этой же области ($x=0,\ r\approx0.4$) происходит смена знака радиальной компоненты осесимметричной составляющей скорости (см. фиг.~4,б). При $x<0$ радиальная скорость положительна, поэтому колебания, возникшие в задней части порыва, относятся в сторону стенки трубы, где относительная скорость отрицательна, и уносятся в хвостовую часть порыва. Наоборот, при $x>0$ радиальная скорость направлена к оси трубы, туда же, в область положительной скорости, сносятся и колебания, обнаруживающиеся в передней части порыва.



\section{Верхняя ветвь решения}
\subsection*{}
\subsection{Метод Ньютона-Крылова}
\subsection{Описание решения с верхней ветви}
\subsection{Механизм самоподдержания}



