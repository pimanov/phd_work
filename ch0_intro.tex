\newpage
\section*{Предисловие}

Турбулнетность -- важная, актуальная задача, но не решенная. 


\chapter*{Введение}

Общеизвестно, что течение жидкости в трубах происходит ламинарным образом при малых скоростях и турбулентным при больших. 
Переход наступает, когда безразмерный параметр, число Рейнольдса $Re=UR/\nu$ ($U$ --- максимальная скорость, $R$ --- радиус трубы, $\nu$ --- кинематическая вязкость жидкости) превышает значение около 2000. Этот факт был установлен в основополагающих экспериментах О.~Рейнольдса [1]. Уже Рейнольдсом было замечено, что турбулентность первоначально проявляется перемежающимся образом, когда участки возмущенного и спокойного движения следуют вдоль трубы друг за другом, практически не меняя своей протяженности. На тот момент причина пространственной локализации турбулентности установлена не была. Подробное экспериментальное исследование локализованных турбулентных структур в трубах было выполнено в [2]. Установлено, что в разных условиях могут возникать структуры заметно разных типов. Структуры первого типа появляются при сильной возмущенности потока на входе в трубу в диапазоне $2000<Re<2700$. Они были названы турбулентными порывами ("turbulent puffs"). Порывы сносятся вниз по потоку со скоростью, близкой к средней скорости течения в трубе, практически не изменяя своей протяженности. Для порыва характерны размытость переднего фронта, на котором скорость на оси трубы постепенно уменьшается от ламинарного значения на 30 -- 40\% и резкость заднего фронта, на котором происходит возвращение к ламинарному течению. В последующей работе [3] установлено, что при $Re<2100$ турбулентные порывы подвержены спонтанному исчезновению, а при $Re>2300$ возможно деление порыва на два следующих друг за другом. Введено понятие {\it равновесного порыва}, характеристики которого не меняются по мере его продвижения вдоль трубы. Согласно [3] это наблюдается при $2100\leqslant Re\leqslant2300$. Другой тип локализованных турбулентных структур --- турбулентные пробки ("turbulent slugs") появляются при б\'{о}льших числах Рейнольдса $ Re>3200$ только когда возмущенность потока на входе недостаточна для непосредственного возникновения турбулентности. Тогда возможен переход через турбулентные пробки --- локализованные образования, расширяющиеся по мере сноса вниз по течению. Продвигаясь по трубе, пробки нагоняют друг друга (передний фронт пробки перемещается быстрее заднего), сливаясь в конечном итоге в единую турбулентную область.

В последние годы было выполнено несколько подробных экспериментальных и численных исследований характеристик и свойств турбулентных порывов [4---10]. Установлено, что турбулентный порыв является нестабильным образованием склонным либо к исчезновению либо к делению. С каждой из двух конкурирующих тенденций может быть связано характерное время: среднее время жизни порыва до его исчезновения и среднее время до его разделения. Первое увеличивается с ростом $Re$, второе уменьшается. Согласно точке зрения, сформулированной в [10], значение $Re=Re^*=2040$, при котором происходит смена доминирования тенденций, является точкой статистического фазового перехода и может быть принята в качестве минимального критического числа Рейнольдса в круглой трубе. При $Re<Re^*$ турбулентный порыв скорее исчезнет, чем успеет разделиться, так что возникновение развитого турбулентного течения невозможно. Наоборот, при $Re>Re^*$ порыв скорее успеет разделиться прежде, чем исчезнет, что приводит к развитию незатухающего турбулентного движения.

Турбулентный порыв представляет собой интересный гидродинамический объект, который в некотором отношении можно рассматривать как структурную единицу турбулентности. Можно сформулировать ряд вопросов, касающихся поведения порыва. До конца не понятен механизм, обуславливающий пространственную локализацию и самоподдержание порыва, неясны причины, побуждающие его к делению или затуханию, неизвестны факторы, определяющие его протяженность и скорость перемещения вдоль трубы.

В последние годы акцент в изучении механизма самоподдержания турбулентности в пристенных течениях смещается от лабораторного эксперимента в сторону эксперимента вычислительного, основанного на численном решении уравнений Навье--Стокса. Турбулентные порывы впервые были рассчитаны в [4], где было показано, что пространственная локализация является внутренним свойством решений уравнений Навье--Стокса при переходных числах Рейнольдса, а не является следствием специальных начальных условий. Попытка объяснения механизма сапоподдержания турбулентного порыва была предпринята в [11]. В системе отсчета связанной с порывом, пульсации в осевой части трубы сносятся вниз по потоку. Их нелинейное взаимодействие порождает медленно меняющиеся полосчатые структуры, концентрирующиеся в пристенной области трубы, где относительная скорость течения отрицательная. Из-за этого полосчатые структуры отстают от порыва. В хвостовой части порыва в областях расположения полос замедления образуются интенсивные сдвиговые слои с точкой перегиба в профиле скорости, где в силу неустойчивости типа Кельвина--Гельмгольца порождаются мелкомасштабные пульсации, попадающие в приосевую область трубы и сносящиеся вниз по потоку. Так, согласно [11] выглядит цикл самопроизводства турбулентных пульсаций внутри порыва и цикл самоподдержания самой этой структуры.

Идеализированная схема, предложенная в [11], выглядит вполне правдоподобно, однако, на наш взгляд, сделанные выводы в должной мере не подкреплены фактическими данными. Реальная динамика порыва сложнее и неопределеннее. Ее изучение осложнено в первую очередь стохастичностью процесса, когда отдельные его фазы следуют друг за другом случайным образом. В этих условиях определенная ясность может быть получена из анализа более простых структур, аппроксимирующих порыв, недавно найденных в [12,13]. Это предельные решения, возникающие на сепаратрисе, разделяющей в фазовом пространстве области притяжения решений, соответствующих ламинарному и турбулентному режимам течения. Такие решения, наследуя ряд качественных характеристик турбулентного порыва, оказываются периодическими по времени в системе отсчета, перемещающейся вдоль трубы с постоянной скоростью. Простота поведения позволяет провести исчерпывающее исследование свойств таких условно периодических решений, которые, как мы полагаем, проясняют определенные детали поведения турбулентного порыва.



