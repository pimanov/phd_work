\chapter{Введение}

\section{Актуальность}



Изучение закономерностей движения жидкостей и газов в трубах имеет большое значение как с практической, так и теоретической точки зрения. Известно, что при небольших скоростях жидкость в трубах движется ламинарным образом. Такое движение хорошо организовано. Ламинарное течение, устанавливающееся на удалении от входа в трубу, имеет аналитическое представление и называется течением Пуазейля. Частицы жидкости в течении Пуазелйя двигаются по прямым линиям, параллельным оси трубы. Профиль скорости в течении Пуазейля имеет форму параболы. 

По мере увеличения скорости жидкости режим течения сменяется на турбулентный, характеризующийся беспорядочными пульсациями скорости, давления, и других характеристик движения. 


Турбулентность можно отнести к важнейшим из нерешенных задач классической физики и современной науки в целом. Турбулентное движение жидкостей и газов характеризуется резким нерегулярным изменением скорости, давления и других характеристик течения в пространстве и во времени. Турбулентные течения наблюдаются повсеместно, например, в таких технических устройствах, как двигатели внутреннего сгорания или реактивные двигатели, высокоскоростной транспорт, трубопроводные системы. Формирование климата Земли, эволюция звезд и галактик тоже являются сугубо турбулентными процессами. Уравнения, описывающие движение жидкости, известны уже более 150 лет, однако современный математический аппарат не позволяет найти ни их решения, соответствующие турбулентному движению, ни более общие интегральные характеристики таких решений. В частности, величина трения между жидкостью и твердой стенкой может быть получена только из обобщения экспериментальных данных, но не из уравнений движения. Принципиально новые возможности исследования турбулентности открыло развитие вычислительной техники. Численный эксперимент позволяет получить полную информацию о течении --- параметры движения в каждой точке пространства в каждый момент времени. Тем не менее, сегодня нет понимания, каким именно образом движется жидкость в турбулентном потоке, на какие особенности следует обращать внимание при его описании, каким образом можно влиять на это движение. 


Одной из наиболее простых и в тоже время физически реализуемых постановок задачи, в которой наблюдается турбулентность, является движение жидкости в прямой трубе круглого сечения. Исследованию такого течения посвящена пионерская работа Осборна Рейнольдса, в которой в частности был обнаружен безразмерный параметр, определяющий характер течения в трубе, называемый числом Рейнольдса. 

Характер течения определяется единственным безразмерным параметром, числом Рейнольдса (Re), оно пропорционально средней скорости течения. Легко себе представить режим, при котором каждая частица жидкости движется по прямой вдоль трубы. Такой режим наблюдается при небольших числах Рейнольдса и называется ламинарным. По мере увеличения числа Рейнольдса в некоторый момент происходит переход к турбулентности, при этом многократно возрастает трение, которое необходимо преодолеть при прокачивании жидкости через трубу, увеличивается скорость перемешивания и скорость теплопереноса в жидкости, что важно, например, в задачах теплоотвода. Отдельный интерес представляет механизм самоподдержания турбулентности, не позволяющий вернуться к ламинарному течению.При больших числах Рейнольдса турбулентность заполняет всю трубу, в то время как при переходных значениях числа Рейнольдса ламинарный и турбулентный режимы течения сосуществуют. Формируются турбулентные порывы -- локализованные турбулентные структуры, отделенные друг от друга ламинарным потоком [1] (см. Рис. 1). Турбулентные порывы сносятся вниз по потоку примерно со средней скоростью течения , сохраняя при этом свою форму и пространственную протяженность. В длину порыв составляет несколько десятков диаметров трубы. Порыв является удобным объектом для исследования, так как в локальной форме в нем заключен механизм самоподдержания турбулентности. Более того, есть основания полагать, что сплошная турбулентность может быть представлена, как совокупность конкурирующих друг с другом порывов.

Изучению порывов посвящено большое количество работ, но до сих пор не известны как механизмы, приводящие к пространственной локализации, так и причины, определяющие скорость сноса порыва. Исследование осложнено присутствием беспорядочных пульсаций, на фоне которых теряется внутренняя структура турбулентности. Приблизиться к пониманию порыва позволяет анализ найденного в 2013 году [2] уникального решения уравнений Навье-Стокса. В фазовом пространстве это решение принадлежит сепаратрисе -- границе, отделяющей области притяжения ламинарного и турбулентного режимов. Это решение оказывается локализованным в пространстве и ухватывает основные особенности турбулентного порыва, но имеет более простую форму и, что важно, является периодическим по времени в системе отсчета, движущейся вместе с порывом (см. Рис. 2). Такое решение неустойчиво и не может быть воспроизведено в эксперименте, но численно его можно найти. Периодичность по времени позволяет провести его детальное исследование. 

В своей работе [3] мы численно воспроизвели как турбулентный порыв, так и решение на сепаратрисе. Расчеты проводились с использованием ресурсов суперкомпьютерного комплекса МГУ. При помощи метода Ньютона для решения нелинейных систем уравнений удалось найти и другие периодические решения, продлевая решение на сепаратрисе в пространстве параметров. На рис. 3 изображена зависимость скорости сноса и периода по времени от числа Рейнольдса. Решение на сепаратрисе оказывается удобным объектом для демонстрации механизма самоподдержания турбулентности. В предельно простом виде его можно сформулировать следующим образом. Решение на сепаратрисе содержит полосы повышенной и пониженной скорости, вытянутые вдоль стенки трубы вверх по потоку (см. Рис. 2). Эти полосы оказываются неустойчивыми, и на них рождается волна, бегущая вниз по порыву с увеличением амплитуды. Через нелинейное взаимодействие волна поддерживает существование полос. Этот механизм, вероятно, является более общим, так как подобные полосы можно выделить не только в турбулентном порыве, но и во многих других пристенных течениях.


\section{Цели работы}

Цель настоящей работы --- изучение локализованных турбулентных структур, возникающих в трубах круглого сечения, в частности, выявление их механизма самоподдержания. 

\section{Метод исследования и достоверность результатов}

В работе исследуются закономерности турбулентного движения жидкости в круглой трубе и в некоторых других геометриях. Течение жидкости воспроизводится численно путем решения уравнений Навье-Стокса для несжимаемой жидкости. Численный метод совмещает конечно-разностную аппроксимацию по пространственным переменным и полу-неявный метод Рунге-Кутты третьего порядки для интегрирования по времени. 

Возможность адекватного воспроизведение характеристик турбулентного течения в расчетах неоднократно продемонстрирована в большом числе работ. В частности, сравнение результатов расчета турбулентного поля скорости при $\Re = ???$ в круглой трубе с результатами эксперимента представлена здесь. В работе приведены результаты численного расчета турублентных порывов, изучению которых посвящена данная работа. То, что турбулентные порывы могут быть воспроизведены в идеализированной постановке, исключающей неровности стенки или неоднородность потока на входе, доказывает, что свойство локализации является внутренним свойством турбуленности и уранений  Навье-Стокса. 


В последние годы акцент в изучении механизма самоподдержания турбулентности в пристенных течениях смещается от лабораторного эксперимента в сторону эксперимента вычислительного, основанного на численном решении уравнений Навье--Стокса. Турбулентные порывы впервые были рассчитаны в [4], где было показано, что пространственная локализация является внутренним свойством решений уравнений Навье--Стокса при переходных числах Рейнольдса, а не является следствием специальных начальных условий. 

\section{Обзор литературы}

	\subsection{Переход к турбулентности в круглых трубах}

В качестве наиболее простого и в тоже время содержательного с практической точки зрения случая, в котором наблюдается турбулентность, можно выделить движение жидкости в прямых трубах круглого сечения, когда жидкость заполняет все пространство внутри трубы. Известно, что турбулентный режим течения в трубах устанавливается только если скорость потока достаточно велика, при малых скоростях реализуется ламинарное течение. Классическим считается результат Осборна Рейнольдса, опубликованный в 1883 году \cite{Reynolds1883}, согласно которому характер течения жидкости определяется безразмерной комбинацией параметров, называемой числом Рейнольдса. Если число Рейнольдса $\Re = RU/\nu$, вычисленное по максимальной скорости $U$, радиусу трубы $R$, и кинематической вязкости $\nu$, ниже критического значения, близкого к $2000$, то реализуется ламинарный режим течения. При больших $\Re$, как правило, течение оказывается турбулентным. 

Стоит отметить, что в лабораторных условиях, снижая уровень возмущений в потоке и организуя плавный вход жидкости в трубу, можно сохранить течения ламинарным при числах Рейнольдса $\Re \sim 10^4$ и больших \cite{Wygnanski1973, Darbyshire1995, vanDoorne2009}, значительно превышающих критическое значение. Это связано с тем, что турбулентность в трубах возникает жестким образом, без потери ламинарным течением устойчивости к малым возмущениям. Переход к турбулентности вызывают возмущения некоторой достаточно большой амплитуды \cite{Grossmann2000}, присутствующие в потоке. Следуя \cite{Darbyshire1995, Hof2003, Peixinho2007, Mellibovsky2009critical}, пороговое значение амплитуды возмущения, способного вызвать переход к турбулентности, асимптотически уменьшается по мере увеличения числа Рейнольдса по закону $\Re^{-\alpha}$, где $\alpha$ принимает значения от 1 до 2 в зависимости от формы возмущения. Таким образом, с увеличением $\Re$ сохранить течение ламинарным становится все сложнее. 

То обстоятельство, что турбулентность в трубах может существовать несмотря на линейную устойчивость ламинарной формы течения, позволяет говорить о механизме самоподдержания турбулентности, поддерживающем в потоке определенный уровень пульсаций. В случае отсутствия такого механизма амплитуда турбулентных пульсаций будет снижаться за счет вязкости. Когда её величина окажется ниже некоторого критического значения, в потоке установится ламинарный режим течения в силу его линейной устойчивости. На практике наблюдается обратная картина, при достаточно больших значениях $\Re$ турбулентность однажды возникнув продолжает свое существование неограниченно долго и вернуть поток в ламинарное состояние не представляется возможным. 

Обзоры, посвященные ламинарно-турбулентному переходу в трубах, приведены в работах \cite{Kerswell2005, Manneville2016, Kreilos2014, Barkley2016}. 


	\subsection{Локализованные турбулентные структуры}

Осборн Рейнольдс также отметил \cite{Reynolds1883}, что турбулентность в трубах первоначально проявляется перемежающимся образом, когда участки возмущенного и спокойного движения следуют вдоль трубы друг за другом, практически не меняя своей протяженности. На тот момент причина пространственной локализации турбулентности установлена не была. Позднее был выполнен ряд подробных экспериментальных исследований \cite{Lindgren1969, Wygnanski1973, Wygnanski1975, Bandyopadhyay1986, Darbyshire1995, vanDoorne2009}. 
Было установлено, что в разных условиях могут возникать структуры заметно разных типов.

Структуры первого типа --- {\it турбулентные порывы} (turbulent puffs) --- появляются при сильной возмущенности потока на входе в трубу в диапазоне $2000<\Re<2700$. Порывы сносятся вниз по потоку со скоростью, близкой к средней скорости течения, практически не меняя своей протяженности. Длина турбулентного порыва составляет несколько десятков диаметров трубы. Ни в какой другой форме, кроме как в форме порывов, турбулентность в указанном диапазоне числе Рейнольдса в течении длительного времени существовать не может \cite{vanDoorne2009, Moxey2010, Samanta2011}. Даже если возмущения вносятся в поток непрерывно или рассматривается эволюция полностью турбулентного поля скорости, полученного при больших значениях $\Re$, при переходных значениях $\Re$ формируются отдельные области фиксированной длины, разделенные ламинарным течением --- турбулентные порывы. Форма порыва не зависит от начального возмущения, но оно может влиять на их количество и расположение вдоль трубы. Если порывов несколько, они могут быть расположены в трубе нерегулярно, однако расстояние между ними не может быть ниже некоторого минимально значения \cite{Samanta2011}, что отражает тенденцию турбулентности к локализации. В работе \cite{Wygnanski1975} установлено, что при $\Re<2100$ турбулентные порывы подвержены спонтанному исчезновению, а при $\Re>2300$ возможно деление порыва на два следующих друг за другом. Введено понятие {\it равновесного порыва}, характеристики которого не меняются по мере его продвижения вдоль трубы. Согласно \cite{Wygnanski1975} это наблюдается при $2100\leqslant \Re \leqslant 2300$. 

Другой тип локализованных турбулентных структур --- {\it турбулентные пробки} ("turbulent slugs") --- появляются при б\'{о}льших числах Рейнольдса $\Re>3200$ только когда возмущенность потока на входе недостаточна для непосредственного возникновения турбулентности. Турбулентные пробки, двигаясь вниз по трубе, увеличивают свою протяженность, вовлекая в турбулентное движение окружающую жидкость на переднем и заднем фронтах. По мере того, как соседние локализованные структуры нагоняют друг друга, сливаясь вместе, происходит переход к сплошной турбулентности. Непосредственный интерес представляет скорость распространения турбулентной фракции в потоке. Описанию скорости перемещения и структуры фронтов турбулентной пробки посвящены работы \cite{Lindgren1969, Wygnanski1973, Nishi2008, Duguet2010, Barkley2015}. Было установлено, в частности, что турбулентность, по крайней мере до $\Re = 10^5$, не распространяется вверх по течению, то есть скорость заднего фронта положительна в системе отсчета, связанной с трубой, \cite{Wygnanski1973}. 

В работах \cite{Moxey2010, Barkley2015, Song2017} было выполнено подробное исследование динамики фронтом, возникающих на границе областей, занятых ламинарным и турбулентным движением. Было показано, что задний фронт турбулентной пробки качественно не отличим от заднего фронта турбулентного порыва. Он имеет ярко выраженную форму, скорость жидкости вблизи оси трубы падает на нем скачком. По мере увеличения $\Re$, при переходе от динамики турбулентного порыва к турбулентной пробке, скорость заднего фронта плавно снижается. В тоже время, передний фронт претерпевает ряд качественных изменений. При $\Re < 2250$, когда турбулентность представлена турбулентным порывами, скорость переднего фронта совпадает со скорость заднего. При больших $\Re$ передний фронт приобретает собственную скорость, которая возрастает по мере увеличения $\Re$. При $2600 < \Re < 3200$ происходит изменение формы переднего фронта. При меньших значениях $\Re$ он размыт. При б\'{о}льших $\Re$ передний фронт приобретает ярко выраженную форму, повторяющую форму заднего фронта. Стоит отметить, что, хотя уже при $\Re > 2250$ средняя скорость переднего фронта превышает среднюю скорость заднего, и области, в которых наблюдается турбулентное течение, расширяются, в потоке не формируется протяженных турбулентных структур вплоть до $\Re = 3000$. Можно заключить, что турбулентные порывы и турбулентные пробки являются формами одного и того же. 

В работах \cite{Barkley2015, Barkley2016} была предложена модель течения в трубах, воспроизводящая динамику локализованных турбулентных структур. Модель оперирует двумя параметрами, наполненностью среднего профиля скорости и уровнем турбулентных пульсаций, как функциями продольной скоординаты и времени. 

В последние годы было выполнено несколько подробных экспериментальных и численных исследований характеристик и свойств турбулентных порывов [4---10]. Установлено, что турбулентный порыв является нестабильным образованием склонным либо к исчезновению либо к делению. С каждой из двух конкурирующих тенденций может быть связано характерное время: среднее время жизни порыва до его исчезновения и среднее время до его разделения. Первое увеличивается с ростом $Re$, второе уменьшается. Согласно точке зрения, сформулированной в [10], значение $Re=Re^*=2040$, при котором происходит смена доминирования тенденций, является точкой статистического фазового перехода и может быть принята в качестве минимального критического числа Рейнольдса в круглой трубе. При $Re<Re^*$ турбулентный порыв скорее исчезнет, чем успеет разделиться, так что возникновение развитого турбулентного течения невозможно. Наоборот, при $Re>Re^*$ порыв скорее успеет разделиться прежде, чем исчезнет, что приводит к развитию незатухающего турбулентного движения.

Турбулентный порыв представляет собой интересный гидродинамический объект, который в некотором отношении можно рассматривать как структурную единицу турбулентности. Можно сформулировать ряд вопросов, касающихся поведения порыва. До конца не понятен механизм, обуславливающий пространственную локализацию и самоподдержание порыва, неясны причины, побуждающие его к делению или затуханию, неизвестны факторы, определяющие его протяженность и скорость перемещения вдоль трубы.

Большинство работ носят описательный характер, но можно выделить несколько, нацеленных на объяснение формы и механизма самоподдержания турбулентного порыва \cite{Duguet2010, Hof2010, Shimizu2009}. В работе \cite{Duguet2010} предложен механизм, объясняющий расширение локализованных турбулентных структур, но они не касаются механизмов их самоподдержания. В работе \cite{Hof2010} механизм самоподдержания турбулентного порыва связывают с точкой перегиба, расположенной на заднем фронте порыва, если рассматривать поле скорости, как функцию радиальной координаты. В непосредственой близости с точкой перегиба в турбулентном порыва резко падает скорость на заднем фронте, и наблюдается резкий скачет кинетической энергии турбулнетных пульсаций. 
Попытка объяснения механизма сапоподдержания турбулентного порыва была предпринята в \cite{Shimizu2009}. В системе отсчета связанной с порывом, пульсации в осевой части трубы сносятся вниз по потоку. Их нелинейное взаимодействие порождает медленно меняющиеся полосчатые структуры, концентрирующиеся в пристенной области трубы, где относительная скорость течения отрицательная. Из-за этого полосчатые структуры отстают от порыва. В хвостовой части порыва в областях расположения полос замедления образуются интенсивные сдвиговые слои с точкой перегиба в профиле скорости, где в силу неустойчивости типа Кельвина--Гельмгольца порождаются мелкомасштабные пульсации, попадающие в приосевую область трубы и сносящиеся вниз по потоку. Так, согласно \cite{Shimizu2009} выглядит цикл самопроизводства турбулентных пульсаций внутри порыва и цикл самоподдержания самой этой структуры.


	\subsection{Пристенные турбулентные структуры} \label{structure_subsection}


Механизм самоподдержания, описанный в \cite{Shimizu2009}, согласуется с представления о механизме самоподдреания сдвиговых течений в целом. 
\begin{comment}
In the vicinity of a wall, turbulent flow is often found to be highly organized as there exist regions, called coherent structures, where the fluid motion is more strongly correlated than in full turbulence. In figure 1.2.1(a) a flow visualization of a turbulent boundary layer is reproduced from Kline et al. (1967), showing alternating regions of low- and high-speed fluid, elongated in the streamwise direction. The spanwise spacing between these so-called streaks is best expressed in wall units, which are based on the wall shear stress τ W = μ∂u/∂y| y=0 q , with μ = νρ the dynamic viscosity + of the fluid. A length scale is defined by l = ν/ τ W /ρ and the spacing is universally observed to be ∼ 100l + (Klebanoff et al., 1962; Kline et al., 1967). Further investigations (see e.g. Blackwelder and Eckelmann, 1979) reveal the existence of downstream vortices, occuring in counter-rotating pairs. Through linear advection, these vortices push fluid from the high-speed free stream towards the wall, thus creating and sustaining a high-speed streak. Vice versa, fluid is pulled from the wall into the free stream, resulting in a low-speed streak; this process of streak 51. Introduction
creation through downstream vortices is termed lift-up effect. Vortices in near-wall turbulence are typically not exactly parallel to the downstream direction but aligend at a small angle. They are furthermore slightly inclined from the wall (Jeong et al., 1997; Schoppa and Hussain, 2002).
In open flows where no boundaries limit the flow in the wall-normal direction, horseshoe or hairpin shaped vortices are commonly observed (Head and Bandyopadhyay, 1981; Robinson, 1991; Adrian et al., 2000; Adrian, 2007). Close to the wall, these structures are similar to a counter-rotating vortex pair, but they are inclined and extend further into the free stream, where they are connected, forming a hairpin shaped structure. The concept goes back more than half a century to Theodorsen (1952), an illustration is reproduced in figure 1.2.1(b). To better understand the dynamical properties of these typical flow structures, simulations are often performed in so-called minimal flow units defined by the fact that turbulence cannot be sustained if any of the dimensions is reduced (Jiménez and Moin, 1991), with some freedom in the definition of sustained turbulence and the ratio of the dimensions. The concept is useful because it allows to extract features of turbulence which are otherwise obfuscated by spatial processes and because a small periodic domain is cheaper in numerical simulations. In such a minimal flow unit, a self-sustaining cycle of near-wall turbulence has been identified (Hamilton et al., 1995; Waleffe, 1997, 2003), connecting the typical flow structures and their instabilities. The cycle starts with a pair of counter-rotating streamwise-aligned vortices, which create a pair of streaks through the lift-up effect. The streaks, initially straight, are linearly unstable to developing a wavy modulation, and the period in which the streaks are created is about as long as the subsequent period during which the instabilities grow. As the modulation becomes too strong, the structures break up, leaving behind a flow with strong downstream variations. The cycle is closed by nonlinear interactions recreating the downstream vortices. This so called self-sustaining process is found in many shear flows and was, for example, observed experimentally by Duriez et al. (2009). It also served as the stimulus for making an analogy between fluid mechanics and magnetohydrodynamic dynamos (Riols et al., 2013). 
\end{comment}

	\subsection{Сдвиговые течения}

This is common for wall-bounded shear flows (Manneville 2015) --- жесткое возбуждение турбулентности



	\subsection{Решение на сепаратрисе}


https://hal.archives-ouvertes.fr/file/index/docid/1021118/filename/S0022112010003435a.pdf

Можно отметить, что течение Пуазейля, соответствующее ламинарному течению, устанавливающемуся на удалении от входа в трубу, устойчиво к малым возмущениям при всех $\Re$, как принято сегодня считать \cite{Kerswell2005}. Линейная устойчивость течения Пуазейля была показана численно до $\Re=10^7$ \cite{Meseguer2003}, а также аналитически для осесимметричных возмущений \cite{Salwen1980}. 

Идеализированная схема, предложенная в [11], выглядит вполне правдоподобно, однако, на наш взгляд, сделанные выводы в должной мере не подкреплены фактическими данными. Реальная динамика порыва сложнее и неопределеннее. Ее изучение осложнено в первую очередь стохастичностью процесса, когда отдельные его фазы следуют друг за другом случайным образом. В этих условиях определенная ясность может быть получена из анализа более простых структур, аппроксимирующих порыв, недавно найденных в [12,13]. Это предельные решения, возникающие на сепаратрисе, разделяющей в фазовом пространстве области притяжения решений, соответствующих ламинарному и турбулентному режимам течения. Такие решения, наследуя ряд качественных характеристик турбулентного порыва, оказываются периодическими по времени в системе отсчета, перемещающейся вдоль трубы с постоянной скоростью. Простота поведения позволяет провести исчерпывающее исследование свойств таких условно периодических решений, которые, как мы полагаем, проясняют определенные детали поведения турбулентного порыва.

\section{Апробация работы}

\section{Благодарности}


