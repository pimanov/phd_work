\chapter{Введение}

\subsection{Актуальность темы} 

Изучение закономерностей движения жидкостей и газов в трубах имеет большое значение как с практической, так и с теоретической точки зрения. Известно, что при небольших скоростях течения жидкость движется ламинарным образом. Такое движение хорошо организовано --- жидкие частицы могут быть объединены в слои, смещающиеся друг относительно друга без перемешивания \cite{Hof2010}. Если скорость жидкости достаточно велика, ламинарный режим течения сменяется турбулентным, характеризующемуся наличием беспорядочных пульсаций скорости, давления, и других величин. Кроме того, при переходе к турбулентности ряд характеристик потока, таких как среднее трение на стенке или средний профиль скорости, качественно меняются. Как было установлено Осборном Рейнольдсом в работе 1883 года \cite{Reynolds1883}, характер течение определяет безразмерная комбинация параметров, называемая числом Рейнольдса. Если число Рейнольдса $Re=UR/\nu$, вычисленное по максимальной скорости течения $U$, радиусу трубы $R$ и кинематической вязкости жидкости $\nu$, ниже критического значения, близкого к $2000$, жидкость движется ламинарным образом. При больших $\Re$, как правило, движение оказывается турбулентным.

Уже Рейнольдсом было замечено, что турбулентность первоначально проявляется перемежающимся образом, когда участки возмущенного и спокойного движения следуют вдоль трубы друг за другом, практически не меняя своей протяженности. На тот момент причина пространственной локализации турбулентности установлена не была. Подробное экспериментальное исследование локализованных турбулентных структур в трубах было выполнено в \cite{Wygnanski1973}. Установлено, что в разных условиях могут возникать структуры заметно разных типов. Структуры первого типа --- турбулентные порывы ({\it turbulent puffs}) --- появляются при сильной возмущенности потока на входе в трубу в диапазоне $2000<\Re<2700$. Порывы сносятся вниз по потоку со скоростью, близкой к средней скорости течения в трубе, практически не изменяя своей протяженности. Для порыва характерны размытость переднего фронта, на котором скорость на оси трубы постепенно уменьшается от ламинарного значения на 30 -- 40\% и резкость заднего фронта, на котором происходит возвращение к ламинарному течению. В последующей работе \cite{Wygnanski1975} установлено, что при $\Re<2100$ турбулентные порывы подвержены спонтанному исчезновению, а при $\Re>2300$ возможно деление порыва на два следующих друг за другом. Введено понятие {\it равновесного порыва}, характеристики которого не меняются по мере его продвижения вдоль трубы. Согласно \cite{Wygnanski1975} это наблюдается при $2100\leqslant\Re\leqslant2300$. 

Локализованные турбулентные структуры другого типа --- турбулентные пробки ({\it turbulent slugs}) появляются при б\'{о}льших числах Рейнольдса $\Re>3200$ только когда возмущенность потока на входе недостаточна для непосредственного возникновения турбулентности. Тогда возможен переход через турбулентные пробки --- локализованные образования, расширяющиеся по мере сноса вниз по течению. Продвигаясь по трубе, пробки нагоняют друг друга (передний фронт пробки перемещается быстрее заднего), сливаясь в конечном итоге в единую турбулентную область.

В последние годы выполнен ряд подробных экспериментальных и численных исследований характеристик и свойств турбулентных порывов \cite{Priymak2004, Peixinho2006, Hof2006finite, Willis2007, Hof2008, Kuik2010, Avila2011}. Установлено, что турбулентный порыв является нестабильным образованием, склонным либо к исчезновению, либо к делению. С каждой из двух конкурирующих тенденций связано характерное время: среднее время жизни порыва до его исчезновения и среднее время до его разделения. Первое увеличивается с ростом $\Re$, второе уменьшается. Согласно точке зрения \cite{Avila2011}, значение $\Re=\Re^*=2040$, при котором происходит смена доминирования тенденций, является точкой статистического фазового перехода и может быть принята в качестве минимального критического числа Рейнольдса в круглой трубе. При $\Re<\Re^*$ турбулентный порыв скорее погибнет, чем успеет разделиться, так что возникновение развитого турбулентного течения невозможно. Наоборот, при $\Re>\Re^*$ порыв скорее успеет произвести потомство прежде, чем погибнет, что приводит к развитию незатухающего турбулентного движения.

Турбулентный порыв представляет собой интересный гидродинамический объект, который в некотором отношении может рассматриваться как структурная единица турбулентности. Можно сформулировать ряд вопросов, касающихся поведения порыва. До конца не понятен механизм, обуславливающий пространственную локализацию и самоподдержание порыва, неясны причины, побуждающие его к делению или затуханию, неизвестны факторы, определяющие его протяженность и скорость перемещения вдоль трубы.

В последние годы акцент в изучении механизма самоподдержания турбулентности в пристенных течениях смещается от лабораторного эксперимента в сторону эксперимента вычислительного, основанного на численном решении уравнений Навье--Стокса. Турбулентные порывы впервые были рассчитаны в \cite{Priymak2004}, где было показано, что пространственная локализация является внутренним свойством решений уравнений Навье--Стокса при переходных числах Рейнольдса, а не является следствием специальных начальных условий. Попытка объяснения механизма сапоподдержания турбулентного порыва была предпринята в \cite{Shimizu2009}. В системе отсчета связанной с порывом, пульсации в осевой части трубы сносятся вниз по потоку. Их нелинейное взаимодействие порождает медленно меняющиеся полосчатые структуры, концентрирующиеся в пристенной области трубы, где относительная скорость течения отрицательна. Из-за этого полосчатые структуры отстают от порыва. В хвостовой части порыва в областях расположения полос замедления образуются интенсивные сдвиговые слои с точкой перегиба в профиле скорости, где в силу неустойчивости типа Кельвина--Гельмгольца порождаются мелкомасштабные пульсации, попадающие в приосевую область трубы и сносящиеся вниз по потоку. Так, согласно \cite{Shimizu2009}, выглядит цикл самопроизводства турбулентных пульсаций внутри порыва и цикл самоподдержания самой этой структуры.

Идеализированная схема, предложенная в \cite{Shimizu2009}, выглядит вполне правдоподобно, однако, на наш взгляд, сделанные выводы в должной мере не подкреплены фактическими данными. Реальная динамика порыва сложнее и неопределеннее. Ее изучение осложнено в первую очередь стохастичностью процесса, когда отдельные его фазы следуют друг за другом случайным образом. В этих условиях определенная ясность может быть получена из анализа более простых структур, аппроксимирующих порыв, недавно найденных в \cite{Skufca2006, Avila2013}. Это предельные решения, возникающие на сепаратрисе, разделяющей в фазовом пространстве области притяжения решений, соответствующих ламинарному и турбулентному режимам течения. Такие решения, наследуя ряд качественных характеристик турбулентного порыва, оказываются периодическими по времени в системе отсчета, перемещающейся вдоль трубы с постоянной скоростью. Мы будем называть такие структуры {\it модельными порывами}. Простота поведения позволяет провести исчерпывающее исследование свойств модельного порыва, которые по мнению автора проясняют определенные детали поведения турбулентного порыва. 

Выделенные в турбулентном порыве полосы повышенной и пониженной скорости, связанные с циклом его самоподдержания, являются характерной особенностью многих пристенных турбулентных течений, включая течение в пограничном слое \cite{Klebanoff1962, Kline1967}. С пристенными полосами связывают цикл самоподдержания пристенной турбулентности \cite{Waleffe1997, Hamilton1995, Schoppa2002}. Полосы повышенной и пониженной скорости могут быть выделены также и в модельном порыве. Этот факт позволяет рассчитывать, что сделанные при исследовании модельного порыва выводы будут применимы не только к турбулентному порыву, но и к более широкому классу пристенных турбулентных течений. 


\section{Цель работы}

Работа направлена на определение закономерностей турбулентного движения жидкости, обеспечивающих существование турбулентного порыва; выявление причин, определяющих форму порыва и его основные свойства. Исследуя турбулентный порыв непосредственно, определенных выводов сделать не удается в силу нерегулярности его поведения. Мы полагаем, что приблизиться к его пониманию позволит изучение модельного порыва \cite{Avila2013}. Модельный порыв воспроизводит ряд особенностей турбулентного порыва, но имеет более простую форму и динамику. В подходящей подвижной системе отсчета он оказывается периодическим по времени, что позволяет выполнить его детальное исследование и строго обосновать полученные результаты.

Целью работы является воспроизведение модельного порыва, описание его свойств и внутренней структуры, их сравнение с приведенными в литературе данным о турбулентным порыве; выявление причин, определяющих форму модельного порыва и его свойства, выделение цикла его самоподдержания. 

С целью установления общности полученных при изучении модельного порыва результатов, в работе предполагается поиск дополнительных инвариантных решений поставленной задачи. В частности, известно, что, опираясь на модельный порыв, можно получить новые периодические по времени решения при отличающихся значениях параметров \cite{Viswanath2007, Dijkstra2014}. Одной из задач диссертационной работы является поиск и изучение таких периодических решений с параметрами, приближающимися к параметрам турбулентного течения. Можно рассчитывать, что закономерности, выделенные при исследовании нескольких решений --- модельного порыва и найденных таким образом новых периодических решений, будут иметь более общий характер. 

Помимо локализованных в пространстве решений поставленной задачи известен также широкий класс решений типа бегущей волны --- периодических вдоль потока, стационарных в некоторой подвижной системе отсчета. Такие решения были найдены во многих сдвиговых течения, в том числе и в круглой трубе \cite{Kawahara2012}. Хотя существуют некоторые сомнения относительно применимости полученных при их исследовании результатов к турбулентным течениям, на них также могут быть проверены некоторые закономерности, выделенные в работе. Основной особенностью бегущих волн, несвойственной турбулентным течениям, является корреляция скорости в любых двух точка, удаленных друг от друга на расстояние, кратное периоду волны, как бы велико оно не было. В цели диссертационной работы входит также поиск бегущих волн и проверка сделанных в работе выводов на такого рода решениях. 


\section{Метод исследования и достоверность результатов}

В данной работе движение жидкости воспроизводится и исследуется численно, путем решения полных трехмерных уравнений Навье-Стокса для несжимаемой жидкости. Постановка задачи приведена в разделе \ref{math_section}. Возможность адекватного воспроизведения результатов физического эксперимента в такого рода расчетах неоднократно была продемонстрирована, начиная с работы \cite{Kim1987}. Сравнение результатов численного моделирования с экспериментальными данными для развитого турбулентного течения в круглых трубах можно найти в работах \cite{Priymak1998, Nikitin2006}, для равновесного турбулентного порыва --- в работе \cite{Priymak2004}. В настоящее время при изучении турбулентных течений численное моделирование применяется на ровне с физическим экспериментом. Расчеты позволяют получить исчерпывающее представление о турбулентном поле скорости и полях других величин, что существенно при изучении закономерностей движения жидкости в турбулентных потоках. 

Численный метод, применяемый в работе, совмещает конечно-разносную аппроксимацию второго порядка точности по пространственным переменным и метод полу-неявный Рунге-Кутты интегрирования по времени \cite{Nikitin2006, Nikitin2006third}. Метод сохраняет ряд консервативных свойств исходной системы уравнений, что особенно важно при моделировании турбулентных течений. Описание метода приведено в разделе \ref{num_method}. Метод используется в лаборатории Общей аэродинамики института механики МГУ уже более 20 лет и хорошо себя зарекомендовал. Программный код, реализующий метод, написан Никитиным Н.В. и отлажен в процессе решения большого числа задач. Подтверждают качество кода и адекватность численного метода результаты моделирования турбулентного течения в трубе при переходных числах Рейнольдса, приведенные в разделе \ref{puff_calc}. Турбулентность в расчетах действительно принимает форму локализованных структур, характеристики которых совпадают с представленными в литературе данными. 

Численные моделирование позволяет не только воспроизводить экспериментальные данные, но и получать режимы течения, недостижимые в эксперименте. Примером может быть решение, принадлежащее сепаратрисе, отделяющей в фазвом пространстве областипритяжения ламинарного и турбулентного режимов течения. 


\section{Обзор литературы}

	\subsection{Переход к турбулентности в круглых трубах}

В качестве наиболее простого и в тоже время содержательного с практической точки зрения случая, в котором наблюдается турбулентность, можно выделить движение жидкости в прямых трубах круглого сечения, когда жидкость заполняет все пространство внутри трубы. Известно, что турбулентный режим течения в трубах устанавливается только если скорость потока достаточно велика, при малых скоростях реализуется ламинарное течение. Классическим считается результат Осборна Рейнольдса, опубликованный в 1883 году \cite{Reynolds1883}, согласно которому характер течения жидкости определяется безразмерной комбинацией параметров, называемой числом Рейнольдса. Если число Рейнольдса $\Re = RU/\nu$, вычисленное по максимальной скорости $U$, радиусу трубы $R$, и кинематической вязкости $\nu$, ниже критического значения, близкого к $2000$, то реализуется ламинарный режим течения. При больших $\Re$, как правило, течение оказывается турбулентным. 

Стоит отметить, что в лабораторных условиях, снижая уровень возмущений в потоке и организуя плавный вход жидкости в трубу, можно сохранить течения ламинарным при числах Рейнольдса $\Re \sim 10^4$ и больших \cite{Wygnanski1973, Darbyshire1995, vanDoorne2009}, значительно превышающих критическое значение. Это связано с тем, что турбулентность в трубах возникает жестким образом, без потери ламинарным течением устойчивости к малым возмущениям. Переход к турбулентности вызывают возмущения некоторой достаточно большой амплитуды \cite{Grossmann2000}, присутствующие в потоке. Следуя \cite{Darbyshire1995, Hof2003, Peixinho2007, Mellibovsky2009critical}, пороговое значение амплитуды возмущения, способного вызвать переход к турбулентности, асимптотически уменьшается по мере увеличения числа Рейнольдса по закону $\Re^{-\alpha}$, где $\alpha$ принимает значения от 1 до 2 в зависимости от формы возмущения. Таким образом, с увеличением $\Re$ сохранить течение ламинарным становится сложнее. 

То обстоятельство, что турбулентность в трубах может существовать несмотря на линейную устойчивость ламинарной формы течения, позволяет говорить о механизме самоподдержания турбулентности, поддерживающем в потоке существование турбулентных пульсаций. В случае отсутствия такого механизма амплитуда турбулентных пульсаций будет снижаться за счет вязкости. Когда её величина окажется ниже некоторого критического значения, в потоке установится ламинарный режим течения в силу его линейной устойчивости. На практике наблюдается обратная картина, при достаточно больших значениях $\Re$ турбулентность однажды возникнув продолжает свое существование неограниченно долго и вернуть поток в ламинарное состояние не представляется возможным. 

Обзоры, посвященные ламинарно-турбулентному переходу в трубах, приведены в работах \cite{Kerswell2005, Manneville2016, Kreilos2014, Barkley2016}. 


	\subsection{Локализованные турбулентные структуры в трубах}

Осборн Рейнольдс также отметил \cite{Reynolds1883}, что турбулентность в трубах первоначально проявляется перемежающимся образом, когда участки возмущенного и спокойного движения следуют вдоль трубы друг за другом, практически не меняя своей протяженности. На тот момент причина пространственной локализации турбулентности установлена не была. Позднее было установлено \cite{Lindgren1969, Wygnanski1973, Wygnanski1975, Bandyopadhyay1986, Darbyshire1995, vanDoorne2009}, что в разных условиях могут возникать структуры заметно разных типов.

Структуры первого типа --- {\it турбулентные порывы} (turbulent puffs) --- появляются при сильной возмущенности потока на входе в трубу в диапазоне $2000<\Re<2700$ \cite{Wygnanski1973}. Порывы сносятся вниз по потоку со скоростью, близкой к средней скорости течения, практически не меняя своей протяженности. Длина турбулентного порыва составляет несколько десятков диаметров трубы. Ни в какой другой форме, кроме как в форме порывов, турбулентность в указанном диапазоне числе Рейнольдса в течении длительного времени существовать не может \cite{vanDoorne2009, Moxey2010, Samanta2011}. Даже если возмущения вносятся в поток непрерывно или рассматривается эволюция полностью турбулентного поля скорости, полученного при больших значениях $\Re$, при переходных значениях $\Re$ формируются отдельные области фиксированной длины, разделенные ламинарным течением --- турбулентные порывы. Форма порыва не зависит от начального возмущения, но оно может влиять на их количество и расположение вдоль трубы. Если порывов несколько, они могут быть расположены в трубе нерегулярно, однако расстояние между ними не может быть ниже некоторого минимально значения \cite{Samanta2011}, что отражает тенденцию турбулентности к локализации. В работе \cite{Wygnanski1975} установлено, что при $\Re<2100$ турбулентные порывы подвержены спонтанному исчезновению, а при $\Re>2300$ возможно деление порыва на два следующих друг за другом. Введено понятие {\it равновесного порыва}, характеристики которого не меняются по мере его продвижения вдоль трубы. Согласно \cite{Wygnanski1975} это наблюдается при $2100\leqslant \Re \leqslant 2300$. 

Другой тип локализованных турбулентных структур --- {\it турбулентные пробки} ("turbulent slugs") --- появляются при б\'{о}льших числах Рейнольдса $\Re>3200$ только когда возмущенность потока на входе недостаточна для непосредственного возникновения турбулентности \cite{Wygnanski1973}. Турбулентные пробки, двигаясь вниз по трубе, увеличивают свою протяженность, вовлекая в турбулентное движение окружающую жидкость на переднем и заднем фронтах. По мере того, как соседние локализованные структуры нагоняют друг друга, сливаясь вместе, происходит переход к сплошной турбулентности. Непосредственный интерес представляет скорость распространения турбулентной фракции в потоке. Описанию скорости перемещения и структуры фронтов турбулентной пробки посвящены работы \cite{Lindgren1969, Wygnanski1973, Nishi2008, Duguet2010, Barkley2015}. Было установлено, в частности, что турбулентность, по крайней мере до $\Re = 10^5$, не распространяется вверх по течению, то есть скорость заднего фронта положительна в системе отсчета, связанной с трубой, \cite{Wygnanski1973}. 

В работах \cite{Moxey2010, Barkley2015, Song2017} было выполнено подробное исследование динамики фронтов, возникающих на границе областей, занятых ламинарным и турбулентным движением. Было показано, что задний фронт турбулентной пробки качественно не отличим от заднего фронта турбулентного порыва. Он имеет ярко выраженную форму, скорость жидкости вблизи оси трубы падает скачком. По мере увеличения $\Re$, при переходе от динамики турбулентного порыва к турбулентной пробке, скорость заднего фронта плавно снижается. В тоже время, передний фронт претерпевает ряд качественных изменений. При $\Re < 2250$, когда турбулентность представлена турбулентным порывами, скорость переднего фронта совпадает со скорость заднего. При больших $\Re$ передний фронт приобретает собственную скорость, которая возрастает по мере увеличения $\Re$. При $2600 < \Re < 3200$ происходит изменение формы переднего фронта. При меньших значениях $\Re$ он размыт. При б\'{о}льших $\Re$ передний фронт приобретает ярко выраженную форму, схожую с формой заднего фронта. Стоит отметить, что, хотя уже при $\Re > 2250$ средняя скорость переднего фронта превышает среднюю скорость заднего, и области, в которых наблюдается турбулентное течение, расширяются, в потоке не формируется протяженных турбулентных структур вплоть до $\Re = 3000$. Можно заключить, что турбулентные порывы и турбулентные пробки являются формами одного и того же. 

В последние годы было выполнено несколько подробных экспериментальных и численных исследований характеристик и свойств турбулентных порывов [4---10]. Установлено, что турбулентный порыв является нестабильным образованием склонным либо к исчезновению либо к делению. С каждой из двух конкурирующих тенденций может быть связано характерное время: среднее время жизни порыва до его исчезновения и среднее время до его разделения. Первое увеличивается с ростом $Re$, второе уменьшается. Согласно точке зрения, сформулированной в [10], значение $Re=Re^*=2040$, при котором происходит смена доминирования тенденций, является точкой статистического фазового перехода и может быть принята в качестве минимального критического числа Рейнольдса в круглой трубе. При $Re<Re^*$ турбулентный порыв скорее исчезнет, чем успеет разделиться, так что возникновение развитого турбулентного течения невозможно. Наоборот, при $Re>Re^*$ порыв скорее успеет разделиться прежде, чем исчезнет, что приводит к развитию незатухающего турбулентного движения.


В работах \cite{Barkley2015, Barkley2016} была предложена феноменологическая модель течения в трубах, воспроизводящая характерные особенности ламинарно-турбулентного перехода. 

ряд ключевых особенностей динамику локализованных турбулентных структур. Модель оперирует двумя параметрами, наполненностью среднего профиля скорости и уровнем турбулентных пульсаций, как функциями продольной скоординаты и времени. 


Турбулентный порыв представляет собой интересный гидродинамический объект, который в некотором отношении можно рассматривать как структурную единицу турбулентности. Можно сформулировать ряд вопросов, касающихся поведения порыва. До конца не понятен механизм, обуславливающий пространственную локализацию и самоподдержание порыва, неясны причины, побуждающие его к делению или затуханию, неизвестны факторы, определяющие его протяженность и скорость перемещения вдоль трубы.

Большинство работ носят описательный характер, но можно выделить несколько, нацеленных на объяснение формы и механизма самоподдержания турбулентного порыва \cite{Duguet2010, Hof2010, Shimizu2009}. В работе \cite{Duguet2010} предложен механизм, объясняющий расширение локализованных турбулентных структур, но они не касаются механизмов их самоподдержания. В работе \cite{Hof2010} механизм самоподдержания турбулентного порыва связывают с точкой перегиба, расположенной на заднем фронте порыва, если рассматривать поле скорости, как функцию радиальной координаты. В непосредственной близости с точкой перегиба в турбулентном порыва резко падает скорость на заднем фронте, и наблюдается резкий скачет кинетической энергии турбулнетных пульсаций. 
Попытка объяснения механизма сапоподдержания турбулентного порыва была предпринята в \cite{Shimizu2009}. В системе отсчета связанной с порывом, пульсации в осевой части трубы сносятся вниз по потоку. Их нелинейное взаимодействие порождает медленно меняющиеся полосчатые структуры, концентрирующиеся в пристенной области трубы, где относительная скорость течения отрицательная. Из-за этого полосчатые структуры отстают от порыва. В хвостовой части порыва в областях расположения полос замедления образуются интенсивные сдвиговые слои с точкой перегиба в профиле скорости, где в силу неустойчивости типа Кельвина--Гельмгольца порождаются мелкомасштабные пульсации, попадающие в приосевую область трубы и сносящиеся вниз по потоку. Так, согласно \cite{Shimizu2009} выглядит цикл самопроизводства турбулентных пульсаций внутри порыва и цикл самоподдержания самой этой структуры.


	\subsection{Параллели с другими сдвиговыми течениями}

Особенности перехода к турбулентности, наблюдаемые в круглых трубах, являются характерными для ряда сдвиговых течений \cite{Manneville2015, Manneville2016}, таких как течения в плоском канале Пуазейля и Куэтта, или в трубах прямоугольного сечения. Отчасти, аналогичные свойства демонстрирует пограничный слой на плоской пластине. При небольших значениях числа Рейнольдса (при обтекании плоской пластины локального числа Рейнольдса, вычисленного по расстоянию до передней кромки) жидкость движется ламинарным образом. При достаточно больших $\Re$ устанавливается турбулентный режим течения. Во многих сдвиговых течениях переход к турбулентности происходит жестким образом без потери ламинарным течением линейной устойчивости. Кроме того, при переходных значениях $\Re$ турбулентность проявляется перемежающимся образом --- области, занятые ламинарным и турбулентным течением сменяют друг друга в продольном направлении.  

В плоском канале жидкость заключена между двумя параллельными плоскими стенками. В плоском канале Пуазейля жидкость приводится в движение внешним перепадом давления, направленным вдоль потока; в плоском канале Куэтта --- смещением одной из стенок относительно второй с постоянной скоростью так, что расстояние между ними остается постоянным. В обоих случаях ламинарный поток, устанавливающийся с течением времени, может быть представлен в аналитическом виде и исследован на линейную устойчивость, в частности, в рамках уравнений Навье-Стокса. В плоском канале Пуазейля установившееся ламинарное течение, называемое течением Пуазейля, теряет линейную устойчивость при $\Re = 5772$ \cite{Orszag1971}, однако турбулентность в потоке наблюдается уже при $\Re = 1000$ \cite{Orszag1980}. Число Рейнольдса определяется по половине ширины канала и максимальной скорости потока. В плоском канале Куэтта турбулентность возникает при числах Рейнольдса, близких к $300$ \cite{Bottin1998}, однако установившееся ламинарное течение в этом случае устойчиво при всех значениях $\Re$ \cite{Romanov1973}. Здесь число Рейнольдса определяется по половине ширины канала и половине разности скоростей стенок. 

Как и в круглой трубе, в плоском канале Куэтта \cite{Prigent2002, Barkley2005} и Пуазейля турбулентность при переходных значениях $\Re$ принимает форму локализованных вдоль потока структур. Плавно снижая число Рейнольдса от значения, при котором наблюдается сплошная турбулентность, в канале Куэтта можно наблюдать формирование пространственной неоднородности потока \cite{Duguet2010Couette}. При $340 < \Re < 415$ турбулентность формирует косые полосы, проходящие под некоторым углом к основному потоку; при $325 < \Re < 340$ сплошные полосы распадаются на отдельные фрагменты и турбулентные пятна; при меньших $\Re$ продолжительное время турбулентность существовать не может. В плоском канале Пуазейля аналогичный эксперимент дает качественно неотличимые результаты. В интервале чисел Рейнольдса $800 < \Re < 1000$ турбулентность также существует в форме косых полос; при числах Рейнольдса, близких к 800, косые полосы распадаются на отдельные участки \cite{Tuckerman2014, Lernoult2014, Sano2015}, при меньших $\Re$ турбулентность перестает быть устойчивой. 

Известно, что ламинарный пограничный слой на плоской пластине теряет линейную устойчивость к волнам Толмина-Шлихтинга при $\Re \sim 520$ \cite{Schlichting2004}. Число Рейнольдса в этом случае определено по расстоянию от передней кромки пластины и скорости набегающего потока на бесконечности. Однако при достаточно высоком уровне возмущений в набегающем потоке турбулентность может возникнуть при значениях $\Re$ ниже критического. В этом случае внесенные в поток возмущения приводят к образованию турбулентных порывов \cite{Katasonov2014}. По мере перемещения вниз по потоку они увеличиваются в размере: их передний фронт перемещается со скоростью $0.9$ скорости набегающего потока в то время, как задний со скорость $0.5$. При этом их ширина и толщина практически не меняются. Порывы представлены модуляцией преимущественно продольной компоненты скорости в пристенном сдвиговом течении. В них могут быть выделены продольные полосы повышенной и пониженной скорости. В тот момент, когда сдвиговые слои, возникающие между полосами, теряют устойчивость, на месте порывов формируются турбулентные пятна. Такие пятна возникают на месте вол Толмина-Шлихтинга в случае мягкого сценария перехода к турбулентности. Сливаясь вместе пятна дают начало сплошной турбулентности. 


	\subsection{Пристенные турбулентные структуры} \label{structure_subsection}

Особенности турбулентного порыва, выделенные в \cite{Shimizu2009}, и связанный с ними механизм самоподдержания являются характерными для широкого класса пристенных турбулентных течений. Известно, что вблизи стенки в турбулентном течении существуют долгоживущие крупномасштабные структуры, способные к самоподдержанию. В первую очередь они представлены пристенными полосами ("streaks") --- вытянутыми вдоль потока областями, скорость жидкости внутри которых выше или ниже среднего значения на данном расстоянии от стенки \cite{Klebanoff1962}. Хотя полосы могу изгибаться и перемещаться вдоль стенки, пропадать и возникать вновь, они хорошо различимы на фоне беспорядочных турбулентных пульсаций \cite{Kline1967}. Геометрические характеристики полос в широком диапазоне параметров оказываются универсальными постоянными в пристенных единицах длины $l^+ = \nu / \sqrt{\tau_{w} / \rho}$, вычисленных по локальным характеристикам потока, а именно кинематической вязкости $\nu$, среднему трению на стенке $\tau_{w}$ и плотности жидкости $\rho$. Среднее расстояние между соседними полосами одного знака оценивают в $100 l^+$. Полосы достигают наибольшую интенсивность на расстоянии от $10 l^+$ до $20 l^+$ от стенки. Длину полосы можно оценить в $1000 l^+$. С пристенными полосами связывают возникновение турбулентных пульсаций, так как полосчатый профиль скорости может быть неустойчив. Между соседними полосами противоположного знака возникает сдвиговый слой, подверженный неустойчивости типа Кельвина-Гельмгольца \cite{}.

Формирование пристенных полос связывают с вытянутыми вдоль потока вихрями \cite{Blackwelder1979, Jeong1997}, перемещающими жидкость в нормальной к основному потоку плоскости. Там, где вихри переносят медленную жидкость от стенки трубы в основной поток, формируются полосы пониженной скорости. Там, где жидкость перемещается ближе к стенке, возникают полосы повышенной скорости. Описанный механизм называют лифт-ап эффектом ("lift-up effect"). Считается, что продольные вихри возникают в следствии линейной неустойчивости полос, в результате нелинейного взаимодействия возникающих на них пульсаций \cite{Hamilton1995, Schoppa2002, Kawahara2003}. Таким образом, с пристенными когерентными структурами может быть связан цикл самоподдержания \cite{Hamilton1995, Waleffe1997}. Описание существующих представлений о нелинейном механизме образования продольных вихрей можно найти в разделе дискуссия в конце главы, посвященной исследованию модельному порыву. Проводится их сравнение с выделенным в работе механизмом.   

В открытых течениях, где в нормальном к стенке направлении поток можно считать неограниченным, формируются вихри, имеющие форму подковы или шпильки для волос \cite{Head1981, Robinson1991, Adrian2000, Adrian2007}. Пара вихрей, имеющая противоположное направление вращения, расположенных по бокам от полосы замедления, смыкаются ниже по течению на некотором удалении от стенки. 

Характерной особенностью пристенных турбулентных течений является логарифмический профиль средней скорости (зависимость продольной скорости от расстояния до стенки). В пристенном масштабе профиль скорости имеет универсальную форму. Возможно, формирование логарифмического профиля скорости связан с существование когерентных структур, так как полосы попадают в буферный слой, в котором происходит переход от ламинарного подслоя к логарифмическому слою \cite{}. 


	\subsection{Инвариантны решения уравнений Навье-Стокса}



	\subsection{Решение на сепаратрисе}

Аналогично, ламинарное течение в круглой трубе, устанавливающееся на удалении от входа в трубу, так же называемое течением Пуазейля, устойчиво при всех числах Рейнольдса, как принято считать \cite{Kerswell2005}. Течение Пуазейля при всех $\Re$ устойчиво к осесимметричным возмущениям \cite{Salwen1980}. Также, численно было показано, что течение Пуазейля устойчиво к произвольным возмущениям до $\Re = 10^7$ \cite{Meseguer2003}. 

https://hal.archives-ouvertes.fr/file/index/docid/1021118/filename/S0022112010003435a.pdf

Можно отметить, что течение Пуазейля, соответствующее ламинарному течению, устанавливающемуся на удалении от входа в трубу, устойчиво к малым возмущениям при всех $\Re$, как принято сегодня считать \cite{Kerswell2005}. Линейная устойчивость течения Пуазейля была показана численно до $\Re=10^7$ \cite{Meseguer2003}, а также аналитически для осесимметричных возмущений \cite{Salwen1980}. 

Идеализированная схема, предложенная в [11], выглядит вполне правдоподобно, однако, на наш взгляд, сделанные выводы в должной мере не подкреплены фактическими данными. Реальная динамика порыва сложнее и неопределеннее. Ее изучение осложнено в первую очередь стохастичностью процесса, когда отдельные его фазы следуют друг за другом случайным образом. В этих условиях определенная ясность может быть получена из анализа более простых структур, аппроксимирующих порыв, недавно найденных в [12,13]. Это предельные решения, возникающие на сепаратрисе, разделяющей в фазовом пространстве области притяжения решений, соответствующих ламинарному и турбулентному режимам течения. Такие решения, наследуя ряд качественных характеристик турбулентного порыва, оказываются периодическими по времени в системе отсчета, перемещающейся вдоль трубы с постоянной скоростью. Простота поведения позволяет провести исчерпывающее исследование свойств таких условно периодических решений, которые, как мы полагаем, проясняют определенные детали поведения турбулентного порыва.

\section{Апробация работы}

\section{Благодарности}


