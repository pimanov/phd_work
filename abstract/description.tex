\section*{\centering  Общая характеристика работы}
\addcontentsline{toc}{chapter}{Общая характеристика работы}

\textbf{Актуальность темы.}
Изучение закономерностей движения жидкостей и газов в трубах имеет большое значение как с практической, так и с теоретической точки зрения. Известно, что при небольших скоростях течения жидкость в трубах движется ламинарным образом. Такое движение хорошо организовано --- жидкие частицы могут быть объединены в слои, смещающиеся друг относительно друга без перемешивания. При достаточно больших скоростях течения, ламинарный режим сменяется турбулентным, характеризующимся наличием беспорядочных пульсаций скорости, давления, и других характеристик. Кроме того, при переходе к турбулентности многие свойства потока, такие как величина трение на стенке или форма среднего профиля скорости, качественно меняются. Как было установлено Осборном Рейнольдсом в работе 1883 года \cite{Reynolds1883}, характер течение определяет безразмерная комбинация параметров, называемая ныне числом Рейнольдса. Если число Рейнольдса $Re=UR/\nu$, вычисленное по максимальной скорости течения $U$, радиусу трубы $R$ и кинематической вязкости $\nu$, ниже критического значения, близкого к $2000$, жидкость движется ламинарным образом. При б\'{о}льших $\Re$ движение, как правило, оказывается турбулентным.

Уже Рейнольдсом было замечено, что турбулентность первоначально проявляется перемежающимся образом, когда участки возмущенного и спокойного движения следуют вдоль трубы друг за другом, практически не меняя своей протяженности. На тот момент причина пространственной локализации турбулентности установлена не была. Подробное экспериментальное исследование локализованных турбулентных структур в трубах было выполнено в \cite{Wygnanski1973}. Установлено, что в разных условиях могут возникать структуры заметно разных типов. Структуры первого типа --- турбулентные порывы ({\it turbulent puffs}) --- появляются при сильной возмущенности потока на входе в трубу в диапазоне $2000<\Re<2700$. Порывы сносятся вниз по потоку со скоростью, близкой к средней скорости течения, практически не меняя свою протяженность. Для порыва характерны размытость передней границы, на которой скорость на оси трубы плавно уменьшается от ламинарного значения на 30 -- 40\%, и резкость задней границы, на которой происходит возвращение к ламинарному течению. В последующей работе \cite{Wygnanski1975} установлено, что при $\Re<2100$ турбулентные порывы подвержены спонтанному исчезновению, а при $\Re>2300$ возможно деление порыва на два следующих друг за другом. Введено понятие {\it равновесного порыва}, характеристики которого не меняются по мере его продвижения вдоль трубы. Согласно \cite{Wygnanski1975} это наблюдается при $2100 < \Re < 2300$. 

Локализованные турбулентные структуры другого типа --- турбулентные пробки ({\it turbulent slugs}) появляются при б\'{о}льших числах Рейнольдса $\Re>3200$ только когда возмущенность потока на входе недостаточна для непосредственного возникновения турбулентности. Тогда возможен переход через турбулентные пробки --- локализованные образования, расширяющиеся по мере сноса вниз по течению. Продвигаясь по трубе, пробки нагоняют друг друга (передняя граница пробки перемещается быстрее задней), сливаясь в конечном итоге в единую турбулентную область.

В последние годы выполнен ряд подробных экспериментальных и численных исследований характеристик и свойств турбулентных порывов \cite{Priymak2004, Peixinho2006, Hof2006finite, Willis2007, Hof2008, Kuik2010, Avila2011}. Установлено, что турбулентный порыв является нестабильным образованием, склонным либо к исчезновению, либо к делению. С каждой из двух конкурирующих тенденций связано характерное время: среднее время жизни порыва до его исчезновения и среднее время до его разделения. Первое увеличивается с ростом $\Re$, второе уменьшается. Согласно точке зрения \cite{Avila2011}, значение $\Re^*=2040$, при котором происходит смена доминирования тенденций, является точкой статистического фазового перехода и может быть принята в качестве минимального критического числа Рейнольдса в круглой трубе. При $\Re<\Re^*$ турбулентный порыв скорее погибнет, чем успеет разделиться, так что возникновение развитого турбулентного течения невозможно. Наоборот, при $\Re>\Re^*$ порыв скорее успеет произвести потомство прежде, чем погибнет, что приводит к развитию незатухающего турбулентного движения.

Турбулентный порыв представляет собой интересный гидродинамический объект, который в некотором отношении может рассматриваться как структурная единица турбулентности. Можно сформулировать ряд вопросов, касающихся поведения порыва. До конца не понятен механизм, обуславливающий пространственную локализацию и самоподдержание порыва, неясны причины, побуждающие его к делению или затуханию, неизвестны факторы, определяющие его протяженность и скорость перемещения вдоль трубы.

В последние годы акцент в изучении механизма самоподдержания турбулентности в пристенных течениях смещается от лабораторного эксперимента в сторону эксперимента вычислительного, основанного на численном решении уравнений Навье--Стокса. Турбулентные порывы впервые были рассчитаны в \cite{Priymak2004}, где было показано, что пространственная локализация является внутренним свойством решений уравнений Навье--Стокса при переходных числах Рейнольдса, а не является следствием специальных начальных условий. Попытка объяснения механизма самоподдержания турбулентного порыва была предпринята в \cite{Shimizu2009}. В системе отсчета связанной с порывом, пульсации в осевой части трубы сносятся вниз по потоку. Их нелинейное взаимодействие порождает медленно меняющиеся полосчатые структуры, концентрирующиеся в пристенной области трубы, где относительная скорость течения отрицательна. Из-за этого полосчатые структуры отстают от порыва. В хвостовой части порыва в областях расположения полос замедления образуются интенсивные сдвиговые слои с точкой перегиба в профиле скорости, где в силу неустойчивости типа Кельвина--Гельмгольца порождаются мелкомасштабные пульсации, попадающие в приосевую область трубы и сносящиеся вниз по потоку. Так, согласно \cite{Shimizu2009}, выглядит цикл самопроизводства турбулентных пульсаций внутри порыва и цикл самоподдержания самой этой структуры.

Идеализированная схема, предложенная в \cite{Shimizu2009}, выглядит вполне правдоподобно, однако, на наш взгляд, сделанные выводы в должной мере не подкреплены фактическими данными. Реальная динамика порыва сложнее и неопределеннее. Ее изучение осложнено в первую очередь стохастичностью процесса, когда отдельные его фазы следуют друг за другом случайным образом. В этих условиях определенная ясность может быть получена из анализа более простых структур, аппроксимирующих порыв, недавно найденных в \cite{Skufca2006, Avila2013}. Это предельные решения, возникающие на сепаратрисе, разделяющей в фазовом пространстве области притяжения решений, соответствующих ламинарному и турбулентному режимам течения. Такие решения, наследуя ряд качественных характеристик турбулентного порыва, оказываются периодическими по времени в системе отсчета, перемещающейся вдоль трубы с постоянной скоростью. Мы будем называть такие структуры {\it модельными порывами}. Простота поведения позволяет провести исчерпывающее исследование свойств модельного порыва, которые, как мы полагаем, проясняют определенные детали поведения турбулентного порыва. 

Продолжение модельного порыва по числу Рейнольдса позволяет получить новые условно периодические решения уравнений Навье-Стокса (периодические в подходящей подвижной системе отчета), имеющие пространственно-локализованную структуру \cite{Avila2013}. Кроме того, в настоящее время известно достаточно большое число решений, имеющих вид трехмерной бегущей волны (периодичных вдоль потока и стационарных в подходящей подвижной системе отсчета) \cite{Kawahara2012}. Такие решения также допускаю детальное исследование и могут быть использованы для установления универсальности выделенных при исследовании модельного порыва закономерностей. 

Характерной особенностью модельного порыва является наличие вытянутых вдоль потока областей, скорость жидкости внутри которых существенно выше или ниже среднего значения --- полос повышенной и пониженной скорости. Такие полосы являются неотъемлемым элементом всех сценариев поддержания пристенной турбулентности \cite{Hamilton1995, Waleffe1997, Schoppa2002}, что дает основания полагать, что полученные при изучении модельного порыва результаты могут быть полезными для понимания динамики не только турбулентного порыва, но и более общего класса пристенных турбулентных течений. 

%Во введении к диссертации определяется 1) актуальность избранной темы, 2) степень ее разработанности, 3) цели и задачи, 4) научная новизна, 5) теоретическая и практическая значимость работы, 6) методология диссертационного исследования, 7) положения, выносимые на защиту, 8) степень достоверности и апробация результатов.

%В автореферате диссертации излагаются 1) положения, выносимые на защиту, 2) основные идеи и выводы диссертации, 3) показываются вклад автора в проведенное исследование, 4) степень новизны и практическая значимость результатов исследования, 4) содержатся сведения об организации, в которой выполнялась диссертация, об оппонентах, о научных руководителях и научных консультантах соискателя ученой степени (при наличии), 5) приводится список публикаций автора диссертации, в которых отражены основные научные результаты диссертации. 

{\bf Цель диссертационной работы} состоит в выявлении механизмов, ответственных за возникновение и поддержание турбулентных порывов. 
С этой целью поставлены и решены следующие задачи: 

\noindent $\bullet$  Проведено численное исследование модельного порыва --- условно периодического решения уравнений Навье-Стокса в геометрии течения в круглой трубе, имеющего пространственно-локализованную структуру. Воспроизводя ряд качественных особенностей турбулентного порыва, модельный порыв имеет более простое временное поведение, что позволяет выполнить его детальное исследование. 

\noindent $\bullet$  Методом продолжения по параметру рассчитаны другие условно периодические решения уравнений Навье-Стокса в геометрии течения в круглой трубе, также имеющие пространственно локализованную структуру. Их анализ позволил оценить универсальность найденных при исследовании модельного порыва закономерностей. 

\noindent $\bullet$  Рассчитаны и исследованы трехмерные решения уравнений Навье-Стокса в геометрии течения в круглой трубе и геометрии течения в плоском канале, имеющие вид бегущей волны. Анализ таких решений также позволил оценить универсальность найденных при изучении модельного порыва закономерностей. 

\textbf{Научная новизна}

\textbf{Теоретическая и практическая значимость}

\textbf{Методология и метод исследования}

\textbf{Положения, выносимые на защиту:}

\textbf{Степень достоверности и апробация результатов}

\textbf{Метод исследования и достоверность результатов.}
В работе движение жидкости воспроизводится и исследуется численно, путем решения полных трехмерных уравнений Навье-Стокса для несжимаемой жидкости. Возможность адекватного описания турбулентных порывов численными решениями уравнений Навье-Стокса впервые показана в \cite{Priymak2004}. Применяемый нами численный метод совмещает конечно-разностную аппроксимацию второго порядка точности по пространственным переменным и полунеявный метод Рунге-Кутты третьего порядка точности интегрирования по времени \cite{Nikitin2006, Nikitin2006third} (детали в разделе \ref{num_method}). Метод используется в лаборатории Общей аэродинамики института механики МГУ и других лабораториях в разных частях света уже более 20 лет и хорошо себя зарекомендовал. Код программы, реализующей метод, написан Н.В.\,Никитиным и отлажен в процессе решения большого числа задач. Расчеты выполнены с привлечением ресурсов суперкомпьютерного комплекса МГУ. Качество программной реализации численного метода и его адекватность целям работы подтверждают результаты моделирования турбулентного течения в трубе при переходных числах Рейнольдса, приведенные в разделе \ref{puff_calc}. Турбулентность в расчетах принимает форму порывов, характеристики которых совпадают с представленными в литературе данными. 

Основным объектом исследования в работе выступает модельный порыв. Решение, соответствующее модельному порыву, является предельным состоянием решения, эволюционирующего на сепаратрисе, разделяющей в фазовом пространстве области притяжения решений, соответствующих ламинарному и турбулентному режимам течения. Решение на сепаратрисе неустойчиво и не может быть получено в эксперименте, но может быть найдено численно. Процедура нахождения решения на сепаратрисе приведена в разделе \ref{edge_seq}. В согласии с \cite{Avila2013} решение на сепаратрисе выходит на условно периодический режим и имеет пространственно-локализованную структуру. Характеристики модельного порыва, полученные на нескольких расчетных сетках, согласуются друг с другом и с характеристиками, приведенными в работах \cite{Avila2013, Chantry2014}. Совпадение результатов подтверждает, что полученное численно решение аппроксимирует соответствующее ему решение уравнений Навье--Стокса и не зависит от выбора численного метода и параметров расчета. В \cite{Avila2013} применен полностью спектральный метод \cite{Meseguer2007} и спектрально-конечно-разностный метод \cite{Willis2009}. В нашей работе применен конечно-разностный метод~\cite{Nikitin2006}. В \cite{Chantry2014} модельный порыв воспроизведен спектрально-конечно-разностным методом \cite{Willis2009}, авторы также сообщают о совпадении результатов с \cite{Avila2013}. 

Оценить степень универсальности полученных при исследовании модельного порыва результатов позволяет анализ других решений уравнений Навье-Стокса, имеющих простое временное поведение. Можно надеяться, что общие для многих решений особенности движения могут иметь место в турбулентном течении. В работе реализован метод Ньютона-Крылова \cite{Viswanath2007, Dijkstra2014} для поиска условно периодических решений (см. раздел  \ref{Newton_seq}). Этот метод позволяет уточнять уже найденные решения и продолжать их по параметру. Продолжение модельного порыва по числу Рейнольдса позволило получить новые условно периодические решения уравнений Навье-Стокса, также, как и модельный порыв, имеющие пространственно-локализованную структуру (детали в разделе \ref{contin_sec}). Сравнение результатов, полученных на нескольких расчетных сетках, с результатами \cite{Avila2013} позволяет сделать выводы о соответствии найденных численных решений решениям уравнений Навье--Стокса. Также, применение метода поиска решения на сепаратрисе и метода Ньютона-Крылова позволило найти три семейства решений в виде бегущих волн (см. разделы \ref{pipeTW_seq}, \ref{ductTW_seq}). Одно семейство решений относится к течению в круглой трубе и два --- к течению в плоском канале. Основные закономерности, выделенные при изучении модельного порыва, оказываются справедливы для всех исследованных решений, что в некоторой степени подтверждает универсальность этих закономерностей. 

В работе получены следующие {\bf научные результаты}:
\begin{enumerate}
\item Рассчитан модельный порыв --- условно периодическое решение уравнений Навье-Стокса в геометрии течения в круглой трубе, имеющее пространственно-локализованную структуру. Описаны его основные характеристики и внутренняя структура. 
\item Методом продолжения по параметру рассчитаны другие условно периодические решения уравнений Навье-Стокса, также имеющие про\-странственно-локализованную структуру. Описаны их основные характеристики и внутренняя структура. 
\item Рассчитано три семейства решений уравнений Навье-Стокса в виде бегущих волн. Одно семейство решений найдено в геометрии течения в круглой трубе и два семейства --- в геометрии течения в плоского канала. Описаны их основные характеристики и внутренняя структура. 
\item В рассчитанных решениях определены основные элементы механизма поддержания колебаний. Во всех исследованных решениях за поддержание колебаний отвечает один механизм, что свидетельствует о его универсальности. 
Поле скорости решений может быть представлено в виде суммы средней и пульсационной составляющих. Характерной особенностью среднего течения являются вытянутые вдоль потока полосы повышенной и пониженной скорости. Пульсации генерируются в результате линейной неустойчивости среднего течения в промежуточной области между полосами на фоне резкого изменения скорости вдоль угловой (поперечной) координаты. Существование полос поддерживают продольные вихри, перемещающие жидкость в нормальной к основному потоку плоскости.
\item Обнаружен нелинейный механизм поддержания продольных вихрей, вызывающих полосчатое искажение в распределении продольной скорости. Существование продольных вихрей поддерживается нелинейным взаимодействием пульсаций продольной скорости и пульсаций продольной завихренности. Пульсации продольной завихренности образуются за счет сжатия и растяжения существующих в среднем течении вихревых трубок пульсациями продольной скорости, что обеспечивает необходимую для поддержания продольны вихрей согласованность фаз между этими пульсациями. 
\end{enumerate}

{\bf Теоретическая и практическая значимость полученных результатов.}
В работе достигнуто определенное понимание механизмов, ответственных за поддержание колебаний в рассмотренных модельных течениях. Полосчатые структуры являются непременным атрибутом всех сценариев поддержания колебаний в широком классе пристенных турбулентных течений. Это позволяет рассчитывать на возможность обобщения полученных в работе результатов на этот класс течений. Понимание механизмов поддержания колебаний имеет первостепенное значения для предсказания характеристик пристенных турбулентных течений и разработки эффективных методов управления такими течениями.

{\bf Положения, выносимые на защиту.} Анализ рассчитанных в работе течений позволяет предложить следующий идеализированный цикл поддержания колебаний в них:
\begin{itemize}
\item В каждом решении поле скорости может быть представлено в виде суммы средней и пульсационной составляющих. Существенной особенностью среднего течения является наличие вытянутых вдоль потока полос повышенной и пониженной скорости. Пульсации возникают в результате линейной неустойчивости среднего течения в областях, расположенных между соседними полосами повышенной и пониженной скорости. В этих областях распределение средней продольной скорости имеет точки перегиба, если рассматривать его как функцию угловой (поперечной) координаты, что позволяет связать образование колебаний с неустойчивостью струйного течения с точками перегиба. 
\item За поддержание полос повышенной и пониженной скорости ответственны продольные вихри, перемещающие жидкость в нормальной к основному потоку плоскости. Там, где медленная жидкость перемещается от стенки в основной поток, образуются полосы пониженной скорости. В промежуточных областях образуются полосы повышенной скорости. 
\item Механизм поддержания продольных вихрей состоит в нелинейном взаимодействии пульсаций продольной скорости и пульсаций продольной завихренности. При этом в области расположения продольных вихрей пульсации продольной завихренности образуются в результате сжатия и растяжения существующих в потоке вихревых трубок пульсациями продольной скорости, что обеспечивает необходимую для поддержания продольных вихрей согласованность фаз между этими пульсациями. 
\end{itemize}

{\bf Личный вклад автора.} 
Научному руководителю, Н.В. Никитину, принадлежит идея исследования, реализация численного метода для прямого интегрирования уравнений движения жидкости, а также руководство работой и ценные советы в процессе её выполнения. Автору диссертации принадлежат численные эксперименты и анализ полученных результатов, а также реализация метода поиска решения на сепаратрисе и метода Ньютона-Крылова для нахождения условно периодических решений уравнений Навье-Стокса. Основная работа по подготовке текста диссертации и иллюстративного материала также принадлежит диссертанту. 


{\bf Апробация работы и публикации.} 
Результаты диссертационной работы представлены на 24 конференциях: 
\begin{itemize}
\item Конференция-конкурс молодых ученых НИИ механики МГУ имени М.В. Ломоносова (2014, 2015, 2016, 2017 годы); 
\item Международная научная конференция студентов, аспирантов и молодых ученых <<Ломоносов>> (МГУ им. М.В.\,Ломоносова, Москва, 2014, 2015, 2016, 2017, 2018 годы); 
\item Конференция <<Ломоносовские чтения>> (НИИ механики, МГУ им. М.В. Ломоносова, 2014, 2015, 2016, 2017, 2018 годы); 
\item Международная конференция <<Нелинейные задачи теории гидродинамической устойчивости и турбулентность>> (Звенигород, 2014, 2016, 2018 годы); 
\item Школа-семинар <<Современные проблемы аэрогидродинамики>> (Сочи, 2014, 2016 годы);  
\item XI Всероссийский съезд по фундаментальным проблемам теоретической и прикладной механики (Казань, 2015 год) --- пленарный доклад;
\item 7th International Symposium on Bifurcations and Instabilities in Fluid Dynamics (Париж, 2015 год);
\item 15th European Turbulence Conference (Делфт, Нидерланды, 2015 год); 
\item 18th International Conference on the Methods of Aerophysical Research (Пермь, 2016 год);
\item Международная конференция <<Турбулентность, динамика атмосферы и климата>> (Москва, 2018 год).
\end{itemize}
Также результаты работы представлены на 5 научных семинарах:
\begin{itemize}
\item Семинар НИИ механики МГУ по механике сплошных сред под руководством А.Г.\,Куликовского, В.П.\,Карликова и О.Э.\,Мельника;
\item Семинар кафедры газовой и волновой динамики Мехмата МГУ им. М.В. Ломоносова под руководством Р.И.\,Нигматулина;
\item Семинар <<Суперкомпьютерные технологии в науке, образовании и промышленности>> на базе научно-образовательного центра <<Суперкомпьютерные технологии>> под руководством В.А.\,Садовничего;
\item Астрофизический семинар отдела теоретической физики ФИАН им. П.Н.\,Лебедева под руководством А.В.\,Гуревича;
\item Семинар лаборатории общей аэродинамики НИИ механики МГУ под руководством Н.В.\,Никитина.
\end{itemize}
По материалом диссертации опубликовано около 30 научных работ, в том числе 3 статьи в журналах из списков Web of Scienc и Scopus:
\begin{enumerate}
\item Никитин Н.В., Пиманов В.О. Численное исследование локализованных структур в трубах // Изв. РАН. МЖГ. 2015. No 5. С. 64--75.
\item Никитин Н.В., Пиманов В.О. О поддержании колебаний в локализованных турбулентных структурах в трубах // Изв. РАН. МЖГ. 2018. No 1. С. 68--76.
\item Пиманов В.О. О поддержании колебаний в трехмерных бегущих волнах в плоском течении Пуазейля // Вест. Моск. Ун-та. Сер. 1. Математика. Механика. 2018. No 4. С. 47--53;
\end{enumerate}
3 статьи в сборниках трудов:
\begin{enumerate}
\item[4.] В.О. Пиманов. Пространственно-локализованные турбулентные структуры в круглой трубе // Труды конференции-конкурса молодых ученых 13--16 октября 2014 г.
\item[5.] В.О. Пиманов. О механизме самоподдержания локализованных турбулентных структур в трубах // Труды конференции-конкурса молодых ученых 12--14 октября 2015 г под редакцией академика РАН А.Г. Куликовского, профессора В.А. Самсонова. Издательство Московского университета. Москва. 2016. С. 44--51; 
\item[6.] В.О. Пиманов. Некоторые детали механизма самоподдержания турбулентности в пристенных течениях // Труды конференции-конкурса молодых ученых 10--12 октября 2016 г. Корректура. С. 46--53;
\end{enumerate} 
24 тезисов докладов, а также статья в сборнике <<Суперкомпьютерные технологии в науке, образовании и промышленности>>:
\begin{enumerate}
\item[7.] В.О. Пиманов, Суперкомпьютерное моделирование --- путь к пониманию турбулентности // Суперкомпьютерные технологии в науке, образовании и промышленности. том 7. Издательство Московского университета. Москва. 2017. С.  163-170. 
\end{enumerate}
Работа отмечена наградами:
\begin{itemize}
\item Диплом 2-ой степени конференции--конкурса молодых ученых НИИ механики МГУ 2014 года;
\item Диплом 1-ой степени за лучшую работу аспиранта и диплом 3-ей степени конференции-конкурса молодых ученых НИИ механики МГУ 2015 года;
\item Диплом 1-ой степени за лучшую работу аспиранта и диплом 3-ей степени конференции-конкурса молодых учёных НИИ механики МГУ 2016 года;
\item Вторая премия конкурса молодых научных сотрудников МГУ~им.~М.В.~Ломоносова 2015 года;
\item Грамота за лучший доклад на международной научной конференции студентов, аспирантов и молодых ученых <<Ломоносов>> 2015, 2017 и 2018 годов.
\end{itemize}


{\bf Структура и объем диссертации.} 
Диссертация состоит из титульного листа, оглавления, введения, четырех глав, заключения и списка литературы. В первой главе приведены постановка задачи и численный метод ее решения. Во второй главе описана процедура получения модельного порыва и его основные свойства. В третьей главе дано описание механизма образования продольных вихрей. Четвертая глава посвящена получению и анализу отличных от модельного порыва решений уравнений Навье-Стокса, исследованных в работе. Общий объем диссертации --- 122 страниц.




