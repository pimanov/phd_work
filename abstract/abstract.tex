\documentclass[a4paper,14pt]{extreport}
%\usepackage{pscyr}
\usepackage[T2A]{fontenc}
\usepackage[utf8]{inputenc}
\usepackage[english,russian]{babel}
\usepackage{comment}

\usepackage{amssymb}
\usepackage[labelsep=period]{caption}
\usepackage{graphicx}
\graphicspath{{figures/}}

\usepackage{geometry}
\geometry{left=3cm}
\geometry{right=1cm}
\geometry{top=2cm}
\geometry{bottom=2cm}

\makeatletter
\bibliographystyle{ugost2008.bst}
\renewcommand{\@biblabel}[1]{#1.}
\providecommand*{\BibDash}{}

\makeatother
\righthyphenmin=2 % Минимальное число символов при переносе

\pagestyle{plain}
\frenchspacing

\usepackage{cite}
\renewcommand{\baselinestretch}{1.24}
\usepackage{indentfirst} %отступ в начале параграфа
% \usepackage{showkeys} %подписи к меткам
\usepackage{amsmath} % мат.формулы
\usepackage{amssymb} % мат.символы
\usepackage{array} % расширенные функции для оформления таблиц
% \usepackage{natbib} % расширенные функции для оформления таблиц
%\usepackage{epsfig}

\renewcommand{\topfraction}{0.9}
\usepackage[pdfborder={0 0 0}]{hyperref}

%-----------------------------------------------------------%

\newcommand{\tocsecindent}{\hspace{0mm}}

%\makeatletter
%\renewcommand*\l@chapter{\@dottedtocline{0}{0em}{1.3em}}
%\makeatother

%-----------------------------------------------------------%
%-----------------------------------------------------------%

\let\vaccent=\v % rename builtin command \v{} to \vaccent{}
\renewcommand{\v}[1]{\mathbf{#1}} % for vectors
\newcommand{\gv}[1]{\ensuremath{\mbox{\boldmath$ #1 $}}} 
% for vectors of Greek letters
\newcommand{\uv}[1]{\ensuremath{\mathbf{\hat{#1}}}} % for unit vector
\newcommand{\abs}[1]{\left| #1 \right|} % for absolute value
\newcommand{\avg}[1]{\left< #1 \right>} % for average
\let\underdot=\d % rename builtin command \d{} to \underdot{}
\renewcommand{\d}[2]{\frac{d #1}{d #2}} % for derivatives
\newcommand{\dd}[2]{\frac{d^2 #1}{d #2^2}} % for double derivatives
\newcommand{\pd}[2]{\frac{\partial #1}{\partial #2}} 
\newcommand{\pdone}[2]{\partial #1 / \partial #2} 
% for partial derivatives
\newcommand{\pdd}[2]{\frac{\partial^2 #1}{\partial #2^2}} 
% for double partial derivatives
\newcommand{\pdc}[3]{\left( \frac{\partial #1}{\partial #2}
 \right)_{#3}} % for thermodynamic partial derivatives
\newcommand{\ket}[1]{\left| #1 \right>} % for Dirac bras
\newcommand{\bra}[1]{\left< #1 \right|} % for Dirac kets
\newcommand{\braket}[2]{\left< #1 \vphantom{#2} \right|
 \left. #2 \vphantom{#1} \right>} % for Dirac brackets
\newcommand{\matrixel}[3]{\left< #1 \vphantom{#2#3} \right|
 #2 \left| #3 \vphantom{#1#2} \right>} % for Dirac matrix elements
\newcommand{\grad}[1]{\nabla #1} % for gradient
\let\divsymb=\div % rename builtin command \div to \divsymb
%\renewcommand{\div}[1]{\nabla \cdot #1} % for divergence
%\newcommand{\curl}[1]{\nabla \times #1} % for curl
\let\baraccent=\= % rename builtin command \= to \baraccent
\renewcommand{\=}[1]{\stackrel{#1}{=}} % for putting numbers above =
\renewcommand{\phi}{\varphi}
\def\No{\textnumero}

\def\v{\mathbf{v}}
\def\u{\mathbf{u}}
\def\x{\mathbf{x}}
\def\n{\mathbf{n}}
\def\V{\mathbf{V}}
\def\U{\mathbf{U}}
\def\F{\mathbf{F}}
\def\n{\mathbf{n}}
\def\c{\mathbf{c}}
\def\p{\mathbf{p}}
\def\d{\partial}
\def\om{\ensuremath{\mbox{\boldmath$\omega$}}} 
\def\Om{\ensuremath{\mbox{\boldmath$\Omega$}}} 
\def\rot{\mathop{}\!\operatorname{rot}}
\def\div{\mathop{}\!\operatorname{div}}
%\def\grad{\mathop{}\!\operatorname{grad}}
\def\grad{\nabla}
%\def\Laplace{\mathop{}\!\mathbin\bigtriangleup}
\def\Laplace{\mathop{}\!\Delta}
\def\Re{\operatorname{Re}}
%\def\i{\uv{i}}

\begin{document}
%В автореферате диссертации излагаются 1) положения, выносимые на защиту, 2) основные идеи и выводы диссертации, 3) показываются вклад автора в проведенное исследование, 4) степень новизны и практическая значимость результатов исследования, 4) содержатся сведения об организации, в которой выполнялась диссертация, об оппонентах, о научных руководителях и научных консультантах соискателя ученой степени (при наличии), 5) приводится список публикаций автора диссертации, в которых отражены основные научные результаты диссертации. 
\begin{titlepage}
\newpage

%\pagestyle{plain} \thispagestyle{empty}

\begin{center}
{\bf МОСКОВСКИЙ ГОСУДАРСТВЕННЫЙ УНИВЕРСИТЕТ} \\[3pt]
{\bf имени М.В. ЛОМОНОСОВА} \\[3pt]
\begin{tabular}{p{\textwidth}}
\hline {} \\[-14pt]
\end{tabular}
МЕХАНИКО-МАТЕМАТИЧЕСКИЙ ФАКУЛЬТЕТ \\[40pt]
\end{center}

\begin{flushright}
{\it На правах рукописи}\\
УДК 532.517.3: 532.542.3\\[50pt]
\end{flushright}

\begin{center}
{\bf Пиманов Владимир Олегович} \\[30pt]
{\bf ЧИСЛЕННОЕ ИССЛЕДОВАНИЕ ЛОКАЛИЗОВАННЫХ} \\ 
{\bf ТУРБУЛЕНТНЫХ СТРУКТУР В ТРУБАХ} \\[30pt]
{\it 01.02.05 -- механика жидкости, газа и плазмы} \\[30pt]
Диссертация на соискание ученой степени \\
кандидата физико-математических наук \\[30mm]
\hfill\parbox[t]{80mm}{Научный руководитель: \\ д.ф.-м.н., проф. Н.В.~Никитин} \\[5cm]
Москва -- 2017
\end{center}

\end{titlepage}

\nocite{Vest18, MZG2017, MZG2015, Kazan2015, KMU2016, KMU2015, KMU2014} %Almanac2017, report
\nocite{Lomonosov2018, Lomonosov2017, Lomonosov2016, Lomonosov2015, Lomonosov2014, LomRead2017, LomRead2016, LomRead2014, NeZaTeGiUs2018, NeZaTeGiUs2016, NeZaTeGiUs2014, Bur2016, Bur2014, Ob2018, Kazan2015conf} %ICMAR2016

\section*{\centering  Общая характеристика работы}
\addcontentsline{toc}{chapter}{Общая характеристика работы}

\textbf{Актуальность темы и степень ее разработанности.}
Изучение закономерностей движения жидкостей и газов в круглых трубах имеет большое значение как с практической, так и с теоретической точек зрения. Известно, что при небольших значениях числа Рейнольдса $\Re$ течение оказывается ламинарным, а при достаточно больших --- турбулентным. При переходных значениях $\Re$ ламинарный и турбулентный режимы течения могут сосуществовать, при этом участки возмущенного и спокойного движения следуют вдоль трубы друг за другом, практически не меняя своей протяженности. 

Экспериментально установлено (Wygnanski I.J. \& Champagne F.H. // JFM. 1973. Vol. 59, no. 2), что в разных условиях возникают локализованные турбулентные структуры заметно разных типов. Структуры первого типа появляются при сильной возмущенности потока на входе в трубу в диапазоне $2000<\Re<2700$. Они сносятся вниз по потоку со скоростью, близкой к средней скорости течения, практически не меняя свою протяженность. Будем называть такие структуры \textit{турбулентными порывами} (turbulent puffs). Для порыва характерны размытость передней границы, на которой скорость на оси трубы плавно уменьшается от ламинарного значения на 30 -- 40\%, и резкость задней границы, на которой происходит возвращение к ламинарному течению. Структуры другого типа --- турбулентные пробки --- появляются при $\Re>3200$ только когда возмущенность потока на входе недостаточна для непосредственного возникновения турбулентности. Тогда возможен переход через турбулентные пробки --- локализованные образования, расширяющиеся по мере сноса вниз по течению. 

Турбулентный порыв представляет собой интересный гидродинамический объект, который в некотором отношении может рассматриваться как структурная единица турбулентности. В последние годы выполнен ряд подробных экспериментальных и численных исследований характеристик и свойств турбулентных порывов. Установлено, что турбулентный порыв является нестабильным образованием, склонным либо к исчезновению, либо к делению. С каждой из двух конкурирующих тенденций связано характерное время: среднее время жизни порыва до его исчезновения и среднее время до его разделения. Первое увеличивается с ростом $\Re$, второе уменьшается. Согласно (Avila K. et al. // Science. 2011. Vol. 333, no. 6039), значение $\Re^*=2040$, при котором происходит смена доминирования тенденций, является точкой статистического фазового перехода и может быть принята в качестве минимального критического числа Рейнольдса в круглой трубе. 

В последние годы акцент в изучении механизма самоподдержания турбулентности смещается от лабораторного эксперимента в сторону эксперимента вычислительного, основанного на численном решении уравнений Навье--Стокса. Турбулентные порывы впервые рассчитаны в (Priymak V.G. \& Miyazaki T. // Phys. Fluids. 2004. Vol. 16, no. 12). Попытка объяснения механизма самоподдержания турбулентного порыва предпринята в (Shimizu M. \& Kida S. // Fluid Dyn. Res. 2009. Vol. 41, no. 4). 

Изучение динамики турбулентного порыва осложнено в первую очередь стохастичностью процесса, когда отдельные его фазы следуют друг за другом случайным образом. В этих условиях определенная ясность может быть получена из анализа более простых структур, аппроксимирующих порыв, недавно найденных в (Avila M. et al. // Phys. Rev. Let. 2013. Vol. 110, no. 22). Это предельные решения, возникающие на сепаратрисе, разделяющей в фазовом пространстве области притяжения решений, соответствующих ламинарному и турбулентному режимам течения. Такие решения, наследуя ряд качественных характеристик турбулентного порыва, оказываются периодическими по времени в подходящей подвижной системе отсчета. Будем называть такие структуры {\it модельными порывами}. Простота поведения позволяет провести исчерпывающее исследование свойств модельного порыва, которые, как мы полагаем, проясняют определенные детали поведения турбулентного порыва. 

Методом продолжения по параметру могут быть получены другие условно периодические решения уравнений Навье-Стокса (периодические в подходящей подвижной системе отчета), имеющие пространственно-локализованную структуру. Также, в настоящее время известно достаточно большое число решений, имеющих вид трехмерной бегущей волны (периодичных вдоль потока и стационарных в подходящей подвижной системе отсчета) (Kawahara G. et al. // An. Rev. Fluid Mech. 2012. Vol. 44). Такие решения также допускаю детальное исследование и могут быть использованы для установления универсальности выявленных при исследовании модельного порыва закономерностей. 

{\bf Цель диссертационной работы} состоит в выявлении механизмов, ответственных за возникновение и поддержание турбулентных порывов. 
С этой целью поставлены и решены следующие \textbf{задачи}: 

\noindent $1.$ Проведено численное исследование модельного порыва --- условно периодического решения уравнений Навье-Стокса в геометрии течения в круглой трубе, имеющего пространственно-локализованную структуру. %Воспроизводя ряд качественных особенностей турбулентного порыва, модельный порыв имеет более простое временное поведение, что позволяет выполнить его детальное исследование. 

\noindent $2.$ Методом продолжения по параметру рассчитаны другие условно периодические решения уравнений Навье-Стокса в геометрии течения в круглой трубе, имеющие пространственно-локализованную структуру. Выполнен их анализ. %Их анализ позволил оценить универсальность найденных при исследовании модельного порыва закономерностей. 

\noindent $3.$ Рассчитаны и исследованы трехмерные решения уравнений Навье-Стокса в геометрии течения в круглой трубе и геометрии течения в плоском канале, имеющие вид бегущей волны. %Анализ таких решений также позволил оценить универсальность найденных при изучении модельного порыва закономерностей. 


{\bf Научная новизна и положения, выносимые на защиту.} На защиту выносятся следующие новые результаты, полученные в диссертации:

\noindent $1.$ Определены основные элементы механизма поддержания колебаний в модельном порыве --- условно периодическом решении уравнений Навье-Стокса с пространственно-локализованной структурой. Поле скорости решения может быть представлено в виде суперпозиции средней и пульсационной составляющих. Характерной особенностью среднего течения является наличие вытянутых вдоль потока полос повышенной и пониженной скорости. Пульсации возникают в результате линейной неустойчивости среднего течения в областях между соседними полосами на фоне резкого изменения скорости вдоль угловой координаты. Нелинейное взаимодействие пульсаций приводит к формированию продольных вихрей, поддерживающих существование полос.

\noindent $2.$ Обнаружен нелинейный механизм поддержания продольных вихрей, вызывающих полосчатое искажение в распределении продольной скорости. Существование продольных вихрей поддерживается нелинейным взаимодействием пульсаций продольной скорости и пульсаций продольной завихренности. Пульсации продольной завихренности образуются за счет сжатия и растяжения существующих в среднем течении вихревых трубок пульсациями продольной скорости, что обеспечивает необходимую для поддержания продольных вихрей согласованность фаз между этими пульсациями. 

\noindent $3.$ Определены основные элементы механизма поддержания колебаний в условно-периодических решениях уравнений Навье-Стокса с пространственно локализованной структурой, полученных продолжением по параметру решения, соответствующего модельному порыву. Также механизм поддержания колебаний определен в нескольких семействах решений, имеющих вид бегущей волны, описывающих течения в круглой трубе и в плоском канале. Во всех исследованных решениях механизм поддержания колебаний аналогичен найденному при исследовании модельного порыва, что в некоторой степени подтверждает универсальность этого механизма. 

{\bf Теоретическая и практическая значимость полученных результатов.}
Неустойчивость полосчатых структур является неотъемлемым элементом всех сценариев поддержания колебаний в пристенных турбулентных течениях, что дает основания полагать, что полученные в работе представления о механизме поддержания колебаний в модельных течениях могут быть обобщены на этот класс течений. Понимание механизмов поддержания колебаний имеет первостепенное значения для предсказания характеристик пристенных турбулентных течений и разработки эффективных методов управления ими.

\textbf{Метод исследования и достоверность результатов.}
В работе движение жидкости воспроизводится численно, путем решения полных трехмерных уравнений Навье-Стокса. Численный метод совмещает конечно-разностную аппроксимацию второго порядка точности по пространственным переменным и полунеявный метод Рунге-Кутты третьего порядка точности интегрирования по времени (Nikitin N. // J. Comp. Phys. 2006. Vol. 217, no. 2). Качество программной реализации численного метода и его адекватность целям работы подтверждаются результатами моделирования турбулентного течения в трубе при переходных значениях числа Рейнольдса, приведенные в диссертации. Решения на сепаратрисе и, в частности, модельный порыв найдены итерационным методом (Avila M. et al. 2013). Метод продолжения по параметру основан на применении метода Ньютона.

Для подтверждения корректности результатов численных расчетов, все исследованные решения найдены на нескольких расчетных сетках. Где возможно, выполнено сравнение с экспериментальными данными и результатами расчетов других авторов. Все сравнения подтверждают, что найденные численные решения соответствуют решениям уравнений Навье-Стокса и отражают физику явления. Проведенный в работе анализ нескольких семейств решений подтверждает универсальность выявленных в работе закономерностей.

{\bf Апробация результатов.} Основные результаты работы представлены на следующих конференциях: 
Конференции-конкурсе молодых ученых НИИ механики МГУ (2014, 2015, 2016, 2017 годы),
Международной научной конференции студентов, аспирантов и молодых ученых <<Ломоносов>> (МГУ им. М.В.\,Ломоносова, Москва, 2014, 2015, 2016, 2017, 2018 годы), 
Конференции <<Ломоносовские чтения>> (НИИ механики МГУ, 2014, 2015, 2016, 2017, 2018 годы),
Международной конференции <<Нелинейные задачи теории гидродинамической устойчивости и турбулентность>> (Звенигород, 2014, 2016, 2018 годы),
Школе-семинаре <<Современные проблемы аэрогидродинамики>> (Сочи, 2014, 2016 годы),
XI Всероссийском съезде по фундаментальным проблемам теоретической и прикладной механики (Казань, 2015 год),
7th International Symposium on Bifurcations and Instabilities in Fluid Dynamics (Paris, 2015),
15th European Turbulence Conference (Delft, Netherlands, 2015),
18th International Conference on the Methods of Aerophysical Research (Пермь, 2016 год),
Международной конференции <<Турбулентность, динамика атмосферы и климата>> (Москва, 2018).

Также результаты работы представлены на научных семинарах: 
Семинаре НИИ механики МГУ по механике сплошных сред под руководством А.\,Г.\,Куликовского, В.\,П.\,Карликова и О.\,Э.\,Мельника, 
Семинаре кафедры газовой и волновой динамики мехмата МГУ им. М.\,В.\,Ломоносова под руководством Р.\,И.\,Нигматулина,
Семинаре <<Суперкомпьютерные технологии в науке, образовании и промышленности>> на базе научно-образовательного центра <<Суперкомпьютерные технологии>> под руководством В.\,А.\,Садовничего,
Астрофизическом семинаре отдела теоретической физики ФИАН им. П.\,Н.\,Лебедева под руководством А.\,В.\,Гуревича.
%Семинаре лаборатории общей аэродинамики НИИ механики МГУ под руководством Н.\,В.\,Никитина.

%Работа отмечена следующими наградами: 
%диплом 2-ой степени конференции--конкурса молодых ученых НИИ механики МГУ 2014 года,
%диплом 1-ой степени за лучшую работу аспиранта и диплом 3-ей степени конференции-конкурса молодых ученых НИИ механики МГУ 2015 года,
%диплом 1-ой степени за лучшую работу аспиранта и диплом 3-ей степени конференции-конкурса молодых учёных НИИ механики МГУ 2016 года,
%вторая премия конкурса молодых научных сотрудников МГУ~им.~М.\,В.\,Ломоносова 2015 года,
%грамота за лучший доклад на международной научной конференции студентов, аспирантов и молодых ученых <<Ломоносов>> 2015, 2017 и 2018 годов.


\textbf{Публикации.} 
По материалом диссертации опубликовано 4 статьи в научных журналах, 3 статьи в сборниках трудов и 15 тезисов докладов. Из них 3 статьи в рецензируемых научных изданиях, индексируемых в базах данных Web of Science и Scopus, одна --- в научном журнале, входящем в перечень изданий, рекомендованных ВАК при Министерстве образования и науки РФ. 

{\bf Личный вклад автора.} 
Все численные эксперименты и анализ результатов расчетов, приведенные в диссертации, выполнены автором лично. Также автором написан пакет программ на языке Python, реализующий алгоритм поиска решения на сепаратрисе и метод Ньютона для поиска условно периодических решений уравнений Навье-Стокса. Подготовка к публикации полученных результатов проводилась вместе с соавторами, причём вклад диссертанта был определяющим. Основная работа по подготовке текста диссертации и иллюстративного материала также принадлежит диссертанту. 

{\bf Структура и объем диссертации.} 
Диссертация состоит из введения, четырех глав, заключения и списка литературы. Общий объем диссертации 124 страницы, включая 43 рисунка. Список литературы содержит 92 пункта. 

%В автореферате диссертации излагаются 1) положения, выносимые на защиту, 2) основные идеи и выводы диссертации, 3) показываются вклад автора в проведенное исследование, 4) степень новизны и практическая значимость результатов исследования, 4) содержатся сведения об организации, в которой выполнялась диссертация, об оппонентах, о научных руководителях и научных консультантах соискателя ученой степени (при наличии), 5) приводится список публикаций автора диссертации, в которых отражены основные научные результаты диссертации. 



\section*{\centering Содержание работы}
\addcontentsline{toc}{chapter}{Содержание работы}

Во \textbf{введении} определены актуальность избранной темы, степень ее разработанности, цели и задачи диссертационной работы, ее научная новизна, теоретическая и практическая значимость, методология диссертационного исследования, положения, выносимые на защиту, степень достоверности полученных результатов, апробация результатов и личный вклад автора. 

В {\bf главе 1} описан метод исследования. В работе движение жидкости моделируется численными решениями полных трехмерных уравнений Навье-Стокса. Существенным достоинством численных методов при изучении механизмов турбулентности является то, что расчет дает полную информацию о течении. Подробный численный анализ ряда пристенных течений показал, что численные решения уравнений Навье-Стокс с высокой точностью воспроизводят наблюдаемые в экспериментах особенности движения. Прямое численное моделирование зарекомендовало себя эффективным методом изучения пристенных турбулентных течений. 

Во \textbf{введении к главе 1} приведен краткий историко-литературный обзор развития численных методов для решения задач гидродинамики и краткий обзор подходов к прямому численному моделированию пристенных турбулентных течений. 

В \textbf{разделе 1.1} приведена постановка задачи. В работе рассматривается движение вязкой несжимаемой жидкости в прямой трубе круглого сечения. Движение жидкости описывается уравнениями Навье-Стокса и неразрывности:
$$
\pd{\v}{t} = - (\v \cdot \nabla) \v - \frac{1}{\rho}\grad p + \nu \nabla^2 \v,
$$
$$
\nabla \cdot \v = 0,
$$
где $\v$ --- поле скорости, $p$ --- давление, $\rho$ и $\nu$ --- постоянные плотность жидкости и кинематический коэффициент вязкости, $t$ --- время. 
На стенках трубы, имеющей радиус $R$, ставится условие прилипания. В продольном направлении на поле скорости накладывается условие периодичности с периодом $L_x$. Жидкость приводится в движение за счет внешнего градиента давления, который определяется из условия постоянства средней скорости $U_m$. 

Задача решается в безразмерных переменных. В качестве основных единиц измерения выступают радиус трубы $R$, максимальная скорость течения Пуазейля $U = 2U_m$ и плотность жидкость $\rho$. Безразмерным параметром системы является число Рейнольдса $ \Re = {R U}/{\nu}$.

Постановка задачи традиционна для прямого расчета развитых турбулентных течений в трубах и каналах. В такой постановке удается воспроизводить характеристики течения, устанавливающегося на большом удалении от входа в трубу, решая уравнения движения в ограниченной расчетной области. Условие периодичности вдоль трубы освобождает от необходимости устанавливать условия на входе и выходе из трубы. В тоже время, увеличивая длину периода $L_x$, можно минимизировать влияние этого условия на поток.

В \textbf{разделе 1.2} описан конечно-разностный метод решения поставленной задачи. В расчетной области вводится структурированная сетка, ребра которой совпадают с координатными линиями цилиндрической системы координат $(x,r,\theta)$. По координатам $x$ и $\theta$ сетка однородна. В нормальном к стенке направлении вводится растяжение сетки, что позволяет сгустить сетку вблизи стенки, где наблюдаются наибольшие градиенты скорости. Дискретизация уравнений выполняется на так называемых смещенных сетках: различные скалярные величины отнесены к различным точкам сетки. Дискретизация по пространственным переменным выполнена со вторым порядком точности. Для интегрирования по времени применен полунеявный метод Рунге-Кутты третьего порядка точности. 

Дискретная система уравнений обладает рядом важных консервативных свойств. Дивергенция поля завихренности в расчете тождественно равна нулю. Дискретный аналог ротора градиента давления тождественно равен нулю, что гарантирует отсутствие влияния давления на эволюцию поля завихренности напрямую. Дискретный аналог дивергенции вязкого слагаемого тождественно равна нулю, что гарантирует отсутствие производства массы, вызванного этим слагаемым. Градиент периодической составляющей давления и нелинейные слагаемые не производят кинетической энергии. 
Наиболее полно метод описан в работе (Nikitin N., J. Comp. Phys., 2006, 217(2)).

В \textbf{разделе 1.3} сказано о реализации численного метода и методике проведения численных экспериментов. Пакет программ, реализующих численный метод, написан на языке программирования Fortran77. Помимо последовательного варианта программы реализован параллельный, позволяющий выполнять расчеты на кластерных вычислительных системах с распределенной памятью. Для коммуникации между процессами использован интерфейс передачи сообщений MPI (Message Passing Interface). Работа выполнена с использованием оборудования Центра коллективного пользования сверхвысокопроизводительными вычислительными ресурсами МГУ имени М.В.\,Ломоносова. Значительная часть кода, необходимого при анализе полученных результатов и управлении численными экспериментами, в том числе реализующая метод продолжения по параметру, была написана на высокоуровневом языке Python. 

В \textbf{разделе 1.4} приведены результаты расчетов движения жидкости в круглой трубе в диапазоне $1670 \leqslant \Re \leqslant 2800$ в достаточно протяженной расчетной области для того, чтобы воспроизвести явление пространственной локализации турбулентности. Турбулентность в расчетах принимает форму локализованных структур, характеристики которых совпадают с характеристиками турбулентных порывов, приведенными в литературе. Это подтверждает адекватность численного метода целям работы и качество его программной реализации. 





\section*{Основные результаты и выводы диссертации}

\section*{Публикации автора по теме диссертации}

\end{document}
