
\chapter{Верхняя ветвь решения}

\section{Введение}

В предыдущей главе, исследуя модельный порыв, удалось получить некоторые представления о его структуре и механизме самоподдержания. В настоящей главе поднимается вопрос об общности полученных результатов. Тот факт, что продольные вихри и пристенные полосы присутствуют как в других инвариантных решениях \cite{Kawahara2012}, так и в пристенной турбулентности \cite{Kline1967, Smith1983, Schoppa2002}, позволяет предполагать, что выделенный механизм генерации пульсаций может быть частично или полностью обобщен на эти случаи. Выполнить строгое исследование турбулентного течения сегодня не представляется возможным, однако полученные результаты могут быть проверены на других инвариантны решения. Опираясь на модельный порыв метод продления в пространстве параметров \cite{Sanchez2004, Viswanath2007, Dijkstra2014} позволяет получить новые локализованные в пространстве инвариантные решения. Среди них могут быть выделены такие, характеристики которых ближе к наблюдаемым в турбулентном течении, что повышает ценность полученных при их исследовании результатов. 

\begin{comment}
Все периодические решения удовлетворяют нелинейному уравнению вида
$$
\v_p = \phi(\v_p, T, c_p, Re).
$$ 
Функция $\phi(\v, t, c, \Re)$ возвращает поле скорости, возникающее в результате эволюции поля скорости $\v$ в течении времени $t$ при числе Рейнольдса $\Re$ в системе отсчета, двигающейся со скоростью $\c$. $\v_p$, $T$ и $c_s$ --- подлежащие определению поле скорости периодического решения, его период во времени и скорость перемещения вдоль трубы.
\end{comment}
Все периодические решения удовлетворяют нелинейному уравнению, которое может быть решено методом Ньютона, обобщенным на случай многих неизвестных, называемым также методом Ньютона-Рафсона \cite{}. Метод Ньютона итерационный и требует начального приближения из некоторой окрестности решения. С произвольным начальным приближением метод не сходится, основная сложности в поиске инвариантных решений сводится к нахождению подходящего начального приближения. Однако, если хотя бы одно решение известно, оно может выступать в качестве начального приближения к решению с близким значением параметров, в частности, числа Рейнольдса, с которым метод сойдется. Таким образом, решение может быть продлено в сторону увеличения или уменьшения числа Рейнольдса. В пространстве $(T, c_f, \Re)$ решение принадлежит однопараметрическому множеству. Следуя работе \cite{Avila2013}, продлевая модельный порыв в сторону уменьшения числа Рейнольдса, при $\Re_{biff} = $ удалось достичь точки бифуркации (В \cite{Avila2013} сообщается о $\Re_{biff} = $), в которой возникает две ветви решения, и перейти с нижней ветви решения на верхнюю. Верхняя ветвь решения характеризуется более высокой интенсивностью пульсаций и меньшей скоростью сноса, чем нижняя, что приближает характеристики решения к турбулентному течению. Если нижняя ветвь решения находится на границе турбулентного бассейна притяжения, так как принадлежит сепаратрисе, верхняя ветвь решения находится внутри турбулентного бассейна притяжения, и возможно участвует в формировании турбулентного аттрактора. 

Исследуя решение с верхней ветви удалось показать, что его структура и механизм самоподдержания аналогичны структуре решения с нижней ветви, что подтверждает общий характер выделенных механизмов. Хотя решение с верхней ветви связано с решение с нижней непрерывным образом, и качественно их механизм поддержания отличаться не может, выделенный механизм мог оказаться неприменим к верхней ветви, если бы был неверно формализовано.

На каждом шаге метода Ньютона-Рафсона возникает необходимость решения матрицы Якоби. В случае задач вычислительной гидродинамики, где число неизвестных достигает миллионов, формирование и даже хранение матрицы Якоби в явном виде невозможно, так как число элементов в ней составляет квадрат числа неизвестных в уравнении. Используя Крыловские методы для решения матрицы Якоби удается радикальным образом сократить число операций и необходимой оперативной памяти, что делает расчеты возможными. В работе реализован метод Ньютона-Крылова \cite{}, основанный на методе GMRES (generalized minimum residual method) \cite{Saad1986}. Альтернативой методу GMRES может быть метод BiCGSTAB (Biconjugate gradient stabilized method) \cite{Sleijpen1993}. 






