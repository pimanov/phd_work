
\chapter{Верхняя ветвь решения}

\begin{comment}
\section{Введение}

В предыдущей главе, исследуя модельный порыв, удалось получить некоторые представления о его структуре и механизме самоподдержания. В настоящей главе поднимается вопрос об общности полученных результатов. Тот факт, что продольные вихри и пристенные полосы присутствуют как в других инвариантных решениях \cite{Kawahara2012}, так и в пристенной турбулентности \cite{Kline1967, Smith1983, Schoppa2002}, позволяет предполагать, что выделенный механизм генерации пульсаций может быть частично или полностью обобщен на эти случаи. Выполнить строгое исследование турбулентного течения сегодня не представляется возможным, однако полученные результаты могут быть проверены на других инвариантны решения. Опираясь на модельный порыв метод продления в пространстве параметров \cite{Sanchez2004, Viswanath2007, Dijkstra2014} позволяет получить новые локализованные в пространстве инвариантные решения. Среди них могут быть выделены такие, характеристики которых ближе к наблюдаемым в турбулентном течении, что повышает ценность полученных при их исследовании результатов. 

%\begin{comment}
Все периодические решения удовлетворяют нелинейному уравнению вида
$$
\v_p = \phi(\v_p, T, c_p, Re).
$$ 
Функция $\phi(\v, t, c, \Re)$ возвращает поле скорости, возникающее в результате эволюции поля скорости $\v$ в течении времени $t$ при числе Рейнольдса $\Re$ в системе отсчета, двигающейся со скоростью $\c$. $\v_p$, $T$ и $c_s$ --- подлежащие определению поле скорости периодического решения, его период во времени и скорость перемещения вдоль трубы.
%\end{omment}
Все периодические решения удовлетворяют нелинейному уравнению, которое может быть решено методом Ньютона, обобщенным на случай многих неизвестных, называемым также методом Ньютона-Рафсона \cite{}. Метод Ньютона итерационный и требует начального приближения из некоторой окрестности решения. С произвольным начальным приближением метод не сходится, основная сложности в поиске инвариантных решений сводится к нахождению подходящего начального приближения. Однако, если хотя бы одно решение известно, оно может выступать в качестве начального приближения к решению с близким значением параметров, в частности, числа Рейнольдса, с которым метод сойдется. Таким образом, решение может быть продлено в сторону увеличения или уменьшения числа Рейнольдса. В пространстве $(T, c_f, \Re)$ решение принадлежит однопараметрическому множеству. Следуя работе \cite{Avila2013}, продлевая модельный порыв в сторону уменьшения числа Рейнольдса, при $\Re_{biff} = $ удалось достичь точки бифуркации (В \cite{Avila2013} сообщается о $\Re_{biff} = $), в которой возникает две ветви решения, и перейти с нижней ветви решения на верхнюю. Верхняя ветвь решения характеризуется более высокой интенсивностью пульсаций и меньшей скоростью сноса, чем нижняя, что приближает характеристики решения к турбулентному течению. Если нижняя ветвь решения находится на границе турбулентного бассейна притяжения, так как принадлежит сепаратрисе, верхняя ветвь решения находится внутри турбулентного бассейна притяжения, и возможно участвует в формировании турбулентного аттрактора. 

Исследуя решение с верхней ветви удалось показать, что его структура и механизм самоподдержания аналогичны структуре решения с нижней ветви, что подтверждает общий характер выделенных механизмов. Хотя решение с верхней ветви связано с решение с нижней непрерывным образом, и качественно их механизм поддержания отличаться не может, выделенный механизм мог оказаться неприменим к верхней ветви, если бы был неверно формализовано.

На каждом шаге метода Ньютона-Рафсона возникает необходимость решения матрицы Якоби. В случае задач вычислительной гидродинамики, где число неизвестных достигает миллионов, формирование и даже хранение матрицы Якоби в явном виде невозможно, так как число элементов в ней составляет квадрат числа неизвестных в уравнении. Используя Крыловские методы для решения матрицы Якоби удается радикальным образом сократить число операций и необходимой оперативной памяти, что делает расчеты возможными. В работе реализован метод Ньютона-Крылова \cite{}, основанный на методе GMRES (generalized minimum residual method) \cite{Saad1986}. Альтернативой методу GMRES может быть метод BiCGSTAB (Biconjugate gradient stabilized method) \cite{Sleijpen1993}. 

В континуальной постановке не формулируется. 
\end{comment}

\section{Продление решений в пространстве параметров}

Решение поставленной задачи $\v_p(x,r,\theta,t)$ является периодическим по времени с периодом $T$, если оно удовлетворяет условию
\begin{equation} \label{tper_eq}
\v_p(x,r,\theta,t) = \v_p(x + c_p T, r, \theta, t + T),
\end{equation}
где $c_p$ --- скорость перемещения решения вдоль трубы. Возможность смещения решения в угловом направлении исключена условием отражения относительно плоскости $\theta = 0$ \eqref{sym_eq}. Периодическое решение инвариантно относительно двух непрерывных симметрий. Решение остается решением после его смещения вдоль трубы на произвольное расстояние, а также при смещении по времени на произвольную величину. 

В конечно-разностной постановке условие периодичности по времени может быть сформулировано иначе
\begin{equation}\label{P_eq}
\phi(\v_p, T, c_p, \Re, \dots) - \v_p = 0.
\end{equation}
Здесь функция $\phi(\v, t, c, \Re, \dots)$ возвращает поле скорости, возникающее в результате эволюции поля скорости $\v$ в течении времени $t$ при числе Рейнольдса $\Re$ в системе отсчета, двигающейся со скоростью $c$. В число параметров функции $\phi$ могут входить также и другие величины, такие как длина периода вдоль трубы $L_x$ или длина периода в угловом направлении $2\pi/n$, если полагать, что $n$ может принимать действительные значения. С практической точки зрения, вычисление функции $\phi$ требует численного интегрирования поля скорости $\v$ по времени. 

Для того, чтобы уравнение \eqref{P_eq} имело однозначное решение, необходимо исключить возможность смещения решения вдоль трубы и во времени. Для этого на поле скорости $\v_p$ накладывается пара дополнительных условий вида 
\begin{equation}\label{Pplus_eq}
r_{1,2}(\v_p) = 0.
\end{equation}
В этом случае, если поле скорости однозначно задается $N$ переменными, то в системе $N+2$ уравнений. Для того, чтобы число неизвестных в системе совпадало с числом уравнений, необходимо вместе с полем скорости $\v_p$ положить неизвестными два параметра системы. В случае, если решение ищется при фиксированном значении $\Re$ на заданной сетке, это могу быть $T$ и $c_p$. 

Система \eqref{P_eq}, \eqref{Pplus_eq} может быть представленная в виде 
\begin{equation}\label{F_eq}
F(\x) = 0, 
\end{equation}
где вектор $\x = (\v_p, T, c_p, \Re, \dots)$ объединяет все переменные системы. Рассмотрим её полный дифференциал системы \eqref{F_eq}
\begin{equation}\label{dF_eq}
dF = \pd{F}{\x}d\x.
\end{equation}
В случае, если фиксированы все параметры кроме трех, например $(T, c_p, \Re)$, Якобиан ${\d F}/{\d \x}$ оказывается прямоугольной матрицей, содержащей $(N+2)$ строк и $(N+3)$ столбцов. В этом случае в окрестности каждого решения существует направление $d\x$, для которого $dF = 0$, при движении вдоль которого решение остается решением, так как значение $F$ сохраняется равным нулю. Тогда решения в пространстве трех параметров принадлежат некоторой кривой, задаваемой одним параметром $\x_p = \Gamma(s)$. Аналогично, в пространстве четырех параметров решения принадлежат двухпараметрическому семейству, и т.д. Точки, где Якобиан имеет не полный ранг, могут быть точками бифуркации, в которой могут сходиться несколько ветвей решения. 

Таким образом, периодическое по времени решение, найденное в предыдущей главе, может быть продлено как в сторону увеличения, так и в сторону уменьшения числа Рейнольдса. Естественно параметры расчетной области положить фиксированными, а в качестве определяемых выбрать период решения по времени $T$ и скорость его перемещения вдоль трубы $c_f$. В пространстве параметров $(T, c_f, \Re)$ решения принадлежат некоторой кривой. Двигаясь по ней в сторону уменьшения $\Re$, удается достичь точки бифуркации, в которой решение возникает (При меньших $\Re$ оно не существует). В этой точке кривая, которой принадлежат решения, совершает разворот так, что в сторону увеличения $\Re$ направлено две ветви решения. Их называют нижней и верхней ветвью решения, причем исходное решение принадлежит нижней ветви. В точке бифуркации удается совершить разворот и перейти на верхнюю ветвь решения, двигаясь по которой в торону увеличения $\Re$, можно получить новое решение при исходном значении $\Re$. На каждой ветви решения число Рейнольдса однозначно определяет решение. 

Можно дополнительно отметить, что в силу симметрии отражения относительно плоскости $\theta = \pi/4$ со сдвигом на половину периода по времени $T/2$, которой обладает модельный порыв, условие периодичности по времени \eqref{tper_eq} может быть заменен условием
\begin{equation}\label{shift_eq}
\v_p(x,r,\pi/4 + \theta,t) = \v_p(x + c_p T/2,r,\pi/4 - \theta,t+T/2).
\end{equation}
Тогда уравнение \eqref{P_eq} уступит место следующему
\begin{equation}\label{P2_eq}
\v_p(x,r,\pi/4 - \theta) = \phi(\v_p, T/2, c_p, \Re, \dots)(x,r,\pi/4 + \theta).
\end{equation}
Его вычисление требует вдвое меньше времени, что может быть существенно при проведении численного исследования. При решении исходной системы \eqref{P_eq} существует некоторая вероятность потерять симметрию \eqref{shift_eq} в процессе продления решения, что исключено при решении системы \eqref{P2_eq}.

\section{Метод Ньютона-Крылова}

Система уравнений \eqref{F_eq} может быть решена численно методом Ньютона, обобщенным на многомерный случай, называемым также методом Ньютона-Рафсона. Метод ньютона итерационный и на каждом шаге уточняет существующее приближение к решению. Пусть $x_m$ --- приближение к решению на шаге $m$, а $\x^*$ --- точное решение. Разложение выражения $F(\x^*)$ в ряд около точки $\x_m$ имеет вид
\begin{equation}
F(\x^*) = F(\x_m) + \pd{F}{\x}\bigg|_{\x_m} (\x^* - \x_m) + O(\Delta \x^2). 
\end{equation}
Пренебрегая малыми второго порядка, учитывая, что $F(\x^*) = 0$, получим линейную систему на поправку к решению $\Delta \x_m = \x_{m+1} - \x_m$
\begin{equation}\label{Newton_eq}
\pd{F}{\x}\bigg|_{\x_m} \Delta \x_m = - F(\x_m). 
\end{equation}
Основной задачей при применении метода Ньютона является решение системы \eqref{Newton_eq} и нахождение $\Delta \x_m$. Выполнение шага метода Ньютона завешается вычисление нового приближения к решению $\x_{m+1} = \x_m + \Delta x_m$. 


\def\l{\mathbf{l}}
В случае решения задач вычислительной гидродинамики размерность системы \eqref{Newton_eq} оказывается большой и может измеряться миллионами, её решение требует значительных вычислительных ресурсов. Еще более сложной задачей является формирование матрицы Якоби в явном виде. На практике, для формирование матрицы Якоби пользуются тем фактом, что её произведение с произвольным вектором единичной длины $\l$ равно производной исходно функции $F$ вдоль этого направления
\begin{equation} \label{Jl_eq}
\pd{F}{\x} \l = \pd{F}{\l}. 
\end{equation}
Аналитическое вычисление производной функции $F$ в случае поиска периодических решений не представляется возможным. 
Она может быть получена численно, как конечная разность, по формуле
\begin{equation}\label{fd_eq}
\pd{F}{\l} \approx \frac{F(\x + \varepsilon \l) - F(\x)}{\varepsilon},
\end{equation}
при достаточно малом значении $\varepsilon$. В численных расчетах значение $\varepsilon$ рекомендуют выбирать порядка $10^{-7}$. Вычисление матрицы Якоби сводится к вычислению производной функции $F$ вдоль каждого из базисных направлений, число которых равно числу неизвестных. В свою очередь, вычисление производной $F$ вдоль одного направления требует вызова функции $F$ в новой точке. В случае поиска периодических решений при вызове функции $F$ выполняется численное интегрирование исходных уравнений в течении периода по времени, что требует значительны вычислительных затрат, и выполнить его $O(N)$ раз на практике невозможно. Даже хранение матрицы Якоби в памяти компьютера представляет серьезную вычислительную задачу.

Решить линейную систему \eqref{Newton_eq} позволяют итерационные методы, основанные на подпространствах Крылова. В этом случае обращение к матрице Якоби происходит только в форме её умножения на вектор, что следуя \eqref{fd_eq} сводится к вычислению конечно-разностной производной функции $F$ вдоль направления, задаваемого этим вектором. При решении системы вида 
\begin{equation}\label{Ax_eq}
Ax = b
\end{equation}
подпространство Крылова $K_i$ представляет собой линейную оболочку $i$ векторов
$$
K_i = L(b, Ab, A^2b, \dots, A^{i-1}b)
$$ 
Имея базис подпространства $K_i$, для того, чтобы построить базис в подпространстве $K_{i+1}$, необходимо выполнить только одно умножение матрицы $A$ на уже известный вектор $A^{i-1}b$. При решении системы \eqref{Ax_eq} решение ищется в базисе подпространства Крылова. Крыловские методы оказываются эффективны при поиске инвариантных решений с небольшим числом неустойчивых направлений (для решений на сепаратрисе оно одно). Для уточнения решения на порядок требуется только несколько десятков базисных векторов в подпространстве Крылова. Таким образом, уточнение решения на порядок требует вычисления функции $F(\x)$ порядка $O(10)$ раз, и не зависит от размерности пространства $\x$. Метод Ньютона, в котором для решение линейной системы \eqref{Newton_eq} применяются методы Крыловского типа, называется также методом Ньютона-Крылова. 


В работе был реализован основанный на подпространствах Крылова метод минимизации невязки \cite{Tirtishnikov}. Также популярными являются обобщенный метод минимизации невязок, называемый GMRES (generalized minimal residual algorithm) \cite{} и метод би-сопряженных градиентов, называемый BiCGSTAB (Biconjugate gradient stabilized method) \cite{Sleijpen1993}. 


\section{Реализация метода Ньютона-Крылова для поставленной задачи}

Хотя корректная постановка задачи нахождения периодических решений требует дополнительных условий \eqref{Pplus_eq}, ограничивающих возможность смещения решения вдоль трубы и во времени, на практике их использование может быть неудобно. Если отказаться от дополнительных условие,


\section{Стратегия продления решения}

Реализованный метод Ньютона-Крылова позволяет находить периодические решения, но только в том случае, когда известно достаточно близкое к решению начальное приближение. С произвольными начальными данными метод Ньютона не сходится. В нашем случае в качестве начального приближения может выступать решение на сепаратрисе, найденное в предыдущей главе. Метод Ньютона-Крылова позволяет его уточнить, но кроме этого, оно может быть использовано в качестве начального приближения для решения с близкими значениями параметров, например, числа Рейнольдса. Если смещение по $\Re$ достаточно мало, метод Ньютона сходится, и дает новое периодическое решение. Таким образом, решение может быть продлено в пространстве параметров. Если уже получено несколько решений, приближение к новому может быть построено интерполяцией. В этой работе в основном применялась линейная интерполяция. Если $\x_1$ и $\x_2$ --- известные решения, полученные при числах Рейнольдса $\Re_1$ и $\Re_2$, то приближение к новому решению $\x_3$ при $\Re_3$ может быть получено по формуле 
\begin{equation}
\x_3 = \frac{\x_2(\Re_3 - \Re_1) - \x_1(\Re_3-\Re_2)}{\Re_2 - \Re_1}
\end{equation}

В случае, когда продвижение выполняется по $\Re$, а в качестве определяемых параметров выступают $T$ и $c_f$, преодолеть точку бифуркации и перейти с нижней ветви решения на верхнюю невозможно. Выполнить такой переход позволяет включение в число определяемых параметров $\Re$ вместо $T$ или $c_f$ на выбор. Другим методом решения может быть включение в число определяемых всех трех параметров $\Re$, $T$ и $c_f$. Как показывает практика, сходимость метода Ньютона при этом практически не меняется, хотя с формальной точки зрения она могла быть испорчена. Если в системе уравнений на периодическое решение \eqref{F_eq} неизвестных больше, чем уравнений, теряется однозначность решения. Линейная система для определения шага метода Ньютона \eqref{Newton_eq} получает бесконечно много решений произвольной длины. В процессе её решения может быть получено любое из них, и хотя оно является решением линейной системы, соответствующий ему шаг в пространстве $\x$ может вывести за пределы сходимости метода Ньютона. 

В более общем случае в пространстве параметров $(\Re, T, c_f)$ можно перейти к новой систем координат, одна из осей которой касается кривой $\Gamma(s)$, которой принадлежат решения. Пусть этой оси соответствует переменная $p_0$. Две другие оси, пусть $p_1$ и $p_2$, перпендикулярны оси $p_0$. При поиске нового решения имеет смысл фиксировать значение $p_0$, отделив тем самым новое решение от уже существующих. Тогда $p_1$ и $p_2$ выступают в качестве определяемых параметров и решение ищется в нормальной к кривой $\Gamma(s)$ плоскости. Получить приближение к направлению $p_0$ можно по уже известным решениям. Такой подход позволяет преодолеть точку бифуркации и другие особенности кривой $\Gamma(s)$ в автоматическом режиме. 

На нижней ветви вблизи решения, соответствующего модельному порыву, если в качестве начального приближения к новому решению выступает уже найденное, максимальный шаг по $\Re$ близок к $10$. При использовании линейной интерполяции, шаг может быть увеличен до величины порядка $100$. Хотя по мере продвижения в пространстве параметров шаг, с которым выполняется переход от уже найденного решения к новому, варьируется, можно выделить общую тенденцию, следуя которой по мере приближения к точке бифуркации допустимый шаг уменьшается. Хотя на верхней ветви решения допустимый шаг несколько увеличивается, он остается ниже, чем на нижней, ввиду большей сложности решения с верхней ветви. 
 






