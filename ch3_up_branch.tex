
\chapter{Верхняя ветвь решения}

\begin{comment}
\section{Введение}

В предыдущей главе, исследуя модельный порыв, удалось получить некоторые представления о его структуре и механизме самоподдержания. В настоящей главе поднимается вопрос об общности полученных результатов. Тот факт, что продольные вихри и пристенные полосы присутствуют как в других инвариантных решениях \cite{Kawahara2012}, так и в пристенной турбулентности \cite{Kline1967, Smith1983, Schoppa2002}, позволяет предполагать, что выделенный механизм генерации пульсаций может быть частично или полностью обобщен на эти случаи. Выполнить строгое исследование турбулентного течения сегодня не представляется возможным, однако полученные результаты могут быть проверены на других инвариантны решения. Опираясь на модельный порыв метод продления в пространстве параметров \cite{Sanchez2004, Viswanath2007, Dijkstra2014} позволяет получить новые локализованные в пространстве инвариантные решения. Среди них могут быть выделены такие, характеристики которых ближе к наблюдаемым в турбулентном течении, что повышает ценность полученных при их исследовании результатов. 

%\begin{comment}
Все периодические решения удовлетворяют нелинейному уравнению вида
$$
\v_p = \phi(\v_p, T, c_p, Re).
$$ 
Функция $\phi(\v, t, c, \Re)$ возвращает поле скорости, возникающее в результате эволюции поля скорости $\v$ в течении времени $t$ при числе Рейнольдса $\Re$ в системе отсчета, двигающейся со скоростью $\c$. $\v_p$, $T$ и $c_s$ --- подлежащие определению поле скорости периодического решения, его период во времени и скорость перемещения вдоль трубы.
%\end{omment}
Все периодические решения удовлетворяют нелинейному уравнению, которое может быть решено методом Ньютона, обобщенным на случай многих неизвестных, называемым также методом Ньютона-Рафсона \cite{}. Метод Ньютона итерационный и требует начального приближения из некоторой окрестности решения. С произвольным начальным приближением метод не сходится, основная сложности в поиске инвариантных решений сводится к нахождению подходящего начального приближения. Однако, если хотя бы одно решение известно, оно может выступать в качестве начального приближения к решению с близким значением параметров, в частности, числа Рейнольдса, с которым метод сойдется. Таким образом, решение может быть продлено в сторону увеличения или уменьшения числа Рейнольдса. В пространстве $(T, c_f, \Re)$ решение принадлежит однопараметрическому множеству. Следуя работе \cite{Avila2013}, продлевая модельный порыв в сторону уменьшения числа Рейнольдса, при $\Re_{biff} = $ удалось достичь точки бифуркации (В \cite{Avila2013} сообщается о $\Re_{biff} = $), в которой возникает две ветви решения, и перейти с нижней ветви решения на верхнюю. Верхняя ветвь решения характеризуется более высокой интенсивностью пульсаций и меньшей скоростью сноса, чем нижняя, что приближает характеристики решения к турбулентному течению. Если нижняя ветвь решения находится на границе турбулентного бассейна притяжения, так как принадлежит сепаратрисе, верхняя ветвь решения находится внутри турбулентного бассейна притяжения, и возможно участвует в формировании турбулентного аттрактора. 

Исследуя решение с верхней ветви удалось показать, что его структура и механизм самоподдержания аналогичны структуре решения с нижней ветви, что подтверждает общий характер выделенных механизмов. Хотя решение с верхней ветви связано с решение с нижней непрерывным образом, и качественно их механизм поддержания отличаться не может, выделенный механизм мог оказаться неприменим к верхней ветви, если бы был неверно формализовано.

На каждом шаге метода Ньютона-Рафсона возникает необходимость решения матрицы Якоби. В случае задач вычислительной гидродинамики, где число неизвестных достигает миллионов, формирование и даже хранение матрицы Якоби в явном виде невозможно, так как число элементов в ней составляет квадрат числа неизвестных в уравнении. Используя Крыловские методы для решения матрицы Якоби удается радикальным образом сократить число операций и необходимой оперативной памяти, что делает расчеты возможными. В работе реализован метод Ньютона-Крылова \cite{}, основанный на методе GMRES (generalized minimum residual method) \cite{Saad1986}. Альтернативой методу GMRES может быть метод BiCGSTAB (Biconjugate gradient stabilized method) \cite{Sleijpen1993}. 

В континуальной постановке не формулируется. 
\end{comment}

\section{Продолжение в пространстве параметров}

Решение поставленной задачи $\v_p(x,r,\theta,t)$ является периодическим по времени с периодом $T$, если оно удовлетворяет условию
\begin{equation}
\v_p(x,r,\theta,t) = \v_p(x + c_p T, r, \theta, t + T),
\end{equation}
где $c_p$ --- скорость перемещения решения вдоль трубы. Возможность смещения решения в угловом направлении исключена условием отражения относительно плоскости $\theta = 0$ \eqref{sym_eq}. Периодическое решение инвариантно относительно двух непрерывных симметрий. Решение остается решением после его смещения вдоль трубы на произвольное расстояние, а также при смещении по времени на произвольную величину. 

В конечно-разностной постановке условие периодичности по времени может быть сформулировано иначе
\begin{equation}\label{P_eq}
\phi(\v_p, T, c_p, \Re, \dots) - \v_p = 0
\end{equation}
где функция $\phi(\v, t, c, \Re, \dots)$ возвращает поле скорости, возникающее в результате эволюции поля скорости $\v$ в течении времени $t$ при числе Рейнольдса $\Re$ в системе отсчета, двигающейся со скоростью $c$. В число параметров функции $\phi$ могут входить также и другие величины, такие как длина периода вдоль трубы $L_x$ или длина периода в угловом направлении $2\pi/n$, если полагать, что $n$ может принимать действительные значения. Практически, вычисление функции $\phi$ требует численного интегрирования поля скорости $\v$ по времени. 

Для того, чтобы уравнение \eqref{P_eq} имело однозначное решение, необходимо исключить возможность смещения решения вдоль трубы и во времени. Для этого на поле скорости $\v_p$ может быть наложена пара дополнительных условий вида 
\begin{equation}\label{Pplus_eq}
r_{1,2}(\v_p) = 0.
\end{equation}
В этом случае, если поле скорости однозначно задается $N$ переменными, то в системе $N+2$ уравнений. Для того, чтобы число неизвестных в системе совпадало с числом уравнений, необходимо вместе с полем скорости $\v_p$ положить неизвестными два параметра системы. В частности, это могут быть $T$ и $c_p$ в случае, если решение ищется при фиксированном значении $\Re$ на заданной сетке. 

Система \eqref{P_eq}, \eqref{Pplus_eq} может быть представленная в виде 
\begin{equation}\label{F_eq}
F(\x) = 0, 
\end{equation}
где вектор $\x = (\v_p, T, c_p, \Re, \dots)$ объединяет все неизвестные системы. Для того, чтобы понять структуру решений системы \eqref{F_eq}, рассмотрим её полный дифференциал
\begin{equation}\label{dF_eq}
dF = \pd{F}{\x}d\x.
\end{equation}
В случае, если фиксированы все параметры кроме трех, например $(T, c_p, \Re)$, Якобиан ${\d F}/{\d \x}$ оказывается прямоугольной матрицей размера $(N+2) \times (N+3)$. В этом случае в окрестности каждого решения существует направление $d\x$, для которого $dF = 0$, при движении вдоль которого решение остается решением. Тогда решения в пространстве трех параметров принадлежат некоторой кривой, задаваемой одним параметром $\x_p = \Gamma(s)$. Аналогично, в пространстве четырех параметров решения принадлежат двухпараметрическому семейству, и т.д. Точки, где Якобиан имеет не полный ранг, могут быть точками бифуркации, в которой могут сходиться несколько ветвей решения. 

Таким образом, найденное в предыдущей главе периодическое по времени решение может быть продлено как в сторону увеличения числа Рейнольдса, так и в сторону его уменьшения. При этом, при условии неизменности граничных условий, наложенных на решение, значение $\Re$ однозначно определяет значения $T$ и $c_f$. Двигаясь в сторону уменьшения числа Рейнольдса, удается достичь точки бифуркации, в которой решение рождается, и при меньших $\Re$ его не существует. В этой точке кривая, которой принадлежит решение, изгибается так, что при движении по ней в каждую строну движение происходт в сторону увеличения $\Re$. 


\section{Метод Ньютона-Крылова}

Система уравнений \eqref{F_eq} может быть решена численно методом Ньютона, обобщенным на многомерный случай, называемым также методом Ньютона-Рафсона. Метод ньютона итерационный и на каждом шаге уточняет существующее приближение к решению. Пусть $x_m$ --- приближение к решению на шаге $m$, а $\x^*$ --- точное решение. Разложение выражения $F(\x^*)$ в ряд около точки $\x_m$, при условии $F(x^*) = 0$, имеет вид
\begin{equation}
F(\x_m) = \pd{F}{\x} (\x_m - \x^*) + O(\Delta x^2). 
\end{equation}
Пренебрегая малыми второго порядка, получим линейную систему на поправку к решению $\Delta \x_m$
\begin{equation}\label{Newton_eq}
\pd{F}{\x} \Delta \x_m = - F(\x_m). 
\end{equation}
Основной задачей при применении метода Ньютона является решение системы \eqref{Newton_eq} и нахождение $\Delta \x_m$. Выполнение шага метода Ньютона завешается вычисление нового приближения к решению $\x_{m+1} = \x_m + \Delta x_m$. 


\def\l{\mathbf{l}}
В случае решения задач гидродинамики размерность системы \eqref{Newton_eq} оказывается большой и может измеряться миллионами, её решение требует значительных вычислительных ресурсов. Более того, формирование матрицы Якоби в явном виде, как привило, невозможно вследствие большой вычислительной сложности задачи. На практике, для формирование матрицы Якоби пользуются тем фактом, что её произведение с произвольным вектором единичной длины $\l$ равно производной исходно функции $F$ вдоль этого направления
\begin{equation}
\pd{F}{\x} \l = \pd{F}{\l}. 
\end{equation}
При исследовании периодических решений производная функции $F$ может быть взята численно, как конечная разность, по формуле
\begin{equation}\label{fd_eq}
\pd{F}{\l} \approx \frac{F(\x + \varepsilon \l) - F(\x)}{\varepsilon},
\end{equation}
при достаточно малом значении $\varepsilon$. В численных расчетах значение $\varepsilon$ рекомендуют выбирать порядка $10^{-7}$. Вычисление матрицы Якоби сводится к вычислению производной функции $F$ вдоль каждого из базисных направлений, число которых равно числу неизвестных. В свою очередь, вычисление производной $F$ вдоль одного направления требует вызова функции $F$. В случае поиска периодических решений при вызове функции $F$ выполняется численное интегрирование исходных уравнений в течении периода по времени, что требует значительны вычислительных затрат, и выполнить его $O(N)$ раз на практике невозможно. 

Решить линейную систему \eqref{Newton_eq} позволяют итерационные методы, основанные на подпространствах Крылова. В этом случае обращение к матрице Якоби происходит только в форме её умножения на вектор, что следуя \eqref{fd_eq} сводится к вычислению конечно-разностной производной вдоль направления, задаваемого этим вектором. При решении системы вида 
\begin{equation}\label{Ax_eq}
Ax = b
\end{equation}
подпространство Крылова $K_i$ представляет собой линейную оболочку векторов
$$
K_i = L(b, Ab, A^2b, \dots, A^{i-1}b)
$$ 
Имея подпространство $K_i$, для того, чтобы построить подпространство $K_{i+1}$, необходимо выполнить только одно умножение матрицы $A$ на уже известный вектор $A^{i-1}b$. При решении системы \eqref{Ax_eq} решение ищется в базисе подпространства Крылова. Крыловские методы оказываются эффективны при поиске инвариантных решений с небольшим числом неустойчивых направлений (для решений на сепаратрисе оно одно). Для уточнения решения на порядок требуется только несколько десятков базисных векторов в подпространстве Крылова. Это связано с тем, что подпространства Крылова лучше приближают собственные вектора матрицы $A$, соответствующие наибольшему по модулю собственному числу, так как оператор $A^i$ с ростом $i$ растягивает такие собственные вектора наиболее сильно. При уточнении приближения к инвариантному решению в первую очередь необходимо разрешить неустойчивые направления, которые соответствуют собственным векторам с наибольшими собственными числами. 


Таким образом, уточнение решения на порядок требует вычисления функции $F(\x)$ порядка $O(10)$ раз, и не зависит от размерности пространства $\x$. 

В работе был реализован метод крыловского типа GMRES (generalized minimum residual method) \cite{Saad1986}. Также популярным является основанный на подпространствах Крылова метод BiCGSTAB (Biconjugate gradient stabilized method) \cite{Sleijpen1993}.




Учитывая симметрию отражения относительно плоскости $\theta = \pi/4$ со сдвигом на половину периода по времени $T/2$, критерием периодичности по времени решения может быть следующий
\begin{equation}
\v_p(x,r,\pi/2 + \theta,t) = \v_p(x + c_p T/2,r,\pi/2 - \theta,t+T/2)
\end{equation}




