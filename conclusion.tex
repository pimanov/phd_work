\addcontentsline{toc}{chapter}{Заключение}
\chapter*{Заключение} 

%В заключении диссертации излагаются 1) итоги выполненного исследования, 2) выводы, 3) рекомендации, 4) перспективы дальнейшей разработки темы.

В геометрии течения в круглой трубе рассчитан модельный порыв --- условно периодическое решение уравнений Навье-Стокса с пространственно-локализованной структурой, являющееся предельным состоянием решения, эволюционирующего на сепаратрисе, разделяющей в фазовом пространстве области притяжения решений, соответствующих ламинарному и турбулентному режимам течения. Описана внутренняя структура модельного порыва и его основные характеристики. Показано, что модельный порыв воспроизводит ряд качественных особенностей турбулентных порывов, наблюдаемых в круглых трубах при переходных числах Рейнольдса. Также, как в турбулентном порыве, в модельном порыве можно выделить вытянутые вдоль потока полосы повышенной и пониженной скорости, но если в первом случае эти полосы перемещаются вдоль стенки и их сплошность разрываются флуктуирующей составляющей движения, то во втором случае они сохраняют свое положение в пространстве и подвержены лишь небольшим колебаниям. Простое временное поведение модельного порыва позволило выполнить его детальное исследование. Полученные при исследовании модельного порыва результаты, мы полагаем, будут полезны для понимания турбулентного порыва. 

Установить универсальность наблюдаемых в модельном порыве закономерностей движения позволяет анализ других инвариантных решений Навье-Стокса, рассчитанных в работе. Методом продолжения по параметру рассчитано соответствующее модельному порыву семейство условно периодических решений уравнений Навье-Стокса с пространственно-локализованной структурой. В частности, найдены решения, оказывающиеся по ряду качественных характеристик ближе к турбулентному порыву, чем модельный порыв. Можно ожидать, что выводы, сделанные на основе исследования модельного порыва и этих решений имеют большее отношение к турбулентному порыву, чем выводы, сделанные при исследовании только модельного порыва. Также в геометрии течения в круглой трубе и течения в плоском канале найдено три семейства решений, имеющих вид бегущей волны. Анализ всех найденных решений позволяет сформулировать идеализированный цикл поддержания колебаний в такого рода решениях. 

Поле скорости каждого решения может быть представлено в виде суммы средней и пульсационной составляющих. Во всех исследованных решениях в среднем течении существуют вытянутые вдоль потока полосы повышенной и пониженной скорости, чередующиеся в угловом (поперечном) направлении. В случае решения в виде бегущей волны среднее поле скорости не зависит от продольной координаты и, соответственно, полосы имеют неограниченную протяженность. В случае решений с пространственно-локализованной структурой полосы имеют ограниченную протяженность. Возбуждение пульсаций связано с линейной неустойчивостью среднего течения. Колебания оказываются сконцентрированы в промежуточных областях между соседними полосами повышенной и пониженной скорости. В этих областях находятся точки перегиба, если рассматривать среднее течение как функцию угловой (поперечной) координаты, что позволяет связать неустойчивость среднего течения с неустойчивость струйных течений с точками перегиба. За поддержание полос ответственны продольные вихри, перемещающие жидкость в нормальной к основному потоку плоскости. 

Существенным результатом работы является описание нелинейного механизма поддержания продольных вихрей. Показано, что во всех решениях продольные вихри образуются в результате нелинейного взаимодействия пульсаций продольной скорости и пульсаций продольной завихренности. Пульсации продольной завихренности в области расположения продольных вихрей образуются за счет сжатия и растяжения существующих вихревых трубок пульсациями продольной скорости, что обеспечивает необходимую для поддержания продольных вихрей согласованность фаз между этими пульсациями. Отметим, что продольные вихри образуются в областях возникновения пульсаций, между полосами повышенной и пониженной скорости, так как именно в этих областях пульсации имеют наибольшую амплитуду и средняя скорость жидкость совпадает с фазовой скоростью пульсаций, что необходимо для образования пульсаций продольной завихренности описанным механизмом. Таким образом, продольные вихри образуются в промежуточной области между полосами повышенной и пониженной скорости, оказываясь расположенными наиболее удачным образом для поддержания существования этих полос. 

Наиболее быстрорастущее решение линейной задачи устойчивости среднего течения также воспроизводит описанный механизм поддержания продольных вихрей, но только в том случае, если при анализе среднего течения на устойчивость учтена не только продольная но и поперечная компоненты средней скорости. Принято считать, что поперечная компонента движения, поддерживая угловую (поперечную) неоднородность среднего течения, не оказывает существенного влияния на характеристики устойчивости среднего течения. Мы видим, что учет поперечной компоненты среднего течения необходим для адекватного воспроизведения механизма поддержания колебаний. 

Полосчатые структуры являются неотъемлемым элементом всех сценариев самоподдержания турбулентности в пристенных течениях, что говорит о вероятной близости выделенного в работе механизма с механизмом самоподдержания однородной (нелокализованной) пристенной турбулентности. На следующем этапе выполнения работы необходимо установить применимость сделанных выводов к турбулентному порву, и к более широкому классу пристенных турбулентных течений. Для этого, по-видимому, необходимо разработать метод промежуточного осреднения, позволяющий выделить в реальном турбулентном течении крупномасштабные и мелкомасштабные структуры. Это позволит обобщить рассуждения, применяемые в работе, на такого рода течения. Имея представления о механизме поддержания колебаний можно предложить эффективные стратегии управления турбулентными течениями с целью снижения или увеличения интенсивности колебаний. Эти стратегии также могут быть опробованы на реальных пристенных турбулентных течениях (численно), и в зависимости от их эффективности могут быть сделаны выводы о роли выделенных механизмов в поддержании такого рода режимов течения. В случае, если разработанные стратегии управления турбулентными потоками покажут себя эффективными, они имеют собственную значительную ценность. 






