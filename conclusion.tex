\addcontentsline{toc}{chapter}{Заключение}
\chapter*{Заключение} 

Общепринято, что непременным атрибутом цикла самоподдержания турбулентности в пристенных течениях являются полосчатые структуры --- долгоживущие конечноамплитудные образования, возникающие в пристенных областях. Рассмотренные в данной работе пространственно-локализованные турбулентные порывы также обладают полосчатыми структурами, что свидетельствует о вероятной близости механизма их самоподдержания с механизмом самоподдержания однородной (нелокализованной) пристенной турбулентности. Изучено численное решение уравнений Навье--Стокса, аппроксимирующее течение в турбулентном порыве. Это предельное решение на сепаратрисе, разделяющей области притяжения ламинарного и турбулентного решений. Простота временного поведения этого решения позволяет провести исчерпывающее исследование механизма его самоподдержания. На рассмотренном примере подтверждена определяющая роль полосчатых структур. Наглядно показан процесс образования вытянутых полос, включающий действие лифтап  эффекта на ограниченном отрезке длины с последующим конвективным растяжением вдоль стенки трубы. Наиболее важный результат проведенного исследования состоит в том, что обнаруженная неустойчивость полосчатого движения противоречит общепринятой точке зрения, согласно которой доминирующей является неустойчивость Кельвина--Гельмгольца, возникающая в пристенных областях полос замедленного движения. В рассмотренном примере уровень колебаний в области полос замедления минимален. Генерация колебаний происходит в промежуточной области между полосами на фоне резкого изменения скорости вдоль угловой координаты. Для выяснения деталей механизма обнаруженной неустойчивости планируется более подробное исследование. Вопрос о степени приложимости сделанных выводов к течению в турбулентном порыве и другим турбулентным течениям остается за рамками данного исследования и находится в настоящее время в стадии изучения.
