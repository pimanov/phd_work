\addcontentsline{toc}{chapter}{Заключение}
\chapter*{Заключение} 


Численное исследование решений уравнений Навье-Стокса, имеющих простое временное поведение, позволило выделить механизм поддержания колебаний, основные элементы которого оказываются общими для всех исследованных решений. В работе исследовано семейство условно периодических решений уравнений Навье-Стокса в геометрии течения в круглой трубе, имеющих пространственно локализованную структуру, и несколько семейств решений уравнений Навье-Стокса в геометриях течения в круглой трубе и течения в плоском канале, имеющих вид трехмерной бегущей волны. Поле скорости каждого решения может быть представлено в виде суммы средней и пульсационной составляющих. Во всех решениях существуют вытянутые вдоль потока области, в которых средняя скорость жидкости существенно выше или ниже среднего значения --- так называемые полосы повышенной и пониженной скорости. В случае решения в виде бегущей волны среднее поле скорости не зависит от продольной координаты и, соответственно, полосы имеют неограниченную протяженность. В случае решений с пространственно-локализованной структурой полосы имеют ограниченную протяженность. Генерация колебаний происходит в результате линейной неустойчивости среднего течения в промежуточной области между полосами на фоне резкого изменения скорость в поперечном направлении. За поддержание полос ответственны продольные вихри, перемещающие жидкость в нормальной к основному потоку плоскости. 

Существенным результатом работы является описание нелинейного механизма поддержания продольных вихрей. Показано, что во всех решениях продольные вихри образуются в результате нелинейного взаимодействия пульсаций продольной скорости и пульсаций продольной завихренности. Пульсации продольной завихренности образуются за счет сжатия и растяжения существующих вихревых трубок пульсациями продольной скорости, что обеспечивает необходимую для поддержания продольных вихрей согласованность фаз между этими пульсациями. Отметим, что продольные вихри образуются в области возникновения пульсаций, между полосами повышенной и пониженной скорости, так как именно в этой области пульсации имеют наибольшую амплитуду и средняя скорость жидкость совпадает с фазовой скоростью пульсаций, что необходимо для образования пульсаций продольной завихренности описанным механизмом. Таким образом, продольные вихри образуются в промежуточной области между полосами повышенной и пониженной скорости, оказываясь расположенными наиболее удачным образом для поддержания существования этих полос. 

Полосчатые структуры являются неотъемлемым элементом всех сценариев самоподдержания турбулентности в пристенных течениях, что говорит о вероятной близости выделенного в работе механизма с механизмом самоподдержания однородной (нелокализованной) пристенной турбулентности. Вопрос о степени приложимости сделанных в работе выводов к течению в турбулентном порыве и другим турбулентным течениям остается за рамками данного исследования и находится в настоящее время в стадии изучения.
