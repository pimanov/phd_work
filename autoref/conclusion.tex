
\section*{\centering Основные результаты и выводы}
\addcontentsline{toc}{chapter}{Основные результаты и выводы}

\noindent $1.$ Определены основные элементы механизма поддержания колебаний в модельном порыве --- условно периодическом решении уравнений Навье-Стокса с пространственно-локализованной структурой. Поле скорости решения может быть представлено в виде суперпозиции средней и пульсационной составляющих. Характерной особенностью среднего течения являются вытянутые вдоль потока полосы повышенной и пониженной скорости. Пульсации возникают в результате линейной неустойчивости среднего течения в областях между соседними полосами на фоне резкого изменения скорости вдоль угловой координаты. Нелинейное взаимодействие пульсаций приводит к формированию продольных вихрей, поддерживающих существование полос.

\noindent $2.$ Показано, что продольная неоднородность среднего течения в модельном порыве не оказывает существенного влияния на формирование пульсаций. 

\noindent $3.$ Обнаружен нелинейный механизм поддержания продольных вихрей, вызывающих полосчатое искажение в распределении продольной скорости. Существование продольных вихрей поддерживается нелинейным взаимодействием пульсаций продольной скорости и пульсаций продольной завихренности. Пульсации продольной завихренности образуются за счет сжатия и растяжения существующих в среднем течении вихревых трубок пульсациями продольной скорости, что обеспечивает необходимую для поддержания продольных вихрей согласованность фаз между этими пульсациями. 

\noindent $4.$ Определены основные элементы механизма поддержания колебаний в семействе условно периодических решений с пространственно локализованной структурой, полученных на основе модельного порыва методом продолжения по параметру. Также определены основные элементы механизма поддержания колебаний в трех семействах решений, имеющих вид бегущей волны, описывающих течения в круглой трубе и в плоском канале. Механизм поддержания колебаний во всех исследованных решениях аналогичен найденному в модельном порыве, что позволяет говорит о некоторой универсальности найденного механизма. 
