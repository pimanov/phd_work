\nocite{Vest18, MZG2017, MZG2015, Kazan2015, KMU2016, KMU2015, KMU2014} %KMU2016, KMU17, Almanac2017, report
\nocite{Lomonosov2018, Lomonosov2017, Lomonosov2016, Lomonosov2015, Lomonosov2014, LomRead2017, LomRead2016, LomRead2014, NeZaTeGiUs2016, NeZaTeGiUs2014, Bur2016, Bur2014, Ob2018, ICMAR2016, Kazan2015conf}

\section*{\centering  Общая характеристика работы}
\addcontentsline{toc}{chapter}{Общая характеристика работы}

\textbf{Актуальность темы и степень ее разработанности.}
Изучение закономерностей движения жидкостей и газов в круглых трубах имеет большое значение как с практической, так и с теоретической точек зрения. Известно, что при небольших значениях числа Рейнольдса $\Re$ течение оказывается ламинарным, а при достаточно больших --- турбулентным. При переходных значениях $\Re$ ламинарный и турбулентный режимы течения могут сосуществовать, при этом участки возмущенного и спокойного движения следуют вдоль трубы друг за другом, практически не меняя своей протяженности. 

Экспериментально установлено (Wygnanski \& Champagne, 1973), что в разных условиях возникают локализованные турбулентные структуры заметно разных типов. Структуры первого типа --- турбулентные порывы --- появляются при сильной возмущенности потока на входе в трубу в диапазоне $2000<\Re<2700$. Порывы сносятся вниз по потоку со скоростью, близкой к средней скорости течения, практически не меняя свою протяженность. Для порыва характерны размытость переднего фронта, на котором скорость на оси трубы плавно уменьшается от ламинарного значения на 30 -- 40\%, и резкость заднего фронта, на котором происходит возвращение к ламинарному течению. Структуры другого типа --- турбулентные пробки --- появляются при $\Re>3200$ только когда возмущенность потока на входе недостаточна для непосредственного возникновения турбулентности. Тогда возможен переход через турбулентные пробки --- локализованные образования, расширяющиеся по мере сноса вниз по течению. 

В последние годы выполнен ряд подробных экспериментальных и численных исследований характеристик и свойств турбулентных порывов. Установлено, что турбулентный порыв является нестабильным образованием, склонным либо к исчезновению, либо к делению. С каждой из двух конкурирующих тенденций связано характерное время: среднее время жизни порыва до его исчезновения и среднее время до его разделения. Первое увеличивается с ростом $\Re$, второе уменьшается. Согласно (Avila et al., 2011), значение $\Re^*=2040$, при котором происходит смена доминирования тенденций, является точкой статистического фазового перехода и может быть принята в качестве минимального критического числа Рейнольдса в круглой трубе. 

Турбулентный порыв представляет собой интересный гидродинамический объект, который в некотором отношении может рассматриваться как структурная единица турбулентности. %Можно сформулировать ряд вопросов, касающихся его поведения. До конца не понятен механизм, обуславливающий пространственную локализацию и самоподдержание порыва, неясны причины, побуждающие его к делению или затуханию, неизвестны факторы, определяющие его протяженность и скорость перемещения вдоль трубы.
В последние годы акцент в изучении механизма самоподдержания турбулентности в пристенных течениях смещается от лабораторного эксперимента в сторону эксперимента вычислительного, основанного на численном решении уравнений Навье--Стокса. Турбулентные порывы впервые рассчитаны в (Priymak \& Miyazaki, 2004).
%, где показано, что пространственная локализация является внутренним свойством решений уравнений Навье--Стокса при переходных числах Рейнольдса, а не является следствием специальных начальных условий. 
Попытка объяснения механизма самоподдержания турбулентного порыва предпринята в (Shimizu \& Kida, 2009). Хотя идеализированная схема самоподдержания порыва, предложенная в этой работе, выглядит вполне правдоподобно, сделанные выводы на наш взгляд не подкреплены фактическими данными в должной мере. 

Реальная динамика порыва сложна и неопределенна. Ее изучение осложнено в первую очередь стохастичностью процесса, когда отдельные его фазы следуют друг за другом случайным образом. В этих условиях определенная ясность может быть получена из анализа более простых структур, аппроксимирующих порыв, недавно найденных в (Avila et al., 2013). Это предельные решения, возникающие на сепаратрисе, разделяющей в фазовом пространстве области притяжения решений, соответствующих ламинарному и турбулентному режимам течения. Такие решения, наследуя ряд качественных характеристик турбулентного порыва, оказываются периодическими по времени в системе отсчета, перемещающейся вдоль трубы с постоянной скоростью. Мы будем называть такие структуры {\it модельными порывами}. Простота поведения позволяет провести исчерпывающее исследование свойств модельного порыва, которые, как мы полагаем, проясняют определенные детали поведения турбулентного порыва. 

Методом продолжения по параметру могут быть получены другие условно периодические решения уравнений Навье-Стокса (периодические в подходящей подвижной системе отчета), имеющие пространственно-локализованную структуру. Также, в настоящее время известно достаточно большое число решений, имеющих вид трехмерной бегущей волны (периодичных вдоль потока и стационарных в подходящей подвижной системе отсчета) (Kawahara et al., 2012). Такие решения также допускаю детальное исследование и могут быть использованы для установления универсальности выделенных при исследовании модельного порыва закономерностей. 

{\bf Цель диссертационной работы} состоит в выявлении механизмов, ответственных за возникновение и поддержание турбулентных порывов. 
С этой целью поставлены и решены следующие \textbf{задачи}: 

\noindent $1.$ Проведено численное исследование модельного порыва --- условно периодического решения уравнений Навье-Стокса в геометрии течения в круглой трубе, имеющего пространственно-локализованную структуру. %Воспроизводя ряд качественных особенностей турбулентного порыва, модельный порыв имеет более простое временное поведение, что позволяет выполнить его детальное исследование. 

\noindent $2.$ Методом продолжения по параметру рассчитаны другие условно периодические решения уравнений Навье-Стокса в геометрии течения в круглой трубе, имеющие пространственно-локализованную структуру. Выполнен их анализ. %Их анализ позволил оценить универсальность найденных при исследовании модельного порыва закономерностей. 

\noindent $3.$ Рассчитаны и исследованы трехмерные решения уравнений Навье-Стокса в геометрии течения в круглой трубе и геометрии течения в плоском канале, имеющие вид бегущей волны. %Анализ таких решений также позволил оценить универсальность найденных при изучении модельного порыва закономерностей. 


{\bf Научная новизна и положения, выносимые на защиту.} На защиту выносятся следующие новые результаты, полученные в диссертации:

\noindent $1.$ Определены основные элементы механизма поддержания колебаний в модельном порыве --- условно периодическом решении уравнений Навье-Стокса с пространственно-локализованной структурой. Поле скорости решения может быть представлено в виде суперпозиции средней и пульсационной составляющих. Характерной особенностью среднего течения являются вытянутые вдоль потока полосы повышенной и пониженной скорости. Пульсации возникают в результате линейной неустойчивости среднего течения в областях между соседними полосами на фоне резкого изменения скорости вдоль угловой координаты. Нелинейное взаимодействие пульсаций приводит к формированию продольных вихрей, поддерживающих существование полос.

\noindent $2.$ Показано, что в модельном порыве продольная неоднородность среднего течения не оказывает существенного влияния на формирование пульсаций. Их образование связано с неоднородностью в поперечной плоскости.

\noindent $3.$ Обнаружен нелинейный механизм поддержания продольных вихрей, вызывающих полосчатое искажение в распределении продольной скорости. Существование продольных вихрей поддерживается нелинейным взаимодействием пульсаций продольной скорости и пульсаций продольной завихренности. Пульсации продольной завихренности образуются за счет сжатия и растяжения существующих в среднем течении вихревых трубок пульсациями продольной скорости, что обеспечивает необходимую для поддержания продольных вихрей согласованность фаз между этими пульсациями. 

\noindent $4.$ Определены основные элементы механизма поддержания колебаний в условно-периодических решениях уравнений Навье-Стокса с пространственно локализованной структурой, полученных продолжением по параметру решения, соответствующего модельному порыву. Также механизм поддержания колебаний определен в нескольких семействах решений, имеющих вид бегущей волны, описывающих течения в круглой трубе и в плоском канале. Во всех исследованных решениях механизм поддержания колебаний аналогичен найденному при исследовании модельного порыва, что в некоторой степени подтверждает универсальность этого механизма. 

{\bf Теоретическая и практическая значимость полученных результатов.}
Полосы повышенной и пониженной скорости являются неотъемлемым элементом всех сценариев поддержания колебаний в пристенных турбулентных течениях, что дает основания полагать, что полученные в работе представления о механизме поддержания колебаний в модельных течениях могут быть обобщены на этот класс течений. Понимание механизмов поддержания колебаний имеет первостепенное значения для предсказания характеристик пристенных турбулентных течений и разработки эффективных методов управления ими.

\textbf{Метод исследования и достоверность результатов.}
В работе движение жидкости воспроизводится численно, путем решения полных трехмерных уравнений Навье-Стокса. Численный метод совмещает конечно-разностную аппроксимацию второго порядка точности по пространственным переменным и полунеявный метод Рунге-Кутты третьего порядка точности интегрирования по времени (Nikitin, J. Comp. Phys. 2006). Качество программной реализации численного метода и его адекватность целям работы подтверждают результаты моделирования турбулентного течения в трубе при переходных значениях числа Рейнольдса, приведенные в диссертации. Решения на сепаратрисе и, в частности, модельный порыв найдены методом пристрелки (Avila et al., 2013). Метод продолжения по параметру основан на применении метода Ньютона. %Приближение к решению возникающей на каждой итерации метода Ньютона линейной системы ищется в подпространствах Крылова методом минимизации невязки.

Для подтверждения корректности результатов численных расчетов, все исследованные решения найдены на нескольких расчетных сетках. Где возможно, выполнено сравнение с экспериментальными данными и результатами расчетов других авторов. Все сравнения подтверждают, что найденные численные решения соответствуют решениям уравнений Навье-Стокса и отражают физику явления. Проведенный в работе анализ нескольких семейств решений подтверждает универсальность выделенных в работе закономерностей. %Расчеты выполнены с привлечением ресурсов суперкомпьютерного комплекса МГУ им. М.В.\,Ломоносова.

\begin{comment}
{\bf Положения, выносимые на защиту.} Анализ модельных течений позволяет предложить следующий идеализированный цикл поддержания колебаний:

\noindent $1.$ В исследованных решениях поле скорости может быть представлено в виде суммы средней и пульсационной составляющих. Существенной особенностью среднего течения является наличие вытянутых вдоль потока полос повышенной и пониженной скорости. Пульсации возникают в результате линейной неустойчивости среднего течения в областях, расположенных между соседними полосами повышенной и пониженной скорости. В этих областях распределение средней продольной скорости имеет точки перегиба, если рассматривать его как функцию угловой (поперечной) координаты, что позволяет связать образование колебаний с неустойчивостью струйного течения с точками перегиба. 

\noindent $2.$ За поддержание полос повышенной и пониженной скорости ответственны продольные вихри, перемещающие жидкость в нормальной к основному потоку плоскости. Там, где медленная жидкость перемещается от стенки в основной поток, образуются полосы пониженной скорости. В промежуточных областях образуются полосы повышенной скорости. 

\noindent $3.$ Механизм поддержания продольных вихрей состоит в нелинейном взаимодействии пульсаций продольной скорости и пульсаций продольной завихренности. При этом в области расположения продольных вихрей пульсации продольной завихренности образуются в результате сжатия и растяжения существующих в потоке вихревых трубок пульсациями продольной скорости, что обеспечивает необходимую для поддержания продольных вихрей согласованность фаз между этими пульсациями. 
%\end{itemize}
\end{comment}

{\bf Апробация результатов.} Основные результаты работы представлены на следующих конференциях: 
Конференции-конкурсе молодых ученых НИИ механики МГУ имени М.В.\,Ломоносова (2014, 2015, 2016, 2017 годы),
Международной научной конференции студентов, аспирантов и молодых ученых <<Ломоносов>> (МГУ им. М.В.\,Ломоносова, Москва, 2014, 2015, 2016, 2017, 2018 годы), 
Конференции <<Ломоносовские чтения>> (НИИ механики, МГУ им. М.В.\,Ломоносова, 2014, 2015, 2016, 2017, 2018 годы),
Международной конференции <<Нелинейные задачи теории гидродинамической устойчивости и турбулентность>> (Звенигород, 2014, 2016, 2018 годы),
Школе-семинаре <<Современные проблемы аэрогидродинамики>> (Сочи, 2014, 2016 годы),
XI Всероссийском съезде по фундаментальным проблемам теоретической и прикладной механики (Казань, 2015 год),
7th International Symposium on Bifurcations and Instabilities in Fluid Dynamics (Paris, 2015),
15th European Turbulence Conference (Delft, Netherlands, 2015),
18th International Conference on the Methods of Aerophysical Research (Пермь, 2016 год),
Международной конференции <<Турбулентность, динамика атмосферы и климата>> (Москва, 2018 год).

Также результаты работы представлены на научных семинарах: 
Семинаре НИИ механики МГУ по механике сплошных сред под руководством А.\,Г.\,Куликовского, В.\,П.\,Карликова и О.\,Э.\,Мельника, 
Семинаре кафедры газовой и волновой динамики мехмата МГУ им. М.\,В.\,Ломоносова под руководством Р.\,И.\,Нигматулина,
Семинаре <<Суперкомпьютерные технологии в науке, образовании и промышленности>> на базе научно-образовательного центра <<Суперкомпьютерные технологии>> под руководством В.\,А.\,Садовничего,
Астрофизическом семинаре отдела теоретической физики ФИАН им. П.\,Н.\,Лебедева под руководством А.\,В.\,Гуревича.
%Семинаре лаборатории общей аэродинамики НИИ механики МГУ под руководством Н.\,В.\,Никитина.

%Работа отмечена следующими наградами: 
%диплом 2-ой степени конференции--конкурса молодых ученых НИИ механики МГУ 2014 года,
%диплом 1-ой степени за лучшую работу аспиранта и диплом 3-ей степени конференции-конкурса молодых ученых НИИ механики МГУ 2015 года,
%диплом 1-ой степени за лучшую работу аспиранта и диплом 3-ей степени конференции-конкурса молодых учёных НИИ механики МГУ 2016 года,
%вторая премия конкурса молодых научных сотрудников МГУ~им.~М.\,В.\,Ломоносова 2015 года,
%грамота за лучший доклад на международной научной конференции студентов, аспирантов и молодых ученых <<Ломоносов>> 2015, 2017 и 2018 годов.


\textbf{Публикации.} 
По материалом диссертации опубликовано 4 статьи в научных журналах, 3 статьи в сборниках трудов и 15 тезисов докладов. Из них 3 статьи опубликованы в рецензируемых научных изданиях, индексируемых в базах данных Web of Science и Scopus, одна статья --- в научном журнале, входящем в перечень изданий, рекомендованных ВАК при Министерстве образования и науки РФ. 

{\bf Личный вклад автора.} 
Все численные эксперименты и анализ результатов расчетов, приведенные в диссертации, выполнены автором лично. Также автором написан пакет программ на языке Python, реализующий алгоритм поиска решения на сепаратрисе и метод Ньютона-Крылова для поиска условно периодических решений уравнений Навье-Стокса. Подготовка к публикации полученных результатов проводилась вместе с соавторами, причём вклад диссертанта был определяющим. Основная работа по подготовке текста диссертации и иллюстративного материала также принадлежит диссертанту. 

{\bf Структура и объем диссертации.} 
Диссертация состоит из введения, четырех глав, заключения и списка литературы. Общий объем диссертации составляет 124 страницы, включая 43 рисунка. Список литературы содержит 92 пункта. 

%В автореферате диссертации излагаются 1) положения, выносимые на защиту, 2) основные идеи и выводы диссертации, 3) показываются вклад автора в проведенное исследование, 4) степень новизны и практическая значимость результатов исследования, 4) содержатся сведения об организации, в которой выполнялась диссертация, об оппонентах, о научных руководителях и научных консультантах соискателя ученой степени (при наличии), 5) приводится список публикаций автора диссертации, в которых отражены основные научные результаты диссертации. 


