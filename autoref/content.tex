\section*{\centering Содержание работы}
\addcontentsline{toc}{chapter}{Содержание работы}

Во \textbf{введении} определены актуальность избранной темы, степень ее разработанности, цели и задачи диссертационной работы, ее научная новизна, теоретическая и практическая значимость, методология диссертационного исследования, положения, выносимые на защиту, степень достоверности полученных результатов, апробация результатов и личный вклад автора. 

В {\bf главе 1} описан метод исследования. В работе движение жидкости моделируется численными решениями полных трехмерных уравнений Навье-Стокса. Существенным достоинством численных методов при изучении механизмов турбулентности является то, что расчет дает полную информацию о течении. Подробный численный анализ ряда пристенных течений показал, что численные решения уравнений Навье-Стокс с высокой точностью воспроизводят наблюдаемые в экспериментах особенности движения. Прямое численное моделирование зарекомендовало себя эффективным методом изучения пристенных турбулентных течений. 

Во \textbf{введении к главе 1} приведен краткий историко-литературный обзор развития численных методов для решения задач гидродинамики и краткий обзор подходов к прямому численному моделированию пристенных турбулентных течений. 

В \textbf{разделе 1.1} приведена постановка задачи. В работе рассматривается движение вязкой несжимаемой жидкости в прямой трубе круглого сечения. Движение жидкости описывается уравнениями Навье-Стокса и неразрывности:
$$
\pd{\v}{t} = - (\v \cdot \nabla) \v - \frac{1}{\rho}\grad p + \nu \nabla^2 \v,
$$
$$
\nabla \cdot \v = 0,
$$
где $\v$ --- поле скорости, $p$ --- давление, $\rho$ и $\nu$ --- постоянные плотность жидкости и кинематический коэффициент вязкости, $t$ --- время. 
На стенках трубы, имеющей радиус $R$, ставится условие прилипания. В продольном направлении на поле скорости накладывается условие периодичности с периодом $L_x$. Жидкость приводится в движение за счет внешнего градиента давления, который определяется из условия постоянства средней скорости $U_m$. 

Задача решается в безразмерных переменных. В качестве основных единиц измерения выступают радиус трубы $R$, максимальная скорость течения Пуазейля $U = 2U_m$ и плотность жидкость $\rho$. Безразмерным параметром системы является число Рейнольдса $ \Re = {R U}/{\nu}$.

Постановка задачи традиционна для прямого расчета развитых турбулентных течений в трубах и каналах. В такой постановке удается воспроизводить характеристики течения, устанавливающегося на большом удалении от входа в трубу, решая уравнения движения в ограниченной расчетной области. Условие периодичности вдоль трубы освобождает от необходимости устанавливать условия на входе и выходе из трубы. В тоже время, увеличивая длину периода $L_x$, можно минимизировать влияние этого условия на поток.

В \textbf{разделе 1.2} описан конечно-разностный метод решения поставленной задачи. В расчетной области вводится структурированная сетка, ребра которой совпадают с координатными линиями цилиндрической системы координат $(x,r,\theta)$. По координатам $x$ и $\theta$ сетка однородна. В нормальном к стенке направлении вводится растяжение сетки, что позволяет сгустить сетку вблизи стенки, где наблюдаются наибольшие градиенты скорости. Дискретизация уравнений выполняется на так называемых смещенных сетках: различные скалярные величины отнесены к различным точкам сетки. Дискретизация по пространственным переменным выполнена со вторым порядком точности. Для интегрирования по времени применен полунеявный метод Рунге-Кутты третьего порядка точности. 

Дискретная система уравнений обладает рядом важных консервативных свойств. Дивергенция поля завихренности в расчете тождественно равна нулю. Дискретный аналог ротора градиента давления тождественно равен нулю, что гарантирует отсутствие влияния давления на эволюцию поля завихренности напрямую. Дискретный аналог дивергенции вязкого слагаемого тождественно равна нулю, что гарантирует отсутствие производства массы, вызванного этим слагаемым. Градиент периодической составляющей давления и нелинейные слагаемые не производят кинетической энергии. 
Наиболее полно метод описан в работе (Nikitin N., J. Comp. Phys., 2006, 217(2)).

В \textbf{разделе 1.3} сказано о реализации численного метода и методике проведения численных экспериментов. Пакет программ, реализующих численный метод, написан на языке программирования Fortran77. Помимо последовательного варианта программы реализован параллельный, позволяющий выполнять расчеты на кластерных вычислительных системах с распределенной памятью. Для коммуникации между процессами использован интерфейс передачи сообщений MPI (Message Passing Interface). Работа выполнена с использованием оборудования Центра коллективного пользования сверхвысокопроизводительными вычислительными ресурсами МГУ имени М.В.\,Ломоносова. Значительная часть кода, необходимого при анализе полученных результатов и управлении численными экспериментами, в том числе реализующая метод продолжения по параметру, была написана на высокоуровневом языке Python. 

В \textbf{разделе 1.4} приведены результаты расчетов движения жидкости в круглой трубе в диапазоне $1670 \leqslant \Re \leqslant 2800$ в достаточно протяженной расчетной области для того, чтобы воспроизвести явление пространственной локализации турбулентности. Турбулентность в расчетах принимает форму локализованных структур, характеристики которых совпадают с характеристиками турбулентных порывов, приведенными в литературе. Это подтверждает адекватность численного метода целям работы и качество его программной реализации. 




