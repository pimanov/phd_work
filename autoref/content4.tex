В \textbf{главе 4} поднимается вопрос об универсальности полученных при исследовании модельного порыва результатов. Для ответа на него рассчитаны и исследованы отличные от модельного порыва решения уравнений Навье-Стокса. 
%В частности, исследованы условно-периодические решения уравнений Навье-Стокса с пространственно локализованной структурой, полученные продолжением решения, соответствующего модельному порыву, по параметру. Также исследовано три семейства решений, имеющих вид бегущей волны. Одно описывает течение в круглой трубе, два других --- в плоском канале. Все исследованные решения воспроизводят общий механизм поддержания колебаний, что подкрепляет представление о его универсальности.

В \textbf{разделе 4.1} описан метод Ньютона для нахождения условно-периодических решений уравнений Навье-Стокса и решений в виде бегущей волны, как частный случай условно-периодических. Приближение к решению возникающей на каждой итерации метода Ньютона линейной системы ищется в подпространствах Крылова методом минимизации невязки. 
%Поиск приближения к решению в подпространствах Крылова позволяет существенно снизить требования к вычислительным ресурсам, необходимым для применения метода Ньютона. 
На применении метода Ньютона основан метод продолжения по параметру. Когда одно решение известно, оно используется в качестве начального приближения к решению при близком значении параметров, с которыми метод сходится. Найденное решение используется в качестве начального приближения к новому решению, и т.д. Так строится цепочка решений, связывающая решения с существенно различными значениями параметров --- решение продолжается по параметру.


В \textbf{разделе 4.2} приведены результаты продолжения модельного порыва по $\Re$, позволившего получить новые условно периодические решения уравнений Навье-Стокса с пространственно локализованной структурой. Решения принадлежат однопараметрическому множеству. Значения амплитуды трехмерной составляющей движения $a$ и скорости перемещения вдоль трубы $c$ как функции $\Re$ приведены на Рисунке \ref{contin_pic}. Жирные точки на графиках соответствуют исходному решению. При $\Re \approx 1400$ обнаружена точка бифуркации, в которой рождается две ветви решений. При каждом $\Re$, превышающем критическое значение, существует два решения, каждое из которых принадлежит свой ветви. 
% При меньших $\Re$ таких решений не существует. При больших $\Re$ существует две ветви решений, то есть при каждом значении $\Re$ существует два решения, каждое из которых принадлежит свой ветви. 
Ветвь, которой принадлежит исходное решение, называют нижней. Вторую ветвь называют верхней. Для верхней ветви характерна большая интенсивность пульсаций и меньшая скорость перемещения вдоль трубы. По этим и другим параметрам решения с верхней ветви оказывается ближе к турбулентному порву, чем исходное решение. Результаты согласуются с (Avila et al. 2013). 


\begin{figure}
\center{\includegraphics[width=0.9\linewidth]{autoref_contin.png}}
\caption{Продолжение модельного порыва по числу Рейнольдса}
\label{contin_pic}
\end{figure} 

В \textbf{разделе 4.3} выполнено исследование верхней ветви порожденного модельным порывом семейства условно-периодических решений. На Рисунке \ref{3D_ub_pic} приведена визуализация мгновенного поля скорости решения с верхней ветви. Несмотря на существенные количественные отличия, решения с верхней ветви воспроизводят тот же механизм поддержания колебаний, что и решение с нижней ветви. Поле скорости решения представляется в виде суперпозиции средней и пульсационной составляющих. Области повышенной и пониженной средней скорости представляют собой вытянутые вдоль потока полосы. На Рисунке \ref{ub_cs_pic}(a) приведены изолинии средней продольной скорости в поперечном сечении трубы, в котором пульсации имеют существенную амплитуду. В центральной части расчетной области, где изолинии находятся на большем удалении от стенки, проходит полоса пониженной скорости. При больших и меньших значениях $\theta$ находятся полосы повышенной скорости. Пульсации возникают в результате линейной неустойчивости среднего течения между соседними полосами повышенной и пониженной скорости. Угловую неоднородность среднего течения поддерживают продольные вихри, которым соответствуют области повышенных и пониженных значений средней продольной завихренности $\Omega_x$. Распределение $\Omega^2_x$ в том же сечении трубы приведено на Рисунке \ref{ub_cs_pic}(б). Определяющий вклад в производство $\Omega_x$ в уравнении \eqref{OX_eq} дают слагаемые \eqref{OXgen_terms}. Вклад слагаемых, соответствующих \eqref{OXgen_terms}, в производство $\Omega^2_x$ приведен на Рисунке \ref{ub_cs_pic}(в). Механизм образования пульсаций продольной завихренности в области формирования продольных вихрей также аналогичен выделенному в модельном порыве. 


\begin{figure}
\center{\includegraphics[width=0.9\linewidth]{autoref_3D_ub.png}}
\caption{Условно-периодическое решение с верхней ветви}
\label{3D_ub_pic}
\end{figure} 

\begin{figure}
\center{\includegraphics[width=1\linewidth]{ub_cs.png}}
\caption{Поле скорости решения с верхней ветви}
\label{ub_cs_pic}
\end{figure}

\textbf{Разделы 4.4} и \textbf{4.5} посвящены анализу трехмерных бегущих волн в течении Гагена-Пуазейля и в плоском течении Пуазейля. Постановка задачи для плоского течения Пуазейля аналогична постановке для течения Гагена-Пуазейля. В каждом из двух течения вид бегущей волны имеют предельные решения на сепаратрисе, найденные в непротяженной расчетной области. Также в плоском течении Пуазейля найдена устойчивая бегущая волна (при наложенных условиях симметрии). Методом продолжения по параметру рассчитаны соответствующие семейства решений. В отличии от модельного порыва, среднее поле скорости бегущей волны не зависит от продольной координаты --- полосы и продольные вихри имеют бесконечную протяженность. Решения достаточно существенно отличаются друг от друга, но, несмотря на это, механизм поддержания колебаний во всех трех семействах решений также аналогичен механизму поддержания колебаний в модельном порыве. 
%Отметим, что в некоторых решениях среднее поле скорости устойчиво к малым возмущениям, но и в этих случаях механизм передачи энергии в пульсационную составляющую движения, по-видимому, линейный, так как наиболее медленно затухающая собственная функция линейной задачи устойчивости повторяет форму и фазовую скорость пульсационной составляющей движения. Также в некоторых решениях слагаемое \eqref{ox1gen_main_terms} оказывается не единственным ответственным за производство $\omega'_x$ в области формирования продольных вихрей, но это слагаемое всегда имеет существенное значение и именно оно обеспечивает необходимую для поддержания продольных вихрей согласованность фаз между пульсациями $\omega'_x$ и $v'_x$. 
 
В \textbf{разделе 4.6} приведены выводы по главе. Основные результаты главы опубликованы в работах автора диссертации \cite{Vest18, KMU2016, Lomonosov2018, Lomonosov2017, LomRead2017, LomRead2016, Ob2018}. 


